\chapter{Groupes}
\label{cha:groupes}

Dans ce chapitre on s'intéresse à l'étude des groupes.

 \begin{definition}
    \label{def:49}
    Soit $(G,+)$ un groupe alors $H\subset G$ est un sous groupe si $(H,+|_{ H })$ est un groupe.
 \end{definition}
 
 \begin{lemma}
    \label{lem:23}
    $(H,\times|_H)$ est un sous groupe de $(G,\times) \Leftrightarrow \forall a,b \in H\quad a\times b^{-1} \in H$
   \end{lemma}
   
    \begin{definition}
    \label{def:49}
    $H\le G$ est un sous groupe normal de $G$ ssi $\forall g \in G\quad Hg=gH$ On note $H\unlhd G$.
     \end{definition}
     
      \begin{definition}
    \label{def:49}
    Soient $g,g' \in G$ alors on pose $Hg \circ Hg' := H(gg')$.
     \end{definition}
     
     \begin{theorem}
    \label{thr:28}
    $\circ$  munit $G/\!\raisebox{-.65ex}{\ensuremath{{H}}}  := \left\{ gH,\quad g\in G \right\}$ d'une structure de groupe ssi $H\unlhd G$.
    \end{theorem}
    
       \begin{definition}
    \label{def:49}
    Soient $(G,+),(G',\times)$ deux groupes et $\phi :G\rightarrow G'$ une application alors $\phi$ est un homomorphisme ssi 
    $\forall a,b \in G, \phi(a+b)=\phi(a) \times \phi(b)$.
    \end{definition}
    
    \begin{remark}
    $ker(\phi) \unlhd G$.
    \end{remark}
    
     \begin{theorem}
    \label{thr:28}
    Si $\phi$ est surjective alors $G/\!\raisebox{-.65ex}{\ensuremath{{ker(\phi)}}} \cong Im(\phi)$.
     \end{theorem}
     
         \begin{definition}
    \label{def:49}
    Soit $(G,+)$ un groupe abélien. On dit que $(G,+)$ est engendré fini si $\exists g_1,\dots,g_n \in G\ $ tqs. $G= \left\{ x_1g_1+\dots+x_ng_n, x_i \in \mathbb{Z} \right\} $.
        \end{definition}
        
        
          \begin{definition}
    \label{def:49}
    Soient $(G_1,+), (G_2,\times)$ deux groupes alors $G_1\otimes G_2 :=(G_1\times G_2, \bullet )$ avec $\forall (g_1,g_2), (g'_1,g'_2) \in G_1\times G_2, (g_1,g_2)\bullet (g'_1,g'_2)= (g_1+g_2, g'_1 \times g'_2)$.
    \end{definition}
    
        \begin{remark}
         $G_1\otimes G_2$ est un groupe.
        \end{remark}
        
        
         \begin{lemma}
    \label{lem:23}
    Soit $\phi:G\rightarrow G$ un automorphisme et $H\unlhd G$ alors $ G/\!\raisebox{-.65ex}{\ensuremath{{H}}} \cong G/\!\raisebox{-.65ex}{\ensuremath{{ \phi (H)}}}$.
    \end{lemma}
        
          \begin{theorem}
    \label{thr:28}
    Soit $G$ un groupe abélien engendré fini alors $\exists d_1,\dots,d_k \in \mathbb{N}_{ \ge 1}$ et $l\in \mathbb{N}$ tq $G  \cong \mathbb{Z}_{d_1} \otimes \dots \otimes \mathbb{Z}_{d_k} \otimes \mathbb{Z}^l$ avec $\mathbb{Z}^l:=\mathbb{Z} \otimes \dots 
   \otimes \mathbb{Z}$ $l$ fois. De plus on a $d_1|\dots\ |d_k$.
    \end{theorem}
    
    \begin{proof}
    Soient $g_1,\dots, g_n \in G$ tqs. $G= \left\{ x_1g_1+\dots+x_ng_n, x_i \in \mathbb{Z} \right\} $ alors on peut vérifier que $\phi :\mathbb{ Z }^{ n }\rightarrow G, (x_ 1,\dots ,x_n)\mapsto x_1g_1+\dots +x_ng_n$ est un homomorphisme surjectif. Mais alors $ker(\phi) \le G$ et donc $\exists B\in \mathbb{Z}^{n\times k}$ avec $rang(B)=k$ et $\Lambda(B)=ker(\phi)$. En calculant la forme normale de Smith de B on obtient que $\exists U\in \mathbb{Z}^{n\times n}, V\in \mathbb{Z}^{k\times k}$ unimodulaires tqs.
$B=U\begin{pmatrix} d_{ 1 } & \quad  & \quad  \\ \quad  & \ddots & \quad  \\ \quad  & \quad  &d_k  \\  \quad & 0 &\quad   \end{pmatrix}V$ avec $1\le d_1|\dots|d_k$ et donc on a $ker(\phi)=\left\{ U\begin{pmatrix} d_{ 1 } & \quad  & \quad  \\ \quad  & \ddots & \quad  \\ \quad  & \quad  &d_k  \\  \quad & 0 &\quad   \end{pmatrix}y, y\in \mathbb{Z}^k \right\}$. Alors par les théorèmes explicités plus haut on obtient $G \cong {\mathbb{Z}^n}/\!\raisebox{-.65ex}{\ensuremath{{\left\{ U\begin{pmatrix} d_{ 1 } & \quad  & \quad  \\ \quad  & \ddots & \quad  \\ \quad  & \quad  &d_k  \\  \quad & 0 &\quad   \end{pmatrix}y, y\in \mathbb{Z}^k \right\}}}}$. 
Puis comme $U: \mathbb{Z}^n \rightarrow  \mathbb{Z}^n$ est un automorphisme on arrive finalement à$\\$ $G \cong {\mathbb{Z}^n}/\!\raisebox{-.65ex}{\ensuremath{{\left\{ (d_1y_1,\dots,d_ky_k,0,\dots,0), y_i \in \mathbb{Z} \right\}}}}$. Il suffit maintenant de constater que$\\$ ${\mathbb{Z}^n}/\!\raisebox{-.65ex}{\ensuremath{{\left\{ (d_1y_1,\dots,d_ky_k,0,\dots,0), y_i \in \mathbb{Z} \right\}}}} \cong \mathbb{Z}_{d_1} \otimes \dots \otimes \mathbb{Z}_{d_k} \otimes \mathbb{Z}^{n-k}$ .
    \end{proof} 
    
        

   
        
        
        
        
        
        