\documentclass[a4paper,11pt,french]{scrbook} 
 


\usepackage[utf8]{inputenc} 
\usepackage{utf8math}

\usepackage{mathrsfs}
\usepackage[french]{babel}
\usepackage{mathtools} % includes amsmath
\usepackage{amssymb}
\usepackage{amsthm}
\usepackage{amscd}
\usepackage{todonotes}

\usepackage{multirow}
\usepackage{enumerate}

\usepackage{tikz}
\usepackage{framed}
\usepackage[colorlinks]{hyperref}
%\usepackage{showlabels}  


\usetikzlibrary{calc}
\usetikzlibrary{intersections}
\usetikzlibrary{arrows}
\usetikzlibrary{decorations.pathmorphing}

\newcommand{\E}{\mathbb{E}}
\newcommand{\N}{\mathbb{N}}
\newcommand{\Q}{\mathbb{Q}}
\newcommand{\R}{\mathbb{R}}
\newcommand{\Z}{\mathbb{Z}}
\newcommand{\C}{\mathbb{C}}
\newcommand{\K}{\mathbb{K}}
\newcommand{\x}{\mathbf{x}}
\newcommand{\y}{\mathbf{y}}
\newcommand{\X}{\mathscr{X}}
\newcommand{\cA}{\mathcal{A}}
\newcommand{\cD}{\mathcal{D}}
\newcommand{\cI}{\mathcal{I}}
\newcommand{\cP}{\mathcal{P}}
\newcommand{\cV}{\mathcal{V}}
\newcommand{\pscal}[1]{\langle {#1} \rangle}
\newcommand{\wt}[1]{\widetilde{#1}}
\newcommand{\car}{\mathrm{Char}}

\providecommand{\one}{\mathbf{1}}
\DeclareMathOperator{\vol}{vol}
\DeclareMathOperator{\rank}{rang}
\DeclareMathOperator{\noy}{noyau}
\DeclareMathOperator{\cone}{cone}
\DeclareMathOperator{\tcone}{tcone}
\DeclareMathOperator{\conv}{conv}
\DeclareMathOperator{\diam}{diam}
\DeclareMathOperator{\poly}{poly}
\DeclareMathOperator{\Ker}{Ker}
\DeclareMathOperator{\Var}{Var}
\DeclareMathOperator{\Id}{id}
\DeclareMathOperator{\tr}{tr}
\DeclareMathOperator{\relint}{relint}
\DeclareMathOperator{\spa}{span}
\DeclareMathOperator{\Tr}{Tr}
\DeclareMathOperator{\diag}{diag}
\DeclarePairedDelimiter\mnorm{\lvert\lvert\lvert}{\rvert\rvert\rvert} % matrix norm



\theoremstyle{plain}
\newtheorem{theorem}{Théorème}[chapter]
\newtheorem{lemma}[theorem]{Lemme}
\newtheorem{proposition}[theorem]{Proposition}
\newtheorem{claim}[theorem]{Claim}
\newtheorem{corollary}[theorem]{Corollaire}

\theoremstyle{definition}
\newtheorem{definition}{Définition}[chapter]
\newtheorem*{notation}{Notation}
\newtheorem{example}{Exemple}[chapter]
\newtheorem{remark}[theorem]{Remarque}
\newtheorem{problem}{Problème}[chapter]
\newtheorem{exercise}{Exercice}[chapter]
\newtheorem{algorithm}{Algorithme}[chapter]



%opening

\title{Algèbre linéaire avancée II}
\author{Friedrich Eisenbrand}

%\includeonly{chapter02-theoreme-spectral}

\begin{document}

\maketitle
  
\section*{Préface}
\noindent Ceci sont mes notes du cours \emph{Algèbre Linéaire Avancée II}.
La qualité de ce texte dépend fortement de la participation  des étudiants. 
Ces  sources sont gérées sur \emph{GitHub}, une plateforme importante de collaboration. Si vous trouvez des fautes, des erreurs typographique, ou même des démonstrations plus élégantes, ou des exemples qui vous aident à comprendre la matière, je vous invite à créer une \emph{branch} des fichiers en question, où dedans vous éditez le texte. Après, vous publiez (\emph{publish}) cette \emph{branch} et vos collègues peuvent discuter vos modifications. Si vous êtes satisfait avec vos modifications, vous me demandez, avec un \emph{Pull Request}, d'accepter vos modifications et finalement, le document peut être changé. Je me réjouis en avance de votre participation. 

  
\section*{Contributions}

Des corrections et modifications ont été implémentées par: 
\begin{itemize}
\item Orane Jecker 
\item Natalia Karaskova
\item Dylan Samuelian
\item Aziz Benmosbah
\item Djian Post
\item Robin Mamie
\item Alfonso Cevallos
\item Kévin Jorand
\item Charles Dufour 
\item Christoph Hunkenschröder
\item Adam Cierniak
\item Mann-Tchi Dang
\item Yasmine Bennis
\item Corentin 
\item Schmiddtt 
\item ktjkim 
\end{itemize}

\tableofcontents


\chapter{Formes bilinéaires}  
\label{cha:prod-scal-et}


\begin{definition}
  \label{def:1}
  Soit $V$
  un espace vectoriel sur un corps $K$.
  Une \emph{forme bilinéaire} sur $V$
  est une correspondance qui à tout couple $(v,w)$
  d'éléments de $V$
  associe un scalaire, noté $\pscal{v,w} ∈ K$,
   satisfaisant aux deux propriétés suivantes:
  \begin{enumerate}[BL 1]
  \item Si $u,v$ et $w$ sont des éléments de $V$, et $α ∈ K$ est un scalaire,  \label{ps:2}
    \begin{displaymath}
      \pscal{u,v+w} = \pscal{u,v}+\pscal{u,w} \quad \text{ et } \pscal{u,α ⋅ w} = α ⋅\pscal{u,w}. 
    \end{displaymath}
  \item 
    Si $u,v$ et $w$ sont des éléments de $V$, et $α ∈ K$ est un scalaire,  \label{ps:3}
    \begin{displaymath}
      \pscal{v+w,u} = \pscal{v,u}+\pscal{w,u} \quad \text{ et } \pscal{α ⋅ u, w} = α ⋅\pscal{u,w}. 
    \end{displaymath}     
  \end{enumerate}
La  forme bilinéaire  est dite \emph{symétrique} si 
pour tout $v,w \in V$ 
\begin{displaymath}
    \pscal{v,w} = \pscal{w,v}. 
\end{displaymath}
\noindent
On dit que la forme bilinéaire est \emph{non dégénérée à gauche} (respectivement \emph{à droite}) si la condition suivante est vérifiée:
\begin{quote}
  Si $v \in V$, et si $\pscal{v,w}=0$ pour tout $w \in V$, alors $v = 0$. 
\end{quote}
Si la forme bilinéaire est non dégénérée à gauche et à droite, on dit qu'elle est \emph{non dégénérée}.
\end{definition}

\begin{example}
 \label{ex:1}
  Soit $V = K^n$, l'application  
  \begin{displaymath}
    \begin{array}{rcl}
      \pscal{\,}\colon    \,  V\times V& \longrightarrow &K \\
                              (u,v)   & \longmapsto & \sum_{i=1}^n u_i v_i
    \end{array}  
  \end{displaymath}
est une forme bilinéaire. 
Vérifions (BL~\ref{ps:2}). Pour tous  $u,v,w ∈ K^n$ et $α ∈ K$:  
\begin{eqnarray*}
  \pscal{u,v+w} & = & \sum_{i=1}^n u_i \, (v_i+w_i) \\
  & = &  \sum_{i=1}^n \left( u_i v_i + u_i w_i \right) \\
  & = &  \sum_{i=1}^n  u_iv_i +\sum_{i=1}^n u_iw_i \\
  &= &\pscal{u,v} + \pscal{u,w}
\end{eqnarray*}
 et 
 \begin{displaymath}
   \pscal{u,α ⋅w} = \sum_{i=1}^n u_i α w_i = α ∑_{i=1}^n u_i w_i = α \pscal{u,w}. 
 \end{displaymath}
On appelle cette forme bilinéaire la \emph{forme bilinéaire standard} de $K^n$.   On vérifie aussi très facilement que la forme bilinéaire standard est symétrique et non dégénérée.
\end{example}




\begin{example}
  \label{ex:2}
  Soit $V$ l'espace des fonctions continues à valeurs réelles, définies sur l'intervalle $[0,2 \cdot \pi]$. Si $f,g \in V$ on pose 
  \begin{displaymath}
    \pscal{f,g} = \int_0^{2\pi} f(x)g(x) \, dx.
  \end{displaymath}
Clairement, $\pscal{,}$ est une forme bilinéaire symétrique sur $V$ non dégénérée. 
\end{example}

\begin{exercise}
  \label{ex:3}
  Montrer que les  formes bilinéaires  des exemples~\ref{ex:1} et \ref{ex:2} sont non dégénérées. 
\end{exercise}



Soit $V$ un espace vectoriel de dimension finie et $B = \{v_1,\dots,v_n\}$ une base de $V$. Pour une forme bilinéaire $f: V × V ⟶ K$ 
 et $x = \sum_i\alpha_i v_i$ et $y = \sum_j \beta_j v_j$ on a 
 \begin{eqnarray*}
   f(x,y) & = & f \left( ∑_{i=1}^n \alpha_i v_i ,   ∑_{j=1}^n \beta_j v_j \right) \\
          & = & ∑_{i=1}^n \alpha_i f \left(  v_i ,   ∑_{j=1}^n \beta_j v_j \right) \\
          & = & ∑_{i,j=1}^n \alpha_i β_j f (  v_i ,  v_j )
    \end{eqnarray*}
alors pour la matrice $A_B^f \in K^{n\times n}$, ayant comme composantes $f(v_i,v_j)$, on a 
\begin{displaymath}
  f(x,y) = [x]_B^T \,A_B^f \,  [y]_B. 
\end{displaymath}
\begin{exercise} Soit $V$ de dimension finie et $B$ une base de $V$. 
  Deux formes bilinéaires $f,g : V × V ⟶ K$
  sont différentes si et seulement si $A_B^f \neq A_B^g$. 
\end{exercise}

Pour mémoire, pour deux bases $B,B'$ et étant donné $[x]_{B'}$,  on trouve les coordonnées de $x$ dans la base $B$,  $[x]_B$,  à l'aide de la matrice de changement de base $P_{B'B}$ comme 
\begin{displaymath}
  [x]_B = P_{B'B} [x]_{B'}. 
\end{displaymath}
Cette formule nous montre que 
\begin{equation}
\label{eq:28}
 A_{B'}^f =  P_{B'B}^T \,A_B^f \,  P_{B'B}. 
\end{equation}

\begin{exercise}
  Soit $V$ un $K$-espace vectoriel de dimension finie et $B$ une base de $V$. 
  Une forme bilinéaire $f : V ×V ⟶ K$ 
  est symétrique si et seulement si $A_B^f$ est symétrique.
\end{exercise}

\begin{proposition}
  \label{prop:7}
  Soit $V$ un $K$-espace vectoriel de dimension finie, $B= \{b_1, \dots , b_n\}$ une base de $V$ et $f : V × V ⟶K$ une forme bilinéaire.   Les conditions suivantes sont équivalentes. 
  \begin{enumerate}[i)]
  \item $\rank(A_B^f) = n$ \label{item:12}
  \item $f$ est non dégénérée à gauche, i.e. si $v \in V$, et si $\pscal{v,w}=0$ pour tout $w \in V$, alors $v = 0$ \label{item:13}
  \item $f$ est non dégénérée à droite, i.e. si $v \in V$, et si $\pscal{w, v}=0$ pour tout $w \in V$, alors $v = 0$ \label{item:14}
  \end{enumerate}
\end{proposition}

\begin{proof} 

Nous montrons \ref{item:12}) et \ref{item:13}) sont équivalentes.  De la même manière, on démontre aussi que \ref{item:12}) et \ref{item:14}) sont équivalentes.

\ref{item:12}) $\Rightarrow$ \ref{item:13}) : Supposons que $\rank(A_B^f) = n$ et soit $v \in V$, $v \neq 0$. Pour $w \in V$ on a 
\begin{displaymath}
  f(v,w) = [v]_B^T A_B^f [w]_B. 
\end{displaymath}
Dès que $[v]_B \neq 0$, on a $[v]_B^T A_B^f \neq 0^T$ (car $\noy(A_B^f) = \{0\} \iff \rank(A_B^f) = n$). Supposons que la $i$-ème composante de $[v]_B^T A_B^f$ n'est pas égale a $0$. Alors $[v]_B^T A_B^f e_i \neq 0$ où toutes les composantes de $e_i$ sont $0$ sauf la $i$-ème composante, 
qui est égale a $1$. Alors $f(v,b_i) \neq 0$. Donc $f$ est non dégénérée à gauche. \newline

\ref{item:13}) $\Rightarrow$ \ref{item:12}) : Si $f$ est non dégénérée à gauche, alors $x^T A_B^f \neq 0$ pour tout $x \in K^n$ tel que $x \neq 0$ (sinon, on aurait trouvé un $x$ tel que $x^T A_B^f y = 0$ pour tout $y \in K^n$). Ceci implique que les lignes de $A_B^f$ sont linéairement indépendantes. Alors $\rank(A_B^f) = n$. 
\end{proof}






\section{Orthogonalité} 
\label{sec:orthogonalite}

\begin{framed}\noindent 
  Pour ce paragraphe~\ref{sec:orthogonalite}, s'il n'est pas spécifié autrement,  $V$
  est toujours un espace vectoriel sur $K$
  muni d'une forme bilinéaire symétrique $\pscal{,}$. 
\end{framed}



\begin{definition}
  \label{def:2}
 Deux éléments $u,v \in V$ sont \emph{orthogonaux} ou \emph{perpendiculaires} si $\pscal{u,v} = 0$, et l'on écrit $v \perp w$. 
\end{definition}

\begin{proposition}
  \label{prop:1}
  Soit $E \subseteq V$ une partie de $V$, alors $E^\perp = \{ v \in V \colon v \perp e \text{ pour tout } e \in E \}$   est un sous-espace vectoriel de $V$. 
\end{proposition}

\begin{proof}
  Pour mémoire: $\emptyset \neq W\subseteq V$ est un sous-espace si les conditions suivantes sont vérifiées. 
  \begin{enumerate}[i)]
  \item Si $u,v \in W$ on a $u+v \in W$.  
  \item Si $c \in K$ et $u \in W$ on a $c \cdot u \in W$. 
  \end{enumerate}
  
\noindent 
  Si $u,v \in E^\perp$ alors pour tout $e \in E$ 
  \begin{displaymath}
    \pscal{e,u+v} = \pscal{e,u} + \pscal{e,-v} = 0 + 0 = 0,
    \pscal{e,u+v} = \pscal{e,u} + \pscal{e,v} = 0 + 0 = 0,
  \end{displaymath}  
  et pour $c \in K$ 
  \begin{displaymath}
    \pscal{e,c\cdot v} = c \,\pscal{e,v} = c \cdot 0 = 0. 
  \end{displaymath}
\end{proof}



\begin{exercise}
  \label{exe:1}
  Soit $E \subseteq V$ et $E^*$ le sous-espace de $V$ engendré par les éléments de $E$. Montrer $E^\perp = {E^*}^\perp$. 
\end{exercise}


\begin{example}
  \label{exe:2}
  Soient $K$ un corps et $(a_{ij}) \in K^{m\times n}$ une matrice à $m$ lignes et $n$ colonnes. Le système homogène linéaire 
  \begin{equation}
    \label{eq:1}
    A\,X = 0,
  \end{equation}
  peut s'écrire sous la forme 
  \begin{displaymath}
    \pscal{A_1,X}=0, \dots ,\pscal{A_m,X}=0, 
  \end{displaymath}
  où les $A_i$  sont les vecteurs lignes de la matrice $A$ et $\pscal{,}$ dénote la forme bilinéaire standard de $K^n$. Soit $W$ le sous-espace de $K^n$ engendré par les $A_i$ et $U$ le sous-espace de $K^n$ des solutions du système~\eqref{eq:1}. Alors on a  $U = W^\perp$ et $\dim(W^\perp) = \dim(U) = n - \rank(A) = \dim(\noy(A))$. 
  
 \end{example}




\begin{definition}
  \label{def:8}
  La caractéristique d'un anneau (unitaire) $R$, $\mathrm{Char}(R)$ 
  est l'ordre de $1_R$
  comme élément du groupe abélien $(R,+)$.
  En d'autres mots, c'est le nombre 
  \begin{displaymath}
    \min_{k \in \N_+} \underbrace{1+ \cdots + 1}_{k \text{ fois }} = 0
  \end{displaymath}
  Si cet ordre est infini, la caractéristique de $R$ est $0$.
\end{definition}

\begin{notation}
  Pour $n \in \N_+$ l'anneau des classes des restes est dénoté comme
  $\Z / n \Z$ ou plus brièvement $\Z_n$ (parfois aussi noté $\Bbb F_n$). Ceci est un corps si et seulement si $n$ est un nombre premier. 
\end{notation}

\begin{example}
\label{exe:10}
   Soit $n \in \N_+$. Alors la  caractéristique de $\Z_n$ est $n$. 
   La caractéristique de $\Q,\R$ et $\C$ est zéro.    
\end{example}

\begin{lemma}
  \label{lem:1}
 Soit $\mathrm{Char}(K)\neq 2$. Si $\pscal{u,u}=0$ pour tout $u \in V$ alors 
  \begin{displaymath}
    \pscal{u,v}=0 \text{ pour tous }u,v \in V
  \end{displaymath}
On dit que la forme bilinéaire symétrique $\pscal{,}$ est  \emph{nulle}. 
\end{lemma}

\begin{proof}
  Soient $u,v \in V$. 
  On peut écrire
  \begin{displaymath}
   2 \cdot  \pscal{u,v} = \pscal{u+v,u+v} - \pscal{u,u} - \pscal{v,v} 
  \end{displaymath}

et comme $2 \neq 0$ on a $\pscal{u,v} = 0$. 
\end{proof}


\begin{definition}
  \label{def:38}
  Une base $\{v_1,\dots,v_n\}$
  de l'espace vectoriel $V$
  est une \emph{base orthogonale} si $\pscal{v_i,v_j}=0$
  pour $i\neq j$.
\end{definition} 

\begin{theorem}
  \label{thr:5}
  Soit $\mathrm{Char}(K)\neq 2$
  et supposons que $V$
  est de dimension finie. Alors $V$ possède une base orthogonale.
\end{theorem}


\begin{proof}
  On montre le théorème par induction. Si $\dim(V) = 1$ alors toute base contient seulement un élément et alors est orthogonale. 

Soit $\dim(V) >1$.  Si $\pscal{u,u} = 0$ pour tout $u$, le lemme~\ref{lem:1} implique que la forme bilinéaire symétrique est nulle et toute base de $V$  est orthogonale.  
Autrement, soit $u \in V$ tel que $\pscal{u,u} \neq 0$ et soit $V_1 = \spa\{u\}$. Pour  $x \in V$ le vecteur 
\begin{displaymath}
  x - \pscal{x,u}/\pscal{u,u} \cdot  u \in V_1^\perp
\end{displaymath}
et alors $V = V_1 + V_1^\perp$. Cette somme est directe parce que chaque élément de $V_1 \cap V_1^\perp$ s'écrit comme $\beta \cdot u$ pour $\beta \in K$. Et $\pscal{u,\beta u} = \beta \pscal{u,u} = 0$ implique $\beta = 0$. 

Alors $\dim(V_1^\perp) < \dim(V)$, et par induction, $V_1^\perp$ possède une base orthogonale $\{v_2,\dots,v_n\}$. Alors $\{v_1,\dots,v_n\}$ est une base orthogonale de $V$. 
\end{proof}  


\subsection*{Exercices} 

\begin{enumerate}
\item Soit $K$ un corps. Si la caractéristique de $K$ est différente de zéro, alors elle est un nombre premier. 
\item Soit $K$ un corps fini. Montrer que $|K| = q^\ell$ pour un nombre premier $q$ et un nombre naturel $\ell ∈ \N$. \emph{Indication: $K$ est un espace vectoriel de dimension finie sur $\Z_q$ pour un $q$ premier.}
\item On considère les vecteurs  \label{item:1}
  \begin{displaymath}
    v_1 =
    \begin{pmatrix}
      1\\1\\0\\0
    \end{pmatrix}, 
 v_2 =
    \begin{pmatrix}
      0\\1\\1\\0
    \end{pmatrix}, 
\text{ et }
 v_3 =
    \begin{pmatrix}
      0\\0\\1\\1
    \end{pmatrix} \in \Z_2^4. 
  \end{displaymath}
Est-ce que $\spa\{v_1,v_2,v_3\}$ possède une base orthogonale par rapport à la forme bilinéaire symétrique standard de l'exemple~\ref{ex:1}? 

\item En considérant le forme bilinéaire symétrique standard de l'exemple~\ref{ex:1}, trouver une base orthogonale du sous-espace de $\Z_3^4$ engendré par 
  \begin{displaymath}
    v_1 =
    \begin{pmatrix}
      1\\1\\1\\0
    \end{pmatrix}, 
 v_2 =
    \begin{pmatrix}
      0\\1\\1\\1
    \end{pmatrix}, 
\text{ et }
 v_3 =
    \begin{pmatrix}
      1\\0\\1\\1
    \end{pmatrix} \in \Z_3^4. 
  \end{displaymath}



\end{enumerate}

\section{Matrices congruentes} 
\label{sec:class-des-matr}

\begin{framed}\noindent 
  Pour ce paragraphe~\ref{sec:class-des-matr}, s'il n'est pas spécifié autrement,  $V$
  est toujours un espace vectoriel sur $K$
  muni d'une forme bilinéaire symétrique $\pscal{,}$. 
\end{framed}


\begin{definition}
  \label{def:13}
  Deux matrices $A,B \in K^{n\times n}$ sont dites \emph{congruentes} s'il existe une matrice $P \in K^{n \times n}$ inversible telle que 
\begin{displaymath}
  A = P^T B P.
\end{displaymath}
Nous écrivons $A \cong B$. 
\end{definition}


\begin{example}
  \label{exe:22}
  Si $V$ est de dimension finie et $B,B'$ sont deux bases, \eqref{eq:28} montrer que $A_B^{\pscal{,}} \cong A_{B'}^{\pscal{,}}$. 
\end{example}

\begin{lemma}
  \label{lem:6}
  La relation $\cong$ est une relation d'équivalence. 
\end{lemma}

\begin{proof}
  Voir exercice. 
\end{proof}


Le relation entre $\cong$ et le concept de l'orthogonalité   est précisée dans le lemme suivant. 
\begin{lemma}
\label{thr:13}
  Soit $B = \{v_1,\dots,v_n\}$ une base de $V$. $V$ possède une base orthogonale si et seulement s'il existe une matrice diagonale $D$ 
telle que 
 $A_B^{\pscal{.}} \cong D$. 
\end{lemma}
\begin{proof}
  Soit $B = \{v_1,\dots,v_n\}$  une base de $V$ et soit $B' = \{u_1,\dots,u_n\}$ une base orthogonale. 
Nous avons 
\begin{equation}
  \label{eq:8}
  \pscal{u_i,u_j}  =  [u_i]_B^T A_B^{\pscal{.}} {[u_j]_B}
\end{equation}  égal à zéro si $i \neq j$. Pour la matrice de passage $P = P_{B' B} \in K^{n \times n}$ dont les colonnes sont les vecteurs $[u_1]_B,\dots,[u_n]_B$, on a alors que 
\begin{displaymath}
  P^T A_B^{\pscal{.}}P
\end{displaymath}
est une matrice diagonale. Aussi $P$ est inversible, dès que $B'$ est une base de $V$. 


Soit $P \in K^{n\times n} $ inversible telle que
\begin{displaymath}
    P^T A {P} =
    \begin{pmatrix}
      c_1\\
      & \ddots \\
      && c_n
    \end{pmatrix}. 
  \end{displaymath}
  La base orthogonale est $B' = \{w_1,\dots,w_n\}$ donnée par 
  \begin{displaymath}
     [w_j]_B =
  \begin{pmatrix}
    p_{1j}\\ \vdots \\ p_{nj}
  \end{pmatrix},
  \end{displaymath}
donc $w_j = \sum_{i=1}^n p_{ij} v_i$. 
\end{proof}




 

\begin{corollary}
  \label{co:4}
  Soit $K$ un corps de caractéristique différente de $2$. Toute matrice symétrique  $A \in K^{n \times n}$ est congruente à une matrice diagonale. 
\end{corollary}
\begin{proof}
Ceci est un corollaire du  lemme~\ref{thr:5} et du théorème~\ref{thr:13} 
parce que  $K^n$ muni de la forme bilinéaire symétrique $\pscal{u,v} = u^TAv$ possède une base orthogonale. 
\end{proof}

%The example below is uncomplete and thus makes no sense. Since I didn't know what was its purpose, I couldn't complete it.
%\begin{example}
%  \label{exe:14}
%  Soient $V$ un espace vectoriel sur $\R$ et $\{v_1,v_2,v_3\}$ une base de V. 
%\\ \todo[inline]{Finish or delete the preceding example}
%\end{example}




Maintenant soit $K = \R$  et $A \in \R^{n \times n}$ symétrique. Le Corollaire~\ref{co:4} implique qu'il existe une matrice inversible $P ∈ ℝ^{n \times n}$ tel que $P^T A P = D$ où $D$ est une matrice diagonale. Si on échange deux colonnes de $P$ et note $P'$ la nouvelle matrice obtenue, alors $P'^T A P' = D'$, où $D'$ est obtenue de $D$ en échangeant les éléments diagonaux correspondants. Alors on peut trouver une matrice inversible $P ∈ ℝ^{n ×n}$ telle que 
\begin{equation}
  \label{eq:29}
  P^T A {P} =
    \begin{pmatrix}
      c_1\\
      & \ddots \\
      && c_n
    \end{pmatrix}. 
\end{equation}
où les  $c_i$ sont ordonnés de sorte que  $c_1,\dots, c_r >0$, $c_{r+1},\dots ,c_s<0$ et $c_{s+1},\dots , c_n = 0$. En multipliant les premières $s+r$ colonnes  et lignes de $P$ par $1/ \sqrt{|c_i|}$ on obtient en fait une factorisation~\eqref{eq:29} telle que 
 $c_1,\dots, c_r =1$, $c_{r+1},\dots ,c_s=-1$ et $c_{s+1},\dots , c_n = 0$. 

Alors on trouve  $P ∈ ℝ^{n ×n}$ inversible telle que 
\begin{equation}
  \label{eq:5}
  P^T A P =
  \begin{pmatrix}
    1 &   \\
      & \ddots &  \\
      &        & 1 \\
      &        &  & -1 \\
      &        &  &    & \ddots \\
      &        &  &    &        & -1 \\
      &        &  &    &        &    & 0 \\
      &        &  &    &        &    &   & \ddots  \\
      &        &  &    &        &    &   &        & 0  \\

  \end{pmatrix}.
\end{equation}


\begin{definition}
  \label{def:37}
  Pour un espace vectoriel sur $ℝ$ de dimension finie, on appelle une base $B$ de $V$ telle que $A_B^{\pscal{,}}$ a la forme décrite en~\eqref{eq:5} une \emph{base de Sylvester}. 
\end{definition}



\begin{algorithm}
  \label{alg:1}
  Cet algorithme trouve une matrice diagonale congruente à la matrice symétrique $A \in K^{n \times n}$ où $K$ est un corps tel que $\car(K) \neq 2$.  L'algorithme procède en $n$ itérations. Après la $(i-1)$-ème itération, $i \geq 1$, (aussi après  la $0$-ème itération) l'algorithme a transformé $A$ en une matrice congruente 
  \begin{equation}
    \label{eq:7}
    \begin{pmatrix}
      c_1 \\
      & c_2 \\
      & & \ddots & &&\\
      & & & c_{i-1} \\
      & & & &  b_{i,i} & \dots & b_{i,n} \\
%      & & & &  b_{i+1,i} & \dots & b_{i+1,n} \\
      & & & &     \vdots       &  & \vdots \\
      & & & &  b_{n,i} & \dots & b_{n,n} \\      
    \end{pmatrix}
  \end{equation}
où les composantes des premières $(i-1)$ lignes et colonnes sont zéro sauf pour la composante sur la diagonale de la matrice. 

\medskip 
\noindent 
Pour $1 \leq i \leq n$, la \emph{$i$-ème itération}  est comme suit. 
\begin{itemize}
\item Si la $i$-ème ligne est zéro, l'algorithme procède avec la $(i+1)$-ème itération. 
\item Autrement, si $b_{ii}=0$ : Soit $j \in \{i+1,\dots,n\}$ tel que $b_{ij} \neq 0$.  Si $b_{jj} \neq 0$ on échange la $i$-ème ligne et la $j$-ème ligne et après    la $i$-ème colonne et la $j$-ème colonne. 
  Autrement on additionne la ligne $j$ sur la ligne $i$ et on additionne la colonne $j$ sur la colonne $i$. 
\item Étant donné que $\car(K) \neq 2$, on a alors maintenant $b_{ii} \neq 0$. 
\item Pour chaque $j \in \{i+1,\dots,n\}$:  on additionne $-b_{ij}/b_{ii}$ fois la $i$-ème ligne sur la $j$-ème ligne et on additionne $-b_{ij}/b_{ii}$ fois la $i$-ème colonne sur la $j$-ème colonne. 
\end{itemize}


\end{algorithm}



\begin{example}
  \label{exe:15}
  Soit $V$ une espace vectoriel sur $\Q$ de dimension $3$ muni d'une forme bilinéaire symétrique. Soit $B = \{v_1,v_2,v_3\}$ une base de $V$ et 
  \begin{displaymath}
    A^{\pscal{.}}_B =
    \begin{pmatrix}
      1 & 0 & 2\\
      0 & 3 & 4\\
      2 & 4 & 0 
    \end{pmatrix}
  \end{displaymath}
Le but est de trouver une  base orthogonale de $V$. 

En utilisant notre algorithme on trouve 
\begin{displaymath}
  P = 
  \begin{pmatrix}
    
1 & 0 & -2 \\
0 & 1 & -\frac{4}{3} \\
0 & 0 & 1
  \end{pmatrix}
\end{displaymath}
tel que 
\begin{displaymath}
  P^T \cdot A^{\pscal{.}}_B \cdot P =
  \begin{pmatrix}
    1 & 0 & 0 \\
0 & 3 & 0 \\
0 & 0 & -\frac{28}{3}
  \end{pmatrix}. 
\end{displaymath}
Alors $B' = \{v_1,v_2,-2v_1 -(4/3) v_2 + v_3\}$ est une base orthogonale de $V$. 
\end{example}


\subsection*{Exercices}

\begin{enumerate}
\item Montrer que $\cong$ est une relation d'équivalence sur l'ensemble des matrices $K^{n\times n}$. 
\item Est-ce que la matrice 
  \begin{displaymath}
    \begin{pmatrix}
      0 & 1 & 0 \\
      1 & 0 & 1\\
      0 & 1 & 0 
    \end{pmatrix} \in \Z_2^{3\times 3} 
  \end{displaymath}
  est congruente à une matrice diagonale? \emph{Renseignement: voir l'exercice~\ref{item:1}. de la section~\ref{sec:orthogonalite}.}  
\item Soit $V$ un espace vectoriel  sur un corps $K$ de dimension finie muni d'une forme bilinéaire symétrique $\pscal{.}$. Soit $B = \{v_1,\dots,v_n\}$ une base de $V$. Montrer que $A_B^{\pscal{.}} \in K^{n \times n}$ est congruente à une matrice diagonale si  et seulement si $V$ possède une base orthogonale. 

\item Soit $K$ un corps de caractéristique $2$ et soit $V$ un espace vectoriel sur $K$ de dimension finie muni d'une forme bilinéaire symétrique. Soit 
  \begin{displaymath}
    C =
    \begin{pmatrix}
      0 & 1 \\
      1 & 0 
    \end{pmatrix}.
  \end{displaymath}
  \begin{enumerate}[a)]
  \item Soit $\dim(V) = 2$. Montrer que $V$ ne possède pas de base orthogonale si et seulement s'il existe une base $B$ de $V$ telle que $A_B^{\pscal{.}} = C$. 
  \item Soit $\dim(V) = n$. Montrer que $V$ ne possède pas de base orthogonale si et seulement s'il existe une base $B$ de $V$ telle que 
    \begin{displaymath} A_B^{\pscal{.}} = 
      \begin{pmatrix}
        d_1 \\
        & d_2 \\
        & & \ddots \\
        & &   & d_k \\
        & &   &  & C \\
        & &   &  & & \ddots \\
        & &   &  & & & C \\
      \end{pmatrix}
    \end{displaymath}
    et $d_1,\dots,d_k = 0$, et le nombre de $C$ n'est pas égal à zéro. 
  \end{enumerate}
\item Modifier l'algorithme~\ref{alg:1} tel qu'il soit aussi correct pour des corps de caractéristique $2$.  Soit l'algorithme découvre que la matrice symétrique $A \in K^{n \times n}$ n'est pas congruente à une matrice diagonale, soit l'algorithme calcule une matrice diagonale congruente à $A$. 
\item Comment peut-on déterminer si un espace vectoriel de dimension finie muni d'une forme bilinéaire symétrique  possède une base orthogonale? Décrire très brièvement une méthode. 
\item Déterminer l'indice de nullité et l'indice de positivité des 
formes bilinéaire symétriques 
 définies par les matrices suivantes
  \begin{displaymath}
    \begin{pmatrix}
      1 & 2 \\
      2 & -1
    \end{pmatrix}
    , \,
    \begin{pmatrix}
      1 & 1 \\
      1 & 1
    \end{pmatrix}, \, 
    \begin{pmatrix}
      2 & 4 & 2 \\
      4 & 3 &  1 \\ 
      2 & 1 &1
    \end{pmatrix}. 
  \end{displaymath}
 
\item Soit $V$ un espace euclidien de dimension $n$. Montrer que $V$ possède une base $B$ telle que pour tout $x,y \in V$
  \begin{displaymath}
    \pscal{x,y} = [x]_B \cdot [y]_B, 
  \end{displaymath}
où $ [x]_B \cdot [y]_B$  dénote la forme bilinéaire standard de $\R^n$ entre $[x]_B$ et $ [y]_B$ .  

\end{enumerate}




\section{Le théorème de Sylvester}
\label{sec:le-theoreme-de}


Soit $V$ un espace vectoriel de dimension finie sur un corps $K$ muni d'une forme bilinéaire symétrique. Nous avons vu (théorème~\ref{thr:5}) que $V$ possède une base orthogonale. Supposons que cette base est $B = \{v_1,\dots,v_n\}$ et considérons $x = \sum_i \alpha_i v_i \in V$ et $y \in \sum_i \beta_i v_i \in V$. La forme bilinéaire  s'écrit 
\begin{eqnarray*}
  \pscal{x,y} & =  & \sum_{i, j} \alpha_i \beta_j \pscal{v_i,v_j} \\
              & = & \sum_i \alpha_i \beta_i \pscal{v_i,v_i} \\
               & = & [x]_B^T 
                    \begin{pmatrix}
                      c_1 & & \\
                                     & \ddots & \\
                                     & & c_n
                    \end{pmatrix} [y]_B
\end{eqnarray*}
où $c_i = \pscal{v_i,v_i}$ pour tout $i$. 
Si $K = \R$ on peut ordonner la base afin d'avoir $c_1,\dots,c_r>0$, $c_{r+1},\dots,c_s < 0$ et $c_{s+1},\dots,c_n = 0$. Nous allons maintenant démontrer, que les nombres $r$ et $s$ sont invariants par rapport à la base $B$ de $V$. 


\begin{definition}
  \label{def:11}
  Le sous espace $V_0 = \{v \in V \colon \pscal{v,x} = 0 \text{ pour tout } x \in V\}$ est appelé l'\emph{espace de nullité} de la forme bilinéaire symétrique $\pscal{.}$. 
\end{definition}

\begin{theorem}
  \label{thr:9}
  Soit $V$
  un espace vectoriel de dimension finie sur un corps $K$
  de caractéristique $\neq 2$
  et soit $V$
  muni d'une forme bilinéaire symétrique. Soit $B = \{v_1,\dots,v_n\}$
  une base orthogonale de $V$.
  La dimension $\dim(V_0)$
  est égale au nombre d'index $i$ tel que $\pscal{v_i,v_i}=0$.
\end{theorem}

\begin{proof}
  Nous utilisons la notation d'en-dessus et écrivons 
  \begin{displaymath}
    \pscal{v,x} =   [v]_B^T
                    \begin{pmatrix}
                      c_1 & & \\
                                     & \ddots & \\
                                     & & c_n
                    \end{pmatrix} [x]_B. 
  \end{displaymath}
Cette expression est égale à zéro pour tout $x ∈V$  si et seulement si $\left([v]_B\right)_i = 0$ pour tout $i$ tel que $c_i \neq 0$. Ceci démontre que $\{ v_i \colon \pscal{v_i,v_i}=0\}$ est une base de l'espace de nullité. 
\end{proof}



\begin{theorem}[Théorème de Sylvester] 
\label{thr:10}
Soit $V$
un espace vectoriel de dimension finie sur $\R$
muni d'une forme bilinéaire symétrique.
Il existe un nombre entier $r ≥0$ tel que, pour chaque base orthogonale 
  $B = \{v_1,\dots,v_n\}$ de $V$, 
 exactement $r$ des indices $i$ satisfont $\pscal{v_i,v_i}>0$.
\end{theorem}


\begin{proof}
  Soient $\{v_1,\dots,v_n\}$
  et $\{w_1,\dots,w_n\}$ des
  bases orthogonales de $V$ ordonnées 
  telles que $\pscal{v_i,v_i} >0 $
  si $1 \leq i \leq r$,
  $\pscal{v_i,v_i} <0 $
  si $r+1 \leq i \leq s$
  et $\pscal{v_i,v_i} =0 $
  si $s+1 \leq i \leq n$.
  De même $\pscal{w_i,w_i} >0 $
  si $1 \leq i \leq r'$,
  $\pscal{w_i,w_i} <0 $
  si $r'+1 \leq i \leq s'$
  et $\pscal{w_i,w_i} =0 $ si $s'+1 \leq i \leq n$.



  On démontre que $v_1,\dots,v_r,w_{r'+1},\dots w_n$
  est linéairement indépendant. Ça implique que $r + n-r' \leq n$
  et alors $r \leq r'$.
  Parce que l'argument est symétrique on peut conclure que $r = r'$.

  Si $v_1,\dots,v_r,w_{r'+1},\dots w_n$ est linéairement dépendant, il existe des scalaires $x_1,\dots x_r$ et $y_{r'+1},\dots y_n$ respectivement  non tous égaux à zéro tels que 
  \begin{displaymath}
    x_1 v_1 + \cdots + x_r v_r = y_{r'+1} w_{r'+1} + \cdots y_n w_n 
  \end{displaymath}
et ça implique, dès que les $v_i$ et respectivement les $w_i$  sont orthogonaux, 
\begin{displaymath}
   x_1^2\pscal{v_1,v_1} + \cdots + x_r^2 \pscal{v_r,v_r} = y_{r'+1}^2 \pscal{w_{r'+1},w_{r'+1}} + \cdots y_n^2 \pscal{w_n,w_n} 
\end{displaymath}
Les $\pscal{v_i,v_i}$ à gauche sont strictement positifs. Les $\pscal{w_i,w_i} $ à droite sont non positifs. Alors $x_1=0,\dots,x_r = 0$ et dès que les $w_i$ sont linéairement indépendants, on a $y_{r'+1}=0,\dots,y_n=0$. 
\end{proof}


\begin{definition}
  \label{def:12}
  L'entier $r$ du théorème de Sylvester est appelé l'\emph{indice de positivité} de la forme bilinéaire symétrique. 
\end{definition}



\begin{example}
  \label{exe:11}
  Trouver une base de Sylvester et les indices de nullité et de positivité de la matrice 
  \begin{displaymath}
A = 
    \begin{pmatrix}
      2 & 4 & 6 \\
      4 & 4 & 3 \\
      6 & 3 & 1 
    \end{pmatrix}
  \end{displaymath}
On utilise des transformations élémentaires sur les colonnes et les mêmes sur les lignes tour à tour en alternant. 

\noindent Les transformations élémentaires sur les colonnes sont représentées par 
\begin{displaymath}
P_1 = 
  \left[\begin{matrix}1 & -2 & -3\\0 & 1 & 0\\0 & 0 & 1\end{matrix}\right]
\end{displaymath}
et transforment la matrice $A$ en 
\begin{displaymath}
   \begin{pmatrix}
      2 & 4 & 6 \\
      4 & 4 & 3 \\
      6 & 3 & 1 
    \end{pmatrix} \cdot  \left[\begin{matrix}1 & -2 & -3\\0 & 1 & 0\\0 & 0 & 1\end{matrix}\right] = \left[\begin{matrix}2 & 0 & 0\\4 & -4 & -9\\6 & -9 & -17\end{matrix}\right]. 
\end{displaymath}
Alors 
\begin{displaymath}
  P_1^T\cdot A \cdot P = \left[\begin{matrix}2 & 0 & 0\\0 & -4 & -9\\0 & -9 & -17\end{matrix}\right]
\end{displaymath}
Avec 
\begin{displaymath}
  P_2 = \left[\begin{matrix}1 & 0 & 0\\0 & 1 & -\frac{9}{4}\\0 & 0 &1\end{matrix}\right]
\end{displaymath}
on obtient 
\begin{displaymath}
  P_2^TP_1^T A P_1 P_2 = \left[\begin{matrix}2 & 0 & 0\\0 & -4 & 0\\0 & 0 & \frac{13}{4} \end{matrix}\right]
\end{displaymath}
L'indice de nullité est zéro et l'indice de positivité est $2$. Le produit $P_1\cdot P_2$ est égal à 
\begin{displaymath}
  P_1 \cdot P_2 = \left[\begin{matrix}1 & -2 & \frac{3}{2}\\0 & 1 & -\frac{9}{4}\\0 & 0 & 1\end{matrix}\right]
\end{displaymath}
Les colonnes sont une base de Sylvester. En divisant chaque colonne et chaque ligne par $\sqrt{2},\sqrt{4}$ et $\sqrt{13/4}$ respectivement, et en échangeant les deux dernières colonnes et deux dernières lignes, on obtient une transformation $P$ telle que $P^TAP =
\begin{pmatrix}
  1& \\
  & 1 & \\
  & & -1
\end{pmatrix}$. 
\end{example}



\subsection*{Exercices} 

\begin{enumerate}
\item Soit $V$ un espace vectoriel de dimension finie sur $\R$ et soit $\pscal{.}$ une forme bilinéaire  sur $V$.  Montrer que $V$ admet une décomposition en somme directe 
  \begin{displaymath}
    V_0 \oplus V^+ \oplus V^-
  \end{displaymath}
où $V_0$ est l'espace de nullité et $V^+$ et $V^-$ sont des sous-espaces tels que 
\begin{displaymath}
  \pscal{v,v} >0 \text{ pour tout } v \in V^+ \setminus\{0\}
\end{displaymath}
et  
\begin{displaymath}
  \pscal{v,v} <0 \text{ pour tout } v \in V^- \setminus\{0\}. 
\end{displaymath}
\end{enumerate}







\section{Le cas réel, défini positif}
\label{sec:le-case-reel}



\begin{definition}
  \label{def:4}
  Soit $V$ un espace vectoriel sur $\R$ muni 
  d'une forme bilinéaire symétrique. 
  La forme bilinéaire symétrique  est définie positive si $\pscal{v,v} \geq 0$ pour tout $v \in V$, et si $\pscal{v,v}>0$ lorsque $v \neq 0$. Une forme bilinéaire symétrique définie positive est un \emph{produit scalaire}. 
\end{definition}

\begin{example}
  \label{exe:3}
  Soit $V = \R^n$. La forme bilinéaire symétrique
  \begin{displaymath}
    \pscal{u,v} = \sum_{i=1}^n u_iv_i 
  \end{displaymath}
  est un produit scalaire, appelé le \emph{produit scalaire ordinaire}. 
  Aussi, la forme bilinéaire  de l'exemple~\ref{ex:2} est un produit scalaire. 
\end{example}

\begin{definition}
  \label{def:5}
  Soit $\pscal{,}$ un produit scalaire. La \emph{longueur} ou la \emph{norme} d'un élément $v \in V$ est le nombre 
  \begin{displaymath}
    \| v \| = \sqrt{\pscal{v,v}}.
  \end{displaymath}
  Un élément $v \in V$ est un \emph{vecteur unitaire} si $\|v\| = 1$. 
\end{definition}


\begin{framed}\noindent 
  Pour le reste de ce paragraphe~\ref{sec:le-case-reel}, s'il n'est pas spécifié autrement,  $V$
  est toujours un espace vectoriel sur $\R$
  muni d'un produit scalaire. 
\end{framed}



\begin{proposition}
  \label{prop:2}
  Pour $v \in V$ et $\alpha \in \R$ on a
  \begin{displaymath}
    \| \alpha \,v \| = |\alpha| \, \|v\|. 
  \end{displaymath}
\end{proposition}



\begin{proof}
  \begin{eqnarray*}    
   \| \alpha \,v \| & = &  \sqrt{\pscal{\alpha \, v, \alpha \, v} } \\
                    & = & \sqrt{\alpha^2 \pscal{v,v}} \\
                    & = & |\alpha | \, \|v\|. 
  \end{eqnarray*}
\end{proof}

\begin{proposition}[Théorème de Pythagore]
\label{prop:4}
Si $v$ et $w$ sont perpendiculaires 
\begin{displaymath}
  \|v+w\|^2 = \|v\|^2 + \|w\|^2. 
\end{displaymath}  
\end{proposition}


\begin{proof}
  \begin{eqnarray*}
     \|v+w\|^2 &= & \pscal{v+w,v+w} \\
               & = & \pscal{v,v+w} + \pscal{w,v+w} \\
               & = & \pscal{v,v} + \pscal{v,w} + \pscal{w,v} + \pscal{w,w} \\
               & = & \|v\|^2 + \|w\|^2
  \end{eqnarray*}
\end{proof}

\begin{proposition}[Règle du parallélogramme] Pour tous $v$ et $w$, on a 
  \begin{displaymath}
    \|v+w\|^2 + \|v-w\|^2 = 2 \|v\|^2 + 2 \|w\|^2.  
  \end{displaymath}  
\end{proposition}



Soit $V$ un espace vectoriel sur un corps $K$ muni d'une forme bilinéaire symétrique  $\pscal{,}$. 
Si $w$ est un élément de $V$ tel que $\pscal{w,w} \neq 0$, pour  tout $v \in V$, il existe un élément unique $\alpha \in K$ tel que $\pscal{w,v -\alpha\, w} = 0$. 

En fait,
\begin{displaymath}
  \pscal{w,v -\alpha\, w} = \pscal{w,v} - \alpha \pscal{w,w}. 
\end{displaymath}
Alors $\pscal{w,v -\alpha\, w} = 0$ si et seulement si $\alpha = \pscal{v,w} / \pscal{w,w}$. 


\begin{definition}
  \label{def:6}
  Soit $V$ un espace vectoriel sur un corps $K$, muni d'un produit scalaire. 
  Soit $w \in V \setminus \{0\}$ tel que  $\pscal{w,w} > 0$. Pour $v \in V$, soit $\alpha = \pscal{v,w} / \pscal{w,w}$.  Le nombre $\alpha$ est la \emph{composante} de $v$ sur $w$, ou \emph{le coefficient de Fourier de $v$ relativement à $w$}. Le vecteur $\alpha \, w$ s'appelle la \emph{projection} de $v$ sur $w$. 
\end{definition}


\begin{example}
  \label{exe:5}
  Soit $V$ l'espace vectoriel de l'exemple~\ref{ex:2} et $f(x) = \sin kx$, où $k \in \N_{>0}$. Alors
  \begin{displaymath}
   \|f\| = \sqrt{\pscal{f,f}} =  \sqrt{\int_0^{2\pi} \sin^2 kx \, dx} = \sqrt{\pi}
  \end{displaymath}
Si $g$ est une fonction quelconque, continue sur $[0,2\,\pi]$, le coefficient de Fourier de $g$ relativement à $f$ est 
\begin{displaymath}
  \pscal{f,g} /   \pscal{f,f}  = \frac{1}{\pi} \int_0^{2\, \pi} g(x) \sin kx \,dx. 
\end{displaymath}
\end{example}



\begin{theorem}[Inégalité de Cauchy-Schwarz]
  Pour tous $v,w \in V$, on a  
  \begin{displaymath}
    |\pscal{v,w}| \leq \|v\| \, \|w\|.
  \end{displaymath}
\end{theorem}


\begin{proof}
  Si $w=0$, les deux termes de cette inégalité sont nuls et elle devient évidente. Supposons maintenant que $w$ est un vecteur unitaire. Si $\alpha = \pscal{v,w}$ est la composante de $v$ sur $w$, $v - \alpha w$ est perpendiculaire à $w$, donc aussi à $\alpha\,w$. D'après le théorème de Pythagore, on trouve 
  \begin{eqnarray*}
    \|v\|^2 & = & \|v - \alpha \,w\|^2 + \|\alpha \, w \|^2 \\
           & = &  \|v - \alpha \,w\|^2 + \alpha^2,
  \end{eqnarray*}
par conséquent $\alpha^2 \leq \|v\|^2$, si bien que $|\alpha| \leq \|v\|$. 

Enfin, si $w \neq 0$, alors $w / \|w\|$ est unitaire. Donc par ce que nous venons de voir,
\begin{displaymath}
  \left| \pscal{v, w / \|w\|} \right| \leq \|v\|.
\end{displaymath}
Cela implique 
\begin{displaymath}
   \left| \pscal{v, w } \right| \leq \|v\| \|w\|.
\end{displaymath}
\end{proof}

On appelle un  espace vectoriel sur $\R$ muni d'un produit scalaire  un \emph{espace euclidien}.  


\begin{theorem}[Inégalité triangulaire]
  \label{thr:1}
  Si $v,w \in V$. 
  \begin{displaymath}
    \|v+w\| \leq \|v\| + \|w\|. 
  \end{displaymath}
\end{theorem}


\begin{proof}
  \begin{eqnarray*}
    \|v+w\|^2 & =     & \|v\|^2 + 2 \pscal{v,w} + \|w\|^2 \\
              & \leq & \|v\|^2 + 2 \|v\|\,\|w\| + \|w\|^2 \\
              & = & (\|v\| + \|w\|)^2,
  \end{eqnarray*}
en recourant à l'inégalité de Cauchy-Schwarz. 
\end{proof}


\begin{lemma}
  \label{lem:2}
  Soit $V$ un espace vectoriel sur $K$ muni d'un produit scalaire et 
  soient $v_1,\dots,v_n$ des éléments de V, deux à deux orthogonaux, tels que $\pscal{v_i,v_i}\neq 0$ pour tout $i$. Soit $\alpha_i = \pscal{v,v_i}/\pscal{v_i,v_i}$ la composante de $v$ sur $v_i$, alors le vecteur 
\begin{displaymath}
  v - a_1v_1- \cdots - a_n v_n\, \text{ où  }\, a_i \in K 
\end{displaymath}
est perpendiculaire à tous $v_1,\dots,v_n$ si et seulement si $a_i$ est la composante de $v$ sur $v_i$, i.e. $a_i=  \pscal{v,v_i}/ \pscal{v_i,v_i}$ pour tout $i$. 
\end{lemma}

\begin{proof}
  Pour le vérifier, il suffit d'en faire le produit scalaire avec $v_j$ pour tout $j$. Tous les termes $\pscal{v_i,v_j}$ donnent zéro si $i\neq j$. Le reste
  \begin{displaymath}
    \pscal{v,v_j} - a_j\pscal{v_jv_j}
  \end{displaymath}
s'annule si et seulement si $a_j = \pscal{v,v_j}/\pscal{v_jv_j}$. 
\end{proof}


 % \begin{definition}
 %   \label{def:3}
 %   Soit $V$ un espace vectoriel muni d'un produit scalaire. Soit $\{v_1,\dots,v_n\}$ une base de $V$. On dira que cette base est \emph{orthogonale} si $\pscal{v_i,v_j}=0$ pour tout $i \neq j$. 
 % \end{definition}

\begin{notation}
  Soient $V$ un espace vectoriel et $v_1,\dots,v_n \in V$. Le sous-espace engendré par $v_1,\dots,v_n$ est dénoté par  $\spa\{v_1,\dots,v_n\}$. 
\end{notation}



\begin{theorem}[Le procédé d'orthogonalisation de Gram-Schmidt]
  \label{thr:2}
  Soient $V$ un espace euclidien et  $\{v_1,\dots,v_n\} \subseteq V$
  un ensemble libre.  
  Il existe un ensemble libre orthogonal $\{u_1,\dots,u_n\}$
  de $V$
  tel que pour tout $i$, 
  $\{v_1,\dots,v_i\}$
  et $\{u_1,\dots,u_i\}$ engendrent le même sous-espace de $V$.
\end{theorem}


\begin{proof}
  On montre le théorème par induction.  On met $u_1 = v_1$
  et on suppose qu'on a construit $\{u_1,\dots,u_{i-1}\}$
  pour $i \geq 2$.
  L'ensemble $\{u_1,\dots,u_{i-1},v_i\}$
  est libre et une base du sous-espace engendré
  par $\{v_1,\dots,v_i\}$.
  On met  
  \begin{displaymath}
    u_i = v_i - \alpha_{1,i}  u_1 - \cdots - \alpha_{i-1,i} u_{i-1}
  \end{displaymath}
  où les $\alpha_{j,i}$
  sont les composantes de $v_i$
  sur $u_j$.
  Comme ça 
  \begin{eqnarray*}
    \spa \{u_1,\dots,u_i\} & = & \spa\{u_1,\dots,u_{i-1},v_i\} \\
                           &=  &\spa\{v_1,\dots,v_i\}.
  \end{eqnarray*}
  Surtout $\{u_1,\dots,u_i\}$ est un ensemble orthogonal.  
\end{proof}



\begin{exercise}
  \label{exe:6}
  Est-ce qu'il faut vraiment supposer que le produit scalaire
  $\pscal{.}$
  soit réel et défini positif et sur $\R$ pour ce procédé? Peux-tu imaginer une condition plus
  faible et satisfaite par le produit scalaire qui permet
  le procédé de Gram-Schmidt? 
\end{exercise}



\begin{definition}
  \label{def:7}
  Une base $\{u_1,\dots,u_n\}$
  d'un espace euclidien est \emph{orthonormale} si elle est
  orthogonale et se compose de vecteurs tous unitaires.
\end{definition}




\begin{corollary}
  \label{co:1}
  Soit $V$
  un espace euclidien de dimension finie. Supposons $V \neq
  \{0\}$. $V$ possède alors une base orthonormale.
\end{corollary}

\begin{proof}
  Soient $\{v_1,\dots,v_n\}$ une base de $V$ et $\{u_1,\dots,u_n\}$ le résultat du procédé Gram-Schmidt appliqué à $\{v_1,\dots,v_n\}$. Alors $\{u_1/ \|u_1\|,\dots,u_n/\|u_n\|\}$ est une base orthonormale de $V$. 
\end{proof}


\begin{example}
\label{exe:7}
Trouver une base orthonormale de l'espace vectoriel engendré par 
\begin{displaymath}
  \begin{pmatrix}
    1\\1\\0\\1
  \end{pmatrix}, 
  \begin{pmatrix}
    1\\-2\\0\\0
  \end{pmatrix}
  \text{ et } 
  \begin{pmatrix}
    1\\0\\-1\\2
  \end{pmatrix}
\end{displaymath}

Notons $A,B$ et $C$ les vecteurs. Soit $A'=A$ et 
\begin{displaymath}
  B' = B - \frac{A'\cdot B}{A'\cdot A'} \cdot A'
\end{displaymath}
On trouve 
\begin{displaymath}
  B' = \frac{1}{3}
  \begin{pmatrix}
    4\\-5\\0\\1
  \end{pmatrix}
\end{displaymath}
On calcule 
\begin{displaymath}
  C' = C - \frac{A'\cdot C}{A'\cdot A'} \cdot A' - \frac{B'\cdot C}{B'\cdot B'} \cdot B'
\end{displaymath}
et on trouve 
\begin{displaymath}
  C' = \frac{1}{7}
  \begin{pmatrix}
    -4\\-2\\-1\\6
  \end{pmatrix}  
\end{displaymath}
La base orthonormale est
\begin{displaymath}
  A' / \|A'\| = \frac{1}{\sqrt{3}}
  \begin{pmatrix}
    1\\1\\0\\1
  \end{pmatrix},
    B'/ \|B'\| = \frac{1}{\sqrt{42}}
  \begin{pmatrix}
    4\\-5\\0\\1
  \end{pmatrix}
\text{ et }
C'/ \|C'\| = \frac{1}{\sqrt{57}}
  \begin{pmatrix}
    -4\\-2\\-1\\6
  \end{pmatrix}  
\end{displaymath}
\end{example}

\begin{corollary}
  \label{co:2}
  Soit $A \in \R^{m\times n}$ une matrice de rang (colonne) plein. On peut factoriser $A$ comme 
  \begin{displaymath}
    A = A' \cdot R
  \end{displaymath}
où les colonnes de  $A' \in \R^{m\times n}$ sont deux à deux orthonormales et $R \in \R^{n \times n}$ est une matrice triangulaire supérieure dont les valeurs diagonales sont positives. 
\end{corollary}



\begin{example}
  \label{exe:8}
  Trouver une  factorisation $Q,R$ du Corollaire~\ref{co:2} de la matrice 
  \begin{displaymath}
    \left(\begin{matrix}1 & 1 & 1\\1 & -2 & 0\\0 & 0 & -1\\1 & 0 & 2\end{matrix}\right)
  \end{displaymath}

On trouve 
\begin{displaymath}
  \left[\begin{matrix}1 & 1 & 1\\1 & -2 & 0\\0 & 0 & -1\\1 & 0 & 2\end{matrix}\right]
= \left[\begin{matrix}1 & \frac{4}{3} & - \frac{4}{7}\\1 & - \frac{5}{3} & - \frac{2}{7}\\0 & 0 & -1\\1 & \frac{1}{3} & \frac{6}{7}\end{matrix}\right] 
\left[\begin{matrix}1 & - \frac{1}{3} & 1\\0 & 1 & \frac{3}{7}\\0 & 0 & 1\end{matrix}\right]
\end{displaymath}
et alors
\begin{displaymath}
   \left[\begin{matrix}1 & 1 & 1\\1 & -2 & 0\\0 & 0 & -1\\1 & 0 & 2\end{matrix}\right] = 
\left[\begin{matrix}\frac{1}{\sqrt{3}} & \frac{2 \sqrt{42}}{21} & - \frac{4 }{\sqrt{105}}\\\frac{1}{\sqrt{3}} & - \frac{5}{\sqrt{42}} & - \frac{2 }{\sqrt{105}}\\0 & 0 & -\frac{\sqrt{105}}{15}\\\frac{1}{\sqrt{3}} & \frac{1}{\sqrt{42}} & \frac{2 \sqrt{105}}{35}\end{matrix}\right] 
\left[\begin{matrix}\sqrt{3} & - \frac{\sqrt{3}}{3} & \sqrt{3}\\0 &\frac{\sqrt{42}}{3} & \frac{\sqrt{42}}{7}\\0 & 0 & \frac{\sqrt{105}}{7}\end{matrix}\right]
\end{displaymath}
\end{example} 




\begin{theorem}[Inégalité de Bessel]
  \label{thr:4}
  Si $v_1,\dots,v_n$ sont des vecteurs unitaires deux à deux orthogonaux et si $\alpha_i = \pscal{v,v_i} / \pscal{v_i,v_i}$ sont les coefficients de Fourier de $v$ relativement à $v_i$ alors 
  \begin{displaymath}
    \sum_{i=1}^n \alpha_i^2 \leq \|v\|^2. 
  \end{displaymath}
\end{theorem}

\begin{proof}
  \begin{eqnarray*}
    0 & \leq & \pscal{v - \sum_{i=1}^n \alpha_i v_i , v - \sum_{i=1}^n \alpha_i v_i} \\
     & = & \pscal{v,v} - 2 \cdot \sum \alpha_i \pscal{v,v_i} + \sum \alpha_i^2 \\
    & = & \pscal{v,v}  - \sum \alpha_i^2 \\
  \end{eqnarray*}
\end{proof}



\subsection*{Exercices}

\begin{enumerate}

\item Soit $V$ un espace vectoriel muni d'une forme bilinéaire $\pscal{.}$. Soit $U \subseteq V$ un sous-espace. Montrer que $\pscal{.}$ restreint à $U$ est un produit scalaire du sous-espace $U$. 
\item Soient $V$ un espace vectoriel muni d'une forme bilinéaire symétrique $\pscal{.}$ et $\{v_1,\dots,v_n\} \subseteq V$ un ensemble de
vecteurs deux à deux orthogonaux. 
  \begin{enumerate}[a)]
  \item   Montrer que $\{v_1,\dots,v_n\}$ est un ensemble libre si pour tout $i$, 
 $\pscal{v_i,v_i} \neq  0$.  
\item  Donner un contre-exemple ou une démonstration de la réciproque. 
\end{enumerate}


\item Considérant l'exemple~\ref{ex:2}, montrer que l'ensemble 
  \begin{displaymath}
    \{1,\sin x, \cos x, \sin(2x), \cos(2x), \sin(3x), \cos(3x), \dots\}
  \end{displaymath}
 est un ensemble de 
vecteurs deux à deux orthogonaux. 
\item Trouver la factorisation $Q\cdot R$  du Corollaire~\ref{co:2} de la matrice 
  \begin{displaymath}
    \begin{pmatrix}
      1 & 1 & 0 &0 \\
      0& 1 & 1 & 0\\
      0 & 0 & 1 & 1\\
      1 & 0 & 0 & 1\\
    \end{pmatrix}
  \end{displaymath}
  Trouver la factorisation de la matrice $n\times n$ 
  \begin{displaymath}
    \begin{pmatrix}
      1 & 1 & 0 & 0 & \cdots & 0\\
      0 & 1 & 1 & 0 & \cdots & 0 \\
      && \vdots &&\\
      0 & \cdots & \cdots&0& 1 & 1\\
      1 & 0 & \cdots &\cdots & 0 & 1
    \end{pmatrix}
  \end{displaymath}
\item Trouver une forme bilinéaire symétrique de $\R^n$ tel qu'il existe des vecteurs $u,v \in \R^n$ avec $\pscal{u,u}<0$ et $\pscal{v,v}>0$. 
\item Soit $V$ un espace vectoriel sur $\R$ muni d'une forme bilinéaire symétrique. S'il existe des vecteurs $u,v \in V$ tels que $\pscal{u,u}<0$ et $\pscal{v,v}>0$, il existe un vecteur $w \neq 0$ tel que $\pscal{w,w}=0$. 
\item Montrer que l'inégalité de Bessel (Théorème \ref{thr:4}) est une égalité si $v$ est dans le sous-espace engendré par les $v_1,\dots,v_n$. 
\item On considère l'espace euclidien des fonctions continues sur l'intervalle $[0,1]$ muni de la forme bilinéaire symétrique 
  \begin{displaymath}
    \pscal{f,g} = \int_0^1 f(x)g(x) \, dx. 
  \end{displaymath}
  \begin{enumerate}[i)]
  \item Soit $V$ le sous-espace engendré par $f(x) = x$ et $g(x) = x^2$. Trouver une base orthonormale de $V$. 
  \item Soit $V$ le sous-espace engendré par $\{1,x,x^2\}$. Trouver une base orthonormale de $V$. 
  \end{enumerate}
\item Soient $V$ un espace euclidien, $\{u_1,\dots,u_n\}$ un ensemble orthonormal et $f,g \in \spa\{u_1,\dots,u_n\}$. Montrer l'\emph{identité de Parseval}
  \begin{displaymath}
    \pscal{f,g} = \sum_i \pscal{f,u_i}\pscal{u_i,g}. 
  \end{displaymath}
\end{enumerate}


\section{La méthode des moindres carrées} 
\label{sec:le-methode-des}

Soient $A \in \R^{m \times n} $ et $b \in \R^m$ et supposons  que le système des équations linéaires 
\begin{equation}
  \label{eq:2}
  Ax = b
\end{equation}
n'a pas de solution. Dans beaucoup d'applications, on cherche un $x \in \R^n$ tel que la distance entre  $Ax$ et $b$ est \emph{minimale}. On aimerait alors résoudre le problème d'optimisation suivant
\begin{equation}
  \label{eq:3}
  \min_{x \in \R^n} \|Ax - b\|. 
\end{equation}

\begin{framed}\noindent 
  Pour le reste de ce paragraphe~\ref{sec:le-methode-des}, s'il n'est pas spécifié autrement, $V$ est toujours un espace euclidien.
\end{framed} 


\begin{theorem}
  \label{thr:3}
  Soient $v_1,\dots,v_n$ des vecteurs deux à deux orthogonaux et tels que $\|v_i\|>0$ pour tout $i$. Soit $v$ un élément de $V$ et soit $\alpha_i = \pscal{v,v_i}/\pscal{v_i,v_i}$ la composante de $v$ sur $v_i$. Pour $a_1,\dots,a_n \in \R$ alors 
  \begin{displaymath}
    \left\| v - \sum_{i=1}^n \alpha_iv_i \right\|  \leq \left\| v - \sum_{i=1}^n a_iv_i \right\|.
  \end{displaymath}
De plus, l'inégalité au-dessus est une égalité si et seulement si $a_i = \alpha_i$ pour tout $i$. 
Alors $\sum_{i=1}^n \alpha_iv_i$ est l'unique  meilleure approximation de $v$ par un vecteur du sous-espace engendré par les $v_1,\dots,v_n$. 
\end{theorem}


\begin{proof}
  \begin{eqnarray*}
    \|v- \sum_{i=1}^n a_iv_i \|^2 & = &  \|v - \sum_{i=1}^n \alpha_iv_i  - \sum_{i=1}^n (a_i - \alpha_i)v_i \|^2 \\
                           & = & \|v - \sum_{i=1}^n \alpha_iv_i \|^2 + \| \sum_{i=1}^n (a_i - \alpha_i)v_i \|^2
  \end{eqnarray*}
en utilisant le lemme~\ref{lem:2} et le théorème de Pythagore. 
\end{proof}

\noindent 
Maintenant nous pouvons décrire un \emph{algorithme} pour résoudre le problème suivant. 
\begin{framed}
  \noindent 
  Soient $v,v_1,\dots,v_n \in V$, trouver $u \in \spa \{v_1,\dots,v_n\}$ tel que la distance 
  \begin{displaymath}
    \|v - u\|
  \end{displaymath}
  est minimale. 
\end{framed}

\begin{algorithm}
\label{alg:2}

~\\
\begin{enumerate}[i)]
\item Trouver une base orthonormale $u_1,\dots,u_k$ du sous-espace $\spa\{v_1,\dots,v_n\}$ 
  avec le procédé de Gram-Schmidt. 
\item Retourner $ u = \sum_{i=1}^k \pscal{v,u_i} u_i$. 
\end{enumerate}
\end{algorithm}






\begin{theorem}
  \label{thr:6}
  Soient $A \in \R^{m\times n}$ et $b \in \R^m$. Les solutions du système
  \begin{equation}
    \label{eq:4}    
    A^TAx = A^T b
  \end{equation}
  sont les solutions  optimales du problème \eqref{eq:3}. 
\end{theorem}

\begin{proof}
  Soit $\{a^*_1,\dots,a^*_k\}$ une base orthonormale du sous-espace $\mathrm{Col}(A)=\{Ax \colon x \in \R^n\}$ engendré par les colonnes de $A$. Le théorème~\ref{thr:3} implique que  les solutions du problème \eqref{eq:3} sont les solutions du système 
  \begin{displaymath}
    A x = \sum_i \pscal{b,a^*_i} \cdot a^*_i. 
  \end{displaymath}
Le lemme~\ref{lem:2} implique que 
\begin{displaymath}
  b - \sum_i \pscal{b,a^*_i} \cdot a^*_i
\end{displaymath}
est perpendiculaire à tout $a_i^*$ et dès que les $a^*_i$ engendrent 
$\mathrm{Col}(A)$ 
on a 
\begin{displaymath}
  A^T (Ax - b) = 0
\end{displaymath}
pour toute solution optimale $x$ de \eqref{eq:3}. 


Maintenant, soit $x$ une solution du système~\eqref{eq:4}. 
Alors $ A x -b$ est 
perpendiculaire à tous les $a_i^*$. Le seul vecteur $v \in \spa\{a_1^*,\dots,a_k^*\} = \mathrm{Col}(A)$ tel que $\pscal{b-v,v}=0$  est $v = \sum_i\pscal{a_i^*,b} \cdot b$. Ceci démontre le théorème.  
\end{proof}



\begin{example}
\label{exe:4}
Trouver une solution de moindre carrées sur les données 
\begin{displaymath}
  A =
  \begin{pmatrix}
    4 & 0 \\ 0 &2 \\ 1 & 1
  \end{pmatrix}
\text{ et }  b =
\begin{pmatrix}
  2\\0\\11
\end{pmatrix}
\end{displaymath}  

\begin{displaymath}
  A^T A =
  \begin{pmatrix}
    17 & 1 \\
    1 & 5
  \end{pmatrix} \text{ et } A^T b =
  \begin{pmatrix}
    19 \\ 11
  \end{pmatrix}. 
\end{displaymath}
La solution du système 
\begin{displaymath}
   \begin{pmatrix}
    17 & 1 \\
    1 & 5
  \end{pmatrix}
  \begin{pmatrix}
    x_1\\x_2
  \end{pmatrix}
= \begin{pmatrix}
    19 \\ 11
  \end{pmatrix}.
\end{displaymath}
est $x^* = (1,2)^T$. 
\end{example}








\section{Formes linéaires, bilinéaires et l'espace dual}
\label{sec:lespace-dual}

Soient $V$ un espace vectoriel sur un corps $K$ et $V^*$ l'ensemble  des  applications linéaires de $V$ dans $K$, où on considère $K$ comme espace vectoriel de dimension $1$ sur lui-même. Clairement, $V^*$ est un espace vectoriel lui-même. 
\begin{definition}
\label{def:9}
  L'ensemble des  applications linéaires  $φ : V ⟶ K$ est noté $V^*$ et, muni de l'addition et de la multiplication scalaire usuelles, est appelé l'\emph{espace dual} de $V$. Les éléments de $V^*$ sont appelés \emph{formes linéaires}.
\end{definition}

\begin{remark}
  \label{def:10}
  Soit $V$ un espace vectoriel sur un corps $K$. Une application 
  \begin{displaymath}
    f\colon V \times V \longrightarrow K
  \end{displaymath}
  est une forme bilinéaire si et seulement si pour tout $v \in V$,
  les applications $g,h: V \longrightarrow K$
  telles que $g(x) = f(v,x)$
  et $h(x) = f(x,v)$ sont des formes linéaires.
\end{remark}

Si $V$ est de dimension finie et si $B = \{v_1,\dots,v_n\}$ est une base de $V$, l'image d'un vecteur $x = \sum_i\alpha_i v_i$ par une forme linéaire $f$ est 
\begin{eqnarray*}
    f(x) & = &  f\left(\sum_i\alpha_i v_i\right) \\
         & = & \sum_i \alpha_i f(v_i)  \\
         & = & (f(v_1),\dots,f(v_n)) [x]_B,               
\end{eqnarray*}
où $[x]_B = (\alpha_1,\dots,\alpha_n)^T$ sont les coordonnées de $x$ dans la base $B$.  





\begin{lemma}
  \label{lem:3} Supposons que $V$ est de dimension finie et $\{v_1,\dots,v_n\}$ est une base de $V$. 
  La fonction $\phi_j\colon V \longrightarrow K$ 
    \begin{displaymath}
  \phi_j \left(\sum_i \alpha_i v_i \right) =  \alpha_j 
\end{displaymath}
est une forme linéaire. 
\end{lemma}
\begin{proof}
  Immédiate. 
\end{proof}

\begin{theorem}
  \label{thr:7}
  Les formes linéaires $\{\phi_j \in V^* \colon j=1,\dots,n\}$ du lemme~\ref{lem:3} précédent  forment une base de $V^*$.
\end{theorem}

\begin{proof}
  Si, pour $\beta_i \in K$, on a $\sum_i \beta_i \phi_i = 0$, alors 
  \begin{displaymath}
 0 =    \left(\sum_i \beta_i \phi_i\right) (v_j) =  \beta_j,
  \end{displaymath}
c'est-à-dire  les $\phi_j$ sont linéairement indépendantes. Les $\phi_j$ engendrent $V^*$ puisque pour $f \in V^*$, $f = \sum_i  f(v_i)  \, \phi_i$. 
\end{proof}

\begin{definition}
  \label{def:14}
  La base $\{\phi_1,\dots,\phi_n\}$ est la \emph{base duale} de la base $\{v_1,\dots,v_n\}$. 
\end{definition}




\begin{lemma}
  \label{lem:5}
  Soit $V$
  un espace vectoriel sur le corps $K$
  de dimension finie muni d'une forme bilinéaire  non dégénéré et soit
  $f\colon V \longrightarrow K$
  une forme linéaire. Il existe un $v \in V$
  tel que $f(x) = \pscal{v,x}$ pour tout $x \in V$.
\end{lemma}

\begin{proof}
  Soient $B = \{v_1,\dots,v_n\}$ une base  de $V$ et $A_B^{\pscal{}} \in K^{n\times n}$ la matrice  associée au produit scalaire et à la base $B$. Soit $a \in K^n$ tel que  $f(x) = a^T[x]_B$ pour tout $x \in V$. Dès que $A_B^{\pscal{}}$ est de rang plein (Proposition~\ref{prop:7}), alors il existe $v \in V$ tel que $[v]_B^TA_B^{\pscal{}} = a^T$. Ceci revient à résoudre un système d'équations linéaires (cf semestre 1) et comme la matrice $A_B^{\pscal{}}$ est de rang plein, on a l'existence (et même l'unicité) d'une solution, i.e.,  $[v]_B^TA_B^{\pscal{}} = a^T$. Ainsi 
$f(x) = \pscal{v,x}$ pour tout $x \in V$. 
\end{proof}




\begin{theorem}[Supplémentaire orthogonal]
  \label{thr:8}
  Soient $V$ un espace vectoriel de dimension finie sur corps $K$ et $W$ un sous-espace de $V$. Soit $\pscal{.}$ une forme bilinéaire symétrique tel que, si restreint sur $W\times W$, elle est non dégénérée. Alors $V = W \oplus W^\perp$.  
\end{theorem}

\begin{proof}
  Pour un élément $u \in W \cap W^\perp$ on a $\pscal{u,w} = 0$ pour tout $w \in W$. Dès que $\pscal{.}$ est non dégénéré sur $W$ on a $u=0$, alors $W \cap W^\perp = \{0\}$. 

Il reste à démontrer que $V = W + W^\perp$. Pour $v \in V$ le lemme~\ref{lem:5} implique qu'il existe un $w_0 \in W$ tel que pour tout $u \in W$,  $\pscal{u,v} = \pscal{u,w_0}$ et ça démontre   $v - w_0 \in W^\perp$ et alors $v \in W + W^\perp$. 
\end{proof}

\subsection*{Exercices} 

\begin{enumerate}
\item Soient $V$ un espace vectoriel de dimension finie,  $f: V \longrightarrow K$ une forme linéaire et $B,B'$ des bases de $V$. Soit 
  \begin{displaymath}
    f(x) = a^T [x]_B 
  \end{displaymath}
  où $a \in K^n$. 
  Décrire $f(x)$ en termes de $P_{B'B}$ et $[x]_{B'}$. 
\item Soient $V$ un espace vectoriel de dimension finie,  $f: V\times V \longrightarrow K$ une forme bilinéaire et $B,B'$ des bases de $V$. Soit 
  \begin{displaymath}
    f(x,y) = [y]_B ^T A_B^f [y]_B.  
  \end{displaymath}
  Décrire $f(x,y)$ en termes de $P_{B'B}$, $[x]_{B'}$, et $[y]_{B'}$. 
\item On considère les vecteurs 
  \begin{displaymath}
    v_1 =
    \begin{pmatrix}
      1\\1\\0\\0
    \end{pmatrix} \text{ et }
 v_2 =
    \begin{pmatrix}
      0\\1\\1\\0
    \end{pmatrix} \in \Z_2^4
  \end{displaymath}
et la forme bilinéaire standard. Trouver une base du $\spa\{v_1,v_2\}^\perp$. Est-ce que $\Z_2^4 = \spa\{v_1,v_2\} \oplus \spa\{v_1,v_2\}^\perp$? 
\item Soit $V \subseteq \R[x]$ l'espace euclidien des polynômes de degré au plus $n$ muni du produit scalaire  $\pscal{p,q} = \int_0^1p(x)q(x) \, dx$.  Décrire la matrice $A_B^{\pscal{}}$ pour $B = \{1,x,\dots,x^n\}$. 
\item Soit $V$ un espace vectoriel sur un corps $K$ et soient $f,g \in V^* \setminus \{0\}$ linéairement indépendants. Montrer que
  \begin{displaymath}
    \ker{f} \cap \ker{g} 
  \end{displaymath}
est de dimension $n-2$. 
\item Soit $V$ un espace vectoriel de dimension finie sur un corps $K$ et soit $\pscal{.}$ une forme bilinéaire symétrique. Exprimez $(W_1 +W_2)^\perp$ et $(W_1\cap W_2)^\perp$ en fonction de $W_1^\perp$ et $W_2^\perp$. 


\item Donner un exemple d'un espace vectoriel  $V$ muni d'un produit scalaire dégénéré et d'un sous-espace $W \subseteq V$  tel que $V$ n'est pas la somme directe de $W$ et $W^\perp$. 
\end{enumerate}








\section{Formes sesquilinéaires et produits hermitiens}
\label{sec:form-sesq-et}

Maintenant nous considérons le cas $K = ℂ$.
Il faut un peu modifier la définition d'un produit scalaire pour
obtenir des résultats similaires à ceux des sections précédentes.  Le
carré de la longueur d'un vecteur
\begin{displaymath}
  v =
  \begin{pmatrix}
    a_1 + i\cdot  b_1 \\
    \vdots \\
    a_n + i \cdot b_n
  \end{pmatrix} \in \C^n
\end{displaymath}
où $a_i,b_i \in \R$, est égal à 
\begin{displaymath}
  \sum_i (a_i^2 + b_i^2) = \sum_i (a_i + i b_i) \cdot (a_i - i b_i) = \sum_i v_i \cdot \overline{v_i}, 
\end{displaymath}
où $v_i = a_i + i \cdot b_i$ et $\bar{v_i}$ est la conjugaison de $v_i$. Ceci suggère la définition suivante. 

\begin{definition}
  \label{def:15}
  Soit $V$
  un espace vectoriel sur un corps $\C$ et 
  \begin{displaymath}
    \pscal{,} : V × V ⟶ ℂ 
  \end{displaymath}
  une correspondance qui, à tout couple $(v,w)$
  d'éléments de $V$
  associe un nombre complexe, noté $\pscal{v,w}$. 
  On considérant 
  les propriétés suivantes: \medskip
  \begin{enumerate}[PH 1]
  \item On a $\pscal{v,w} = \overline{\pscal{w,v}}$ pour tout $v,w \in V$.  \label{ph1}
  \item Si $u,v$ et $w$ sont des éléments de $V$,  \label{ph2}
    \begin{displaymath}
      \pscal{u,v+w} = \pscal{u,v}+\pscal{u,w} \, \text{ et } \,  \pscal{v+w,u} = \pscal{v,u}+\pscal{w,u}
    \end{displaymath}
  \item Si $x \in \C$ et $u,v \in V$,  \label{ph3}
    \begin{displaymath}
       \pscal{x \cdot u , v} = x \pscal{u,v} \, \text{ et } \,   \pscal{u , x \cdot v} = \overline{x}\cdot  \pscal{u,v}.
    \end{displaymath}  
  \end{enumerate}
  \noindent
on dit que $\pscal{,}$ est 
\begin{enumerate}[i)]
\item \emph{une forme sesquilinéaire}, si $\pscal{,}$ satisfait 
  PH~\ref{ph2} et PH~\ref{ph3}. 
\item une \emph{une forme hermitienne}, si $\pscal{,}$ satisfait 
  PH~\ref{ph1}, PH~\ref{ph2} et PH~\ref{ph3}. 
\item un \emph{produit hermitien}, si   $\pscal{,}$ satisfait 
  PH~\ref{ph1}, PH~\ref{ph2} et PH~\ref{ph3} et 
  \begin{displaymath}
    \pscal{v,v} >0, \text{ pour tout } v \in V \setminus \{0\}. 
  \end{displaymath}
\end{enumerate}

Une forme  sesquilinéaire  est \emph{non dégénéré à gauche} si la condition suivante est vérifiée. 
\begin{quote}
  Si $v \in V$ et si $\pscal{v,w}=0$ pour tout $w \in V$, alors $v = 0$. 
\end{quote}

\end{definition}


\begin{remark}
Si $\pscal{,} $  est une forme hermitienne, 
pour tout $v \in V$ $\pscal{v,v} \in \R$ des que  $\pscal{v,v} = \overline{\pscal{v,v}}$.  On dit que la forme hermitienne   est \emph{définie positif} si $\pscal{v,v} >0$ pour tout $v \in V \setminus\{0\}$. Alors un produit hermitien est une forme hermitienne définie positif. 
\end{remark}

\begin{example}
  \label{exe:12}
  Le produit \emph{hermitien standard} de $\C^n$ 
  \begin{displaymath}
    \pscal{u,v} = \sum_i u_i \overline{v_i}
  \end{displaymath}
  satisfait les condition PH~\ref{ph1}-\ref{ph3} et est défini positif.  
\end{example}


Les notions d'\emph{orthogonalité, de perpendicularité, de base orthogonale} et \emph{de supplémentaire orthogonal}  sont définies comme avant, de même que les notions de \emph{coefficients de Fourier} et la \emph{projection de $v$ sur $w$}. 

\begin{example}
  \label{exe:13}
  Soit $V$ l'espace vectoriel des fonctions $f\colon \R \longrightarrow \C$   continues sur l'intervalle $[0, 2\pi]$. Pour $f,g \in V$ on pose
  \begin{displaymath}
    \pscal{f,g} = \int_0^{2\pi} f(x) \overline{g(x)} \, dx.
  \end{displaymath}
C'est un produit hermitien défini positif. Les fonctions $f_n (x)=  e^{inx}$ pour $n \in \Z$ sont orthogonales dès que 
\begin{displaymath}
  f_n(x) \overline{f_m(x)} = e^{i(n-m)x} = \cos((n-m) x) + i \cdot \sin((n-m) x)
\end{displaymath}
et alors 
\begin{displaymath}
  \pscal{f_n,f_n} = \int_0^{2 \pi} 1 \, dx = 2 \pi 
\end{displaymath}
et pour $n \neq m$ 
\begin{displaymath}
  \pscal{f_n,f_m} = \int_0^{2 \pi} \cos( (n-m) x) \, dx  + i \cdot \int_0^{2 \pi} \sin( (n-m) x) \, dx = 0.
\end{displaymath}
Pour $f \in V$ la composante de $f$ sur $f_n$, ou le coefficient de Fourier de $f$ relativement à $f_n$, est 
\begin{displaymath}
  \frac{\pscal{f,f_n}}{\pscal{f_n,f_n}} = \frac{1}{2\pi} \cdot \int_0^{2\pi} f(x) e^{-inx} \, dx. 
\end{displaymath}
\end{example}





Soient $V$ un espace vectoriel sur $\C$ de dimension finie et $f: V \times V \longrightarrow \C$ une forme sesquilinéaire. Pour une base  $B = \{v_1,\dots,v_n\}$ de $V$ et $x = \sum_i \alpha_i v_i$ et $y = \sum_i \beta_i v_i$ on a 
\begin{displaymath}
  \pscal{x,y} = \sum_{ij} \alpha_i\overline{\beta_j} f(v_i,v_j)
\end{displaymath}
et avec la matrice $A_B^f = (f(v_i,v_i))_{1 \leq i,j \leq n}$   alors 
\begin{equation}
  \label{eq:9}
  \pscal{x,y} = [x]_B ^T A_B^{f} \overline{[y]_B} 
\end{equation}
où pour un vecteur $v \in \C^n$ le vecteur $\overline{v}$ est tel que $\overline{v}_i = \overline{v_i}$ pour tout $i$. Pour une matrice $A \in \C^{m \times n}$, $\overline{A} \in \C^{m \times n}$ est la matrice telle que 
$\overline{A}_{ij} = \overline{A_{ij}}$ pour out $i,j$. 
\begin{definition}
  \label{def:17}
  Une matrice $A \in \C^{n \times n}$ est appelée \emph{hermitienne} si on a 
  \begin{displaymath}
    A = \overline{A^T}. 
  \end{displaymath}
\end{definition}


\begin{proposition}
  \label{prop:3}
  Soit  $V$  un espace vectoriel sur $\C$ de dimension finie et soit $B$  une base de $V$. Une forme sesquilinéaire $f$ est une forme hermitienne si et seulement si $A_B^f$ est hermitienne. 
\end{proposition}


\begin{definition}
  \label{def:18}
  Deux matrices $A,B \in \C^{n \times n}$ sont \emph{congruentes complexes} s'il existe une matrice inversible $P \in \C^{n \times n}$ telle que $A = {P^T} \cdot B \cdot \overline{P}$. Nous écrivons $A \cong_\C B$. 
\end{definition}
$\cong_\C$ est aussi une relation d'équivalence. On peut aussi modifier l'algorithme~\ref{alg:1} tel qu'il calcule une matrice diagonale congruente complexe par rapport à une matrice hermitienne $A \in \C^{n \times n}$,  voir l'exercice~\ref{item:7}. Alors on a le théorème suivant. 

\begin{theorem}
  \label{thr:11}
  Soit $V$ un espace vectoriel sur $\C$ de dimension finie, muni d'une forme  hermitienne. Alors $V$ possède une base orthogonale. 
\end{theorem}
\begin{proof}
  Soit $B = \{v_1,\dots,v_n\}$  une base de $V$. Pour $x,y \in V$ on a 
  \begin{eqnarray*}
    \pscal{x,y} & = & [x]_B^T A_B^{\pscal{.}} \overline{[y]_B}. 
  \end{eqnarray*}
  Soit $P \in \C^{n\times n} $ inversible telle que
  \begin{displaymath}
    P^T A \overline{P} =
    \begin{pmatrix}
      c_1\\
      & \ddots \\
      && c_n
    \end{pmatrix}. 
  \end{displaymath}
  La base orthogonale est $w_1,\dots,w_n$ dont $ [w_j]_B$ est  la $j$-ème colonne de $P$, 
  \begin{displaymath}
     [w_j]_B =
  \begin{pmatrix}
    p_{1j}\\ \vdots \\ p_{nj}
  \end{pmatrix}, 
  \end{displaymath} 
alors $w_j = \sum_{i=1}^n p_{ij} w_i$. 

\end{proof}



\begin{example}
  \label{exe:16}
  On considère la matrice hermitienne 
  \begin{displaymath}
    A = \left[\begin{matrix}0 & - i & 3 + 4 i\\i & -2 & 12\\3 - 4 i & 12 & 5\end{matrix}\right]
  \end{displaymath}
et le but est de trouver une matrice inversible $P \in \C^{3 \times 3}$ telle que 
\begin{displaymath}
  P^T \cdot A \cdot \overline{P}
\end{displaymath}
est une matrice diagonale. Nous échangeons la première et la deuxième colonne ainsi que la première et la deuxième ligne et obtenons 
\begin{displaymath}
\left[\begin{matrix}-2 & i & 12\\- i & 0 & 3 + 4 i\\12 & 3 - 4 i & 5\end{matrix}\right]. 
\end{displaymath}
Après on transforme 
\begin{displaymath}
\left[\begin{matrix}1 & 0 & 0\\- 0.5 i & 1 & 0\\6 & 0 & 1\end{matrix}\right]\cdot  \left[\begin{matrix}-2 & i & 12\\- i & 0 & 3 + 4 i\\12 & 3 - 4 i & 5\end{matrix}\right]\cdot  
\left[\begin{matrix}1 & 0.5 i & 6\\0 & 1 & 0\\0 & 0 & 1\end{matrix}\right]
  = 
\left[\begin{matrix}-2 & 0 & 0\\0 & 0.5 & 3 - 2 i\\0 & 3 + 2 i & 77\end{matrix}\right]
\end{displaymath}
La prochaine transformation est 

\begin{displaymath}
  \left[\begin{matrix}1 & 0 & 0\\0 & 1 & 0\\0 & -6 - 4 i & 1\end{matrix}\right] \cdot 
\left[\begin{matrix}-2 & 0 & 0\\0 & 0.5 & 3 - 2 i\\0 & 3 + 2 i & 77\end{matrix}\right] \cdot 
\left[\begin{matrix}1 & 0 & 0\\0 & 1 & -6 + 4 i\\0 & 0 & 1\end{matrix}\right] = 
\left[\begin{matrix}-2 & 0 & 0\\0 & 0.5 & 0\\0 & 0 & 51\end{matrix}\right]. 
\end{displaymath}
Pour 
\begin{displaymath}
P =   \left[\begin{matrix}0 & 1 & 0\\1 & 0 & 0\\0 & 0 & 1\end{matrix}\right] \cdot 
\left[\begin{matrix}1 & - 0.5 i & 6\\0 & 1 & 0\\0 & 0 & 1\end{matrix}\right]
\cdot 
\left[\begin{matrix}1 & 0 & 0\\0 & 1 & -6 - 4 i\\0 & 0 & 1\end{matrix}\right]
\end{displaymath}
on obtient 
\begin{displaymath}
  P^T \cdot A \cdot \overline{P} = \left[\begin{matrix}-2 & 0 & 0\\0 & 0.5 & 0\\0 & 0 & 51\end{matrix}\right]. 
\end{displaymath}



\end{example}
 


\subsection*{Exercices}

\begin{enumerate}
\item Soit $V$ un espace vectoriel sur $\C$ et $f\colon V\times V \longrightarrow \C$ une application satisfaisant les axiomes 

  \begin{enumerate}[i)] 
  \item On a $f({v,w}) = \overline{f({w,v})}$ pour tout $v,w \in V$.  
  \item Si $u,v$ et $w$ sont des éléments de $V$,  
    \begin{displaymath}
      f({u,v+w}) = f({u,v}) +f({u,w}) 
    \end{displaymath}
  \item Si $x \in \C$ et $u,v \in V$, 
    \begin{displaymath}
      f({x \cdot u , v}) = x f({u,v}). 
    \end{displaymath}  
  \end{enumerate}
Montrer que $f$ est une forme hermitienne. 
\item Soit $V$ un espace vectoriel sur $\C$ de dimension finie et $f: V \times V \longrightarrow \C$ une forme sesquilinéaire. Pour une base  $B = \{v_1,\dots,v_n\}$ de $V$ et $x = \sum_i \alpha_i v_i$ et $y = \sum_i \beta_i v_i$ montrer en détail que 
  \begin{displaymath}
  \pscal{x,y} = [x]_B ^T A_B^{f} \overline{[y]_B} 
\end{displaymath}
avec la matrice $A_B^f = (f(v_i,v_j))_{1 \leq i,j \leq n}$. Indiquez l'application des axiomes PH~\ref{ph2}) et PH~\ref{ph3}) dans les pas correspondants. 
\item Démontrer la proposition~\ref{prop:3}. 
\item Soit $V$ un espace vectoriel sur $\C$ de dimension $n$ et soit $\pscal{.}$ une forme sesquilinéaire. Montrer que $\pscal{.}$ est non dégénéré si et seulement si $\rank(A_B^{\pscal{.}}) = n$  pour chaque base $B$ de $V$. 
\item Montrer que $\cong_\C$ est une relation d'équivalence. \label{item:6}
\item Modifier l'algorithme~\ref{alg:1} afin qu'il calcule une matrice diagonale congruente complexe par rapport à une matrice hermitienne $A \in \C^{n \times n}$. \label{item:7}
\item Soient $V$ un espace vectoriel sur $\C$ et $\dim(V) = 3$ et $B = \{v_1,v_2,v_3\}$ une base de $V$. Avec les matrices $A_i \in \C^{3\times 3}$ décrites en bas et les applications  $f_i(x,y) = [x]_B^T A_i \overline{[y]_B}$, cocher ce qui s'applique.  


\bigskip 

  \begin{center}
 
    \begin{tabular}{|c|c|c|c|}
      \hline 
      & $A_1$ & $A_2$ & $A_3$ \\\hline 
      forme sesquilinéaire & & & \\ \hline 
      forme hermitienne & & & \\ \hline 
    \end{tabular}
  \end{center}
  
  \medskip 


  \begin{displaymath}
    A_1 = 
    \begin{pmatrix}
      2 & 1 & 3 \\
      1 & 0 & 2 \\
      3 & 2 & 0
    \end{pmatrix}, \, 
    A_2 = 
 \begin{pmatrix}
      2 & 1+i & 3 \\
      1 & 0 & 2 \\
      3 & 2 & 0
    \end{pmatrix}, \, 
     A_3 = 
    \begin{pmatrix}
      2 & 1+ 2 \cdot i & 3 - i \\
      1 - 2 \cdot i & 0 & 2-i \\
      3-i & 2+i & 0
    \end{pmatrix}. 
  \end{displaymath}


  

\end{enumerate}





\section{Espaces hermitiens} 
\label{sec:espaces-hermitiens}



\begin{framed}\noindent 
  Pour le reste de ce paragraphe, si pas spécifié autrement,  $V$
  est toujours un espace vectoriel sur $\C$
  muni d'un produit hermitien, alors $V$ est un espace hermitien. 
\end{framed}




\begin{definition}
  \label{def:h5}
  Soit $\pscal{\,}$ un produit hermitien défini positif. La \emph{longueur} ou la \emph{norme} d'un élément $v \in V$ est le nombre 
  \begin{displaymath}
    \| v \| = \sqrt{\pscal{v,v}}.
  \end{displaymath}
  Un élément $v \in V$ est un \emph{vecteur unitaire} si $\|v\| = 1$. 
\end{definition}

Aussi, l'inégalité de Cauchy-Schwarz est démontrée comme avant: 
\begin{equation}
  \label{eq:6}
  | \pscal{u,v}| \leq \|u\| \|v\|. 
\end{equation}


Les propriétés suivantes sont facilement vérifiées. 
\begin{enumerate}[i)]
\item Pour tout $v \in V$, $\|v\|\geq 0$ et $\|v\| = 0$ si et seulement si $v = 0$. \label{item:8}
\item Pour $\alpha \in \C$ et $v \in V$ on a $\| \alpha \cdot v \| = |\alpha| \cdot \|v\|$. \label{item:9}
\item Pour chaque $u,v \in V$ $\|u+v\| \leq \|u\| + \|v\|$. \label{tr:2}
\end{enumerate}

Aussi, nous avons le théorème de Pythagore, l'inégalité de Bessel et la règle du parallélogramme.  
L'équivalent du procédé de Gram-Schmidt pour les espaces hermitiens est comme suit. 


\begin{theorem}[Le procédé d'orthogonalisation de Gram-Schmidt]
\label{thr:12}
  Soit $V$ un espace hermitien et  $\{v_1,\dots,v_n\} \subseteq V$
  un ensemble libre.  
  Il existe un ensemble libre orthogonal $\{u_1,\dots,u_n\}$
  de $V$
  tel que pour tout $i$,
  $\{v_1,\dots,v_i\}$
  et $\{u_1,\dots,u_i\}$ engendrent le même sous-espace de $V$.
\end{theorem}

Comme avant, une base orthonormale est une base orthogonale consistant de vecteurs unitaires et le procédé de Gram-Schmidt nous donne le corollaire suivant. 

\begin{corollary}
\label{co:6}
Soit $V$
un espace hermitien de dimension finie. Supposons $V \neq
  \{0\}$. $V$ possède alors une base orthonormale.
\end{corollary}



\subsection*{Exercices} 

\begin{enumerate}
\item Montrer l'inégalité de Cauchy-Schwarz. 
\item Montrer l'inégalité triangulaire~\ref{tr:2}). 
\item Montrer qu'un espace hermitien de dimension finie possède une base $B$ telle que $\pscal{x,y} = [x]_B^T \cdot [y]_B$, où $\cdot$ est le produit hermitien standard. \label{item:4}
\end{enumerate}

 
%%% Local Variables:
%%% mode: latex
%%% TeX-master: "notes"
%%% End:

\chapter{Le théorème spectral et la décomposition en valeurs singulières}

\label{cha:appl-auto-adjo}

Dans ce chapitre, nous allons étudier les espaces euclidiens et hermitiens d'une manière plus profonde. Lorsque l'on parle de $\C$ ou de $\R$, nous allons utiliser la lettre $\K$ pour dénoter $\C$ ou $\R$. On va rappeler quelques notions importantes du cours du premier semestre. Un \emph{endomorphisme}  est une application linéaire $f \colon V \longrightarrow V$. Si $V$ est un espace vectoriel de dimension finie et si $B = \{v_1,\dots,v_n\}$ est une base de $V$, on a 
\begin{displaymath}
  f(x) = \phi_B^{-1} (A_B \phi_B(x)),
\end{displaymath}
où $\phi_B$ est l'ismomorphisme $\phi_B \colon V \longrightarrow K^n$, $\phi_B(x) = [x]_B$ sont les coordonnées de $x$ par rapport à la base $B$. On a le diagramme suivant 
\begin{displaymath}
  {
  \begin{CD}
    V     @>f>>  V\\
    @VV \phi_B V        @VV \phi_B V\\ 
    K^n     @>A_B \cdot x>>  K^n
  \end{CD}} 
\end{displaymath} 
Les colonnes de la matrice $A_B$ sont les coordonnées de $f(v_1),\dots,f(v_n)$ dans la base $B$. 
Une matrice $A \in K^{n \times n}$ est \emph{diagonalisable} s'il existe une matrice inversible $P \in K^{n \times n}$ telle que $P^{-1}\cdot A \cdot P$ est une matrice diagonale. 

\section{Les endomorphismes auto-adjoints} 
\label{sec:les-endom-auto}
\begin{framed}\noindent 
  Dans ce paragraphe~\ref{sec:les-endom-auto},  $V$
  est toujours un espace euclidien ou un espace hermitien de dimension finie. 
\end{framed}

\begin{definition}
\label{def:19}
Un endomorphisme $F$ est \emph{auto-adjoint} si 
\begin{displaymath}
  \pscal{F(v),w} = \pscal{v,F(w)} \text{ pour tous } v,w \in V. 
\end{displaymath}
\end{definition}

\begin{theorem}
  \label{thr:14}
  Soient $B = \{v_1,\dots,v_n\}$ une base orthonormale de $V$ et $F$ un endomorphisme. Alors $F$ est auto-adjoint si et seulement si sa matrice $A_B$ dans la base $B$ est symétrique ($\K = \R$) ou hermitienne ($\K = \C$). 
\end{theorem}

\begin{proof}
  On traite seulement le cas $\K=\C$. Le cas $\K = \R$ est démontré d'une manière analogue. Nous avons, où $\cdot$ dénote le produit hermitien standard, 
  \begin{displaymath}
    \pscal{F(v), w} = (A_B [v]_B ) \cdot [w]_B = [v]_B^T A_B^T \overline{[w]_B},
  \end{displaymath}
et 
\begin{displaymath}
  \pscal{v, F(w)}  = [v]_B^T \overline{A_B} \overline{[w]_B}. 
\end{displaymath}
Alors si $\overline{A_B} = A_B^T$, il est clair que $\pscal{F(v), w} = \pscal{v, F(w)}$ et donc $F$ est auto-adjoint. \\
Et si $F$ est auto-adjoint, en choisissant $v = v_i$, $w = v_j$, on obtient 		\begin{displaymath}
	\pscal{F(v_i), v_j} = e_i^TA_B^Te_j = (A_B^T)_{i, j} = \pscal{v_i, F(v_j)} = e_i^T \overline{A_B} e_j = (\overline{A_B})_{i, j},
	\end{displaymath}
donc $A_B^T = \overline{A_B}$
\end{proof}


\begin{lemma}
  \label{lem:8}
  Soit $A \in \C^{n \times n}$ une matrice hermitienne. Les valeurs propres de $A$ sont réelles. 
\end{lemma}
\begin{proof}
  Soient $\lambda \in \C$ une valeur propre et $v \neq 0$ son vecteur propre. Alors 
  \begin{displaymath}
    \lambda \, v^T \overline{v}  = v^T {A}^T \overline{v} = 
    v^T \overline{A \, v} = \overline{\lambda}  \, v^T \overline{v}. 
  \end{displaymath}
\end{proof}


\begin{corollary}
\label{co:3}
Soit $F$
un endomorphisme auto-adjoint, alors toutes ses valeurs propres sont
réelles.
\end{corollary}
\begin{proof}
  Soit $B = \{v_1,\dots,v_n\}$
  une base orthonormale. Les valeurs propres de $F$
  sont les valeurs propres de la matrice hermitienne $A_B$.
\end{proof}

\begin{corollary}
  \label{co:7}
  Une matrice symétrique $A \in \R^{n \times n}$ (hermitienne $A \in \C^{n \times n}$) a une valeur propre réelle. 
\end{corollary}

\begin{proof}
  Le polynôme caractéristique $p(x) = \det(A - x \cdot I_n)$ a une racine complexe selon le théorème fondamental de l'algèbre. Les valeurs propres de $A$ sont les racines de $p(x)$.  Mais toutes ces racines sont réelles selon le         corollaire~\ref{co:3}.  
\end{proof}



\begin{lemma}
  \label{lem:9}
  Soient $F$ un endomorphisme auto-adjoint et $u,v \neq 0$ deux vecteurs propres dont leurs valeurs propres  sont différentes, alors $\pscal{u,v}=0$. 
\end{lemma}

\begin{proof}
  Soient $\lambda \neq \gamma $ les valeurs propres correspondant aux vecteurs propres $u,v \neq 0$ respectivement. Puisque $\lambda,\gamma \in \R$ on a
  \begin{displaymath}
    \lambda \pscal{u,v} = \pscal{F(u),v} = \pscal{u,F(v)} = \gamma \pscal{u,v}
  \end{displaymath}
et alors $\pscal{u,v}=[u]_B \cdot [v]_B=0$, où $\cdot$ dénote le produit scalaire/hermitien standard et $B$ est une base orthonormale de $V$.  
\end{proof}



\begin{definition}
  \label{def:20}
  Une matrice inversible  $U \in \R^{n \times n}$ est \emph{orthogonale} si $ U^{-1}= U^T  $.  Une matrice inversible $U \in \C^{n \times n}$ est \emph{unitaire} si $U^{-1} = \overline{U}^T$. 
\end{definition}
\noindent 
Si $U$ est orthogonale (unitaire), les  colonnes de $U$ sont une base orthonormale de $\R^n$ ($\C^n$), où l'orthonormalité est entendue au sens du produit scalaire (hermitien) standard. 
\begin{notation}
Nous allons écrire $A^*$ pour dénoter $\overline{A}^T$ pour une matrice $A$.   
\end{notation}




\begin{theorem}[Théorème spectral]
\label{thr:16}
  Soit $A \in \K^{n\times n}$ une matrice symétrique (hermitienne), alors $A$ est diagonalisable avec une matrice orthogonale (unitaire) $P \in \K^{n \times n}$ telle que 
  \begin{equation}
    \label{diagonal}
    P^* \cdot A \cdot P =
    \begin{pmatrix}
      \lambda_1 \\
      & \ddots \\
      && \lambda_n
    \end{pmatrix}
  \end{equation}
  où $\lambda_1,\dots,\lambda_n \in \R$ sont les valeurs propres de $A$. 
\end{theorem}


\begin{proof}
  Soit $A \in \R^{n\times n}$
  une matrice symétrique. Le cas où $A \in \C^{n \times n}$
  est hermitienne est laissé en exercice.

  Le théorème est démontré par induction. Si $n=1$, l'assertion est triviale. \\
  Supposons le théorème vrai jusqu'à $n-1 \in \N, n \ge 2$. Montrons le pour $n$. \\
  Soit $\lambda_1 \in \R$
  une valeur propre de $A$
  et $v \neq 0 \in \R^n$
  un vecteur propre correspondant à $\lambda_1$.
  Avec la méthode de Gram-Schmidt, on peut trouver une base
  orthonormale $\{v,u_2,\dots,u_n\}$ dans $\R^n$. Soit $U \in \R^{n \times n-1}$ la matrice dont les colonnes sont $u_2,\dots,u_n$.  On considère la matrice
  \begin{displaymath}
    U^T \cdot A \cdot U \in \R^{n-1 \times n-1}.
  \end{displaymath}
  Cette matrice est symétrique. En effet :
  \begin{displaymath}
    (U^T \cdot A \cdot U)^T = U^T \cdot A^T \cdot (U^T)^T = U^T \cdot A \cdot U,
  \end{displaymath}
  car $A$ est symétrique. Par hypothèse d'induction, cette matrice peut être diagonalisée  avec une matrice orthogonale  $K \in \R^{ n-1 \times n-1}$, alors 
  \begin{displaymath}
    K^T \cdot  U^T \cdot A \cdot U \cdot K =
    \begin{pmatrix}
      \lambda_2 \\
      & \ddots \\
      && \lambda_n
    \end{pmatrix}. 
  \end{displaymath}
Maintenant, soit $P \in \R^{n \times n}$ la matrice 
\begin{displaymath}
  P = \left( v, U\,K\right) \in \R^{n \times n}
\end{displaymath} 
La matrice $P$ est orthogonale puisque 
\begin{displaymath}
  P^T \, P =
  \begin{pmatrix}
    v^T \, v & v^T U \, K \\
    (U\,K)^T \, v &  (U\,K)^T (U\,K)
  \end{pmatrix} = 
  \begin{pmatrix}
    1 \\
    & \ddots \\
    && 1 
  \end{pmatrix}. 
\end{displaymath}
car $v^T \cdot v = 1$, $U^T \cdot U = I_{n-1}$ et $v^T \cdot U = 0$. Et 
\begin{displaymath}
  P^T \, A \, P =
  \begin{pmatrix}
    v^T A \\
    K^T U^T A 
  \end{pmatrix}
  \begin{pmatrix}
    v & U\,K
  \end{pmatrix} =
  \begin{pmatrix}
    \lambda_1 \\
    & \ddots \\
    && \lambda_n
  \end{pmatrix}  
\end{displaymath}
\end{proof}


\begin{corollary}
  \label{co:8}
  Soit $V$ un espace euclidien (hermitien) de dimension finie et $F$ un endomorphisme auto-adjoint. Alors $V$ possède une base $\{v_1,\dots,v_n\}$ orthonormale de vecteurs propre de $F$. 
\end{corollary}

\begin{proof}
  Soit $B = \{u_1,\dots,u_n\}$ une base de $V$ tel que $\pscal{x,y} = [x]_B \cdot [y]_B$ où $\cdot $ dénote le produit hermitien standard (voir chapitre~\ref{sec:espaces-hermitiens}, exercice~\ref{item:4}) et soit $A_B^{(F)}$  la matrice symétrique (hermitienne) telle que $F(x)  = \phi_B^{-1}(A_B^{(F)} [x]_B)$. Selon théorème~\ref{thr:16}, $A_B^{(F)}$ est diagonalisable avec une matrice orthogonale (unitaire) $P$ 
  \begin{displaymath}
    P^* \cdot 
    A_B^{(F)} \cdot  P  = \begin{pmatrix}
      \lambda_1 \\
      & \ddots \\
      & & \lambda_n
    \end{pmatrix}
  \end{displaymath}
 Soient $p_1,\dots,p_n$  les colonnes de $P$. La base orthonormale de vecteurs propres de $F$ est $\{v_1,\dots,v_n\}$ où $v_i = \phi_B^{-1}(p_i)$. 
\end{proof}



Comment peut-on calculer la diagonalisation \eqref{diagonal}? Voici un procédé pour diagonaliser une matrice symétrique (hermitienne) $A \in \K^{n \times n}$. 

\begin{enumerate}[i)] 
\item Trouver les racines $\lambda_1,\dots,\lambda_k \in \R$ du polynôme caractéristique 
  \begin{displaymath}
    p(x) = \det(A - x\cdot I). 
  \end{displaymath}
\item Pour tout $j \in \{1,\dots,k\}$:
  \begin{enumerate}[a)] 
  \item Trouver une base $b^{(j)}_1,\dots,b^{(j)}_{d_j}$ du noyau de la matrice $A - \lambda_j \, I$, par exemple avec l'algorithme de Gauss. 
  \item Trouver une base orthonormale $p^{(j)}_1,\dots,p^{(j)}_{d_j}$ du $\spa\{b^{(j)}_1,\dots,b^{(j)}_{d_j}\}$, par exemple avec le procédé de Gram-Schmidt. 
  \end{enumerate}
  \item $P = \left(p^{(1)}_1,\dots,p^{(1)}_{d_1},\dots,p^{(k)}_1,\dots,p^{(k)}_{d_k}\right)$ 
\end{enumerate}




\begin{example}
  \label{exe:17}
  Soit $V$ un espace euclidien de dimension $3$ et soit $F$ un endomorphisme auto-adjoint de $V$. Soit $B=\{v_1,v_2,v_3\}$ une base telle que 
  \begin{displaymath}
    A_B =
    \begin{pmatrix}
      3 & -2 & 4\\-2 & 6 & 2\\4 & 2 & 3
    \end{pmatrix}
  \end{displaymath}
Trouver une base orthonormale qui se compose des vecteurs propres. 


\bigskip 

Le polynôme caractéristique de $A_B$ est 
\begin{displaymath}
  p(x) = \det(A_B - x\, I) =  -x^3 + 12\, x^2 -21 x - 98 = -(x-7)^2(x+2)
\end{displaymath}
On trouve les bases des espaces propres
\begin{displaymath}
  \lambda_1 = 7: \, \,
b_1^{(1)} = \begin{pmatrix}
    1 \\ 0 \\ 1 
  \end{pmatrix}, \, 
b_2^{(1)} =   \begin{pmatrix}
    -1/2\\1\\0
  \end{pmatrix} \, \quad \text{ et } 
  \lambda_2 = -2: \, \, 
  b_1^{(2)} = 
  \begin{pmatrix}
    -1 \\ -1/2 \\1
  \end{pmatrix}  
\end{displaymath}

Les vecteurs $b_1^{(1)}$
et $b_2^{(1)}$
ne sont pas orthogonaux.  Le procédé de Gram-Schmidt produit
${b_2^{(1)}}^* = (-1/4,1,1/4)^T$.
Les vecteurs $b_1^{(1)},{b_2^{(1)}}^*,b_1^{(2)}$
sont une base orthogonale de vecteurs propres. Maintenant il reste à
les normaliser et on obtient
\begin{displaymath}
p_1^{(1)} = 
\left[\begin{matrix}\frac{\sqrt{2}}{2}\\0\\\frac{\sqrt{2}}{2}\end{matrix}\right], \, 
{p_2^{(1)}} = \left[\begin{matrix}- \frac{\sqrt{2}}{6}\\\frac{2 \sqrt{2}}{3}\\\frac{\sqrt{2}}{6}\end{matrix}\right], \, 
p_1^{(2)} = \left[\begin{matrix}-\frac{2}{3}\\-\frac{1}{3}\\\frac{2}{3}\end{matrix}\right]. 
\end{displaymath}

Alors $\left\{\frac{\sqrt{2}}{2}v_1 + \frac{\sqrt{2}}{2} v_3,- \frac{\sqrt{2}}{6}v_1 + \frac{2 \sqrt{2}}{3}v_2 + \frac{\sqrt{2}}{6} v_3 , -\frac{2}{3}v_1-\frac{1}{3}v_2+\frac{2}{3}v_3\right\}$ est une base orthonormale de vecteurs propre de $F$. 

\end{example}

\subsection*{Exercices}

\begin{enumerate}
\item Soit $z = x + iy \in \C^n$, où $x,y \in \R^n$. Montrer que $x$ et $y$ sont linéairement indépendants sur $\R$ si et seulement si $z$ et $\overline{z}$ sont linéairement indépendants sur  $\C$. 
\end{enumerate}


\section{Formes quadratiques réelles et matrices symétriques réelles}
\label{sec:form-quadr-reell}

Le lemme~\ref{lem:8} et le corollaire~\ref{co:7} démontrent qu'une matrice symétrique réelle possède une valeur propre réelle. Cette démonstration passe par les nombres complexes et utilise le théorème fondamental de l'algèbre. Pour le cas où $A = A^T \in \R^{n \times n}$, nous allons maintenant démontrer l'assertion du  corollaire~\ref{co:7}  d'une manière géométrique. L'ensemble 
\begin{displaymath}
  S^{n-1} = \{x \in \R^n \colon \|x\| = 1 \} 
\end{displaymath}
est appelé la \emph{$n$-sphère.} 

\begin{definition}
  \label{def:21}
  Une \emph{forme quadratique} est une fonction
  $f\colon \R^n \longrightarrow \R$,
  $f(x) = x^T A x$
  où $A \in \R^{n \times n}$ est une matrice symétrique.
\end{definition}
En fait, si $B \in \R^{n \times n}$
n'est pas symétrique, la matrice $A = 1/2 (B^T + B)$
est symétrique et $x^TBx = 1/2 (x^TB^T x + x^T B x) = x^T Ax$.
Alors la fonction $g(x) = x^TBx$ est aussi une forme quadratique.

Une {forme quadratique} est un polynôme de degré $2$
et une fonction continue. Puisque $S^{n-1}$
est compact et $f(x)$
est continue, $f(x)$
possède un maximum sur $S^{n-1}$.
Nous sommes prêts à démontrer le lemme d'une manière géométrique.

\begin{lemma}
  \label{lem:7}
  Soient $A \in \R^{n \times n}$ une matrice symétrique et $v \in S^{n-1}$ le maximum de la fonction $f(x) = x^TAx$ sur $S^{n-1}$. On a $Av = \lambda\, v$ pour  un $\lambda \in \R$. En particulier, $A$ possède une valeur propre réelle. 
\end{lemma}

\begin{proof}
  Supposons qu'il n'existe pas de $\lambda \in \R$  tel que $A\,v  =  \lambda v$. Alors,  en particulier, $A\, v \neq 0$ et on peut écrire 
  \begin{displaymath}
    A\,v = \alpha \, v + \beta \, w
  \end{displaymath}
où $w \in S^{n-1}$, $w \perp v$ et $\beta \neq 0$. Pour $x \in [-1,1]$ on a 
\begin{displaymath}
  \sqrt{(1 - x^2)} \,v + x\, w \in S^{n-1}.
\end{displaymath}On voit facilement que $\|\sqrt{(1 - x^2)} \,v + x\, w\|=1$ en utilisant le fait que $v\perp w$, et $v,w\in S^{n-1}$.
Nous considérons  la fonction $g\colon\; [-1,1] \rightarrow \R$ 
\begin{displaymath}
  g(x) = \left(\sqrt{(1 - x^2)} \,v + x\, w \right)^T A \left(\sqrt{(1 - x^2)} \,v + x\, w \right). 
\end{displaymath}
Notons que $g(0)=f(v)$, donc si $v$ maximise $f$ sur la $n$-sphère, en particulière $x=0$ doit maximiser $g(x)$ dans l'intervalle [-1,1]. Si on démontre que $g'(0) \neq 0$, nous avons déduit une contradiction et la démonstration est faite. 

Comme $w^TAv = v^TAw$, clairement 
\begin{displaymath}
  g(x) = (1-x^2) v^T A v + (2 \cdot \sqrt{(1 - x^2)} \cdot x) \, w^T A v + x^2 w^TAw.
\end{displaymath}
Ceci démontre que $g'(0) = 2 \cdot w^T A v = 2 \cdot \beta  \neq 0$. 

\end{proof}





\begin{definition}
  \label{def:f22}
  Une matrice symétrique $A \in \R^{n \times n}$ est 
  \begin{itemize}
  \item définie positive, si $x^TAx>0$ pour tout $x \in \R^n \setminus  \{0\}$ 
  \item définie négative, si $x^TAx<0$ pour tout $x \in \R^n \setminus  \{0\}$ 
  \item semi-définie positive, si $x^TAx\geq 0$ pour tout $x \in \R^n$ 
  \item semi-définie négative, si $x^TAx\leq 0$ pour tout $x \in \R^n$.  
  \end{itemize}
  La forme quadratique $x^TAx$ correspondante est appelée définie positive, définie négative, semi-définie positive ou semi-définie négative, en accord avec $A$. 
\end{definition}



\begin{theorem}
  \label{thr:15}
  Une matrice symétrique $A \in \R^{n \times n}$ est 
  \begin{enumerate}
  \item définie positive,
    si et seulement si toutes ses valeurs propres sont strictement positives. 
  \item définie négative, si et seulement si toutes ses valeurs propres sont strictement négatives. 
  \item semi-définie positive, si et seulement si toutes ses valeurs propres sont positives (ou zéro).  
  \item semi-définie négative, si et seulement si toutes ses valeurs propres sont negatives (ou zéro).  
  \end{enumerate}
\end{theorem}

\begin{proof}
  D'après le théorème~\ref{thr:16} il existe une 
  matrice orthogonale $U \in \R^{n\times n}$ telle que 
  \begin{equation}
    \label{eq:10}    
    A = U
    \begin{pmatrix}
      \lambda_1 \\
      & \ddots \\
      && \lambda_n
    \end{pmatrix} U^T. 
  \end{equation}
  où les $\lambda_i$ sont les valeurs propres de $A$. Les colonnes  $u_1,\dots,u_n$ de $U$ forment une base orthonormale de $\R^n$  de vecteurs propres de $A$. 
  Soit $x \in \R^n$, alors $x = \sum_i \alpha_i u_i$ et 
  \begin{displaymath}
    x^TAx = \sum_i (\alpha_i^2) \lambda_i.  
  \end{displaymath}
D'ici l'assertion suit directement. 
\end{proof}

Le \emph{$k$-mineur principal} d'une matrice $A$ est le déterminant de la matrice qui est construite en choisissant les premières $k$ lignes et colonnes de $A$; c'est-à-dire $\det(B_k)$ où $B_k \in \R^{k\times k}$ telle que $b_{ij} = a_{ij}$, $1 \leq i,j \leq k$. Soit $K  = \{l_1,\dots,l_k\}\subseteq \{1,\dots,n\}$ où $l_1<l_2<\dots<l_k$. 
La matrice $B_K \in \R^{k\times k}$  est définie par $b_{ij} = a_{l_il_j}$, pour $1 \leq i,j\leq k$. Un \emph{$k$-mineur symétrique} de $A$ est le déterminant d'une matrice $B_K$; c'est-à-dire $\det(B_K)$.  

\begin{theorem}
  \label{thr:17}
  Soit $A \in  \R^{n \times n}$ une matrice symétrique.   
  \begin{enumerate}[a)]
  \item $A$ \label{posdef:1}
    est définie positive si et seulement si tous ses mineurs
    principaux sont strictement positifs.
  \item $A$ \label{posdef:2}
    est semi-définie positive si et seulement si tous ses mineurs symétriques sont non
    négatifs.
  \end{enumerate}
\end{theorem}

\begin{proof}
%Nous   considérons la factorisation~\eqref{eq:10}. La 

On démontre~\ref{posdef:1}), tandis que \ref{posdef:2}) est une exercice. 
Si $A$ est définie positive, alors 
les matrices $B_k$, $ 1 \leq k \leq n$,  sont symétriques  et définies positive. Leurs valeurs propres sont toutes positives.  Selon le théorème~\ref{thr:16}, $\det(B_k)$ est le produit des valeurs propres de $B_k$, alors $\det(B_k)>0$. 


Supposons maintenant que $\det(B_k)>0$ pour tout $k \in \{1,\dots,n\}$. Nous appliquons notre algorithme~\ref{alg:1}. Avant la $i$-ème itération, la matrice $A$ est de la forme 
\begin{displaymath}
  P^T A P = \begin{pmatrix}
      c_1 \\
      & c_2 \\
      & & \ddots & &&\\
      & & & c_{i-1} \\
      & & & &  b_{i,i} & \dots & b_{i,n} \\
%      & & & &  b_{i+1,i} & \dots & b_{i+1,n} \\
      & & & &     \vdots       &  & \vdots \\
      & & & &  b_{n,i} & \dots & b_{n,n} \\      
    \end{pmatrix}. 
\end{displaymath} 
 On observe que $b_{ii} \neq 0$. Parce que, si $b_{ii} = 0$ pour la première fois, nous n'avons jamais échangé de colonnes ou de lignes avant et  $ 0 = c_1\cdots c_{i-1} \cdot b_{ii} = \det(B_i)$, et c'est une contradiction.  
 Alors il n'est jamais nécessaire d'échanger des lignes et colonnes et notre algorithme trouve une matrice $R$, triangulaire supérieure, dont les éléments diagonaux sont tous égaux à $1$, et telle que 
\begin{displaymath}
  R^T \cdot A \cdot R =  
\begin{pmatrix}
      c_1 \\
             & \ddots & \\
       & & c_{n}
    \end{pmatrix}
\end{displaymath}
On observe que $\det(B_k) = c_1\cdots c_k$, alors toutes les $c_i$ sont positives. Alors $A$ est définie positive. 
\end{proof}


\begin{theorem}
  \label{thr:18}
  Soit $A \in \R^{n\times n}$ une matrice symétrique et $f(x) = x^TAx$ la forme quadratique correspondante à $A$. On a
  \begin{equation}
    \label{eq:11}
    \max_{x \in S^{n-1}} f(x) = \lambda_1  \, \text{ et } \,  \min_{x \in S^{n-1}} f(x)  = \lambda_n
  \end{equation}
  où $\lambda_1$ et $\lambda_n$ sont les valeurs propres maximale et minimale de $A$ respectivement. 
\end{theorem}

\begin{proof}
  Nous utilisons la factorisation 
  \begin{displaymath}
    A = P^T
    \begin{pmatrix}
      \lambda_1 \\
      & \ddots \\
      && \lambda_n
    \end{pmatrix} P
  \end{displaymath}
où $P \in \R^{n\times n}$ est une matrice orthogonale dont les colonnes sont $p_1,\dots,p_n$. Si $x = \sum_i \alpha_i \,p_i$,   alors 
\begin{displaymath}
\|x\|^2 = \sum_i \alpha_i^2   \text{ et } x^T A x =  \sum_i (\lambda_i \alpha_i^2)
\end{displaymath}
et si $\|x\|^2 = 1$, 
\begin{displaymath}
\lambda_n =  \lambda_n  \sum_i  \alpha_i^2  \leq  \sum_i (\lambda_i \alpha_i^2) \leq \lambda_1  \sum_i  \alpha_i^2 = \lambda_1. 
\end{displaymath}
Ça démontre que $p_1$ et $p_n$ sont les solutions optimales des problèmes d'optimisation~\eqref{eq:11}. 

\end{proof}



\begin{definition}
  \label{def:23}
  Soit $A \in \R^{n \times n}$ une matrice symétrique. Pour $x \in \R^n\setminus \{0\}$ le  \emph{quotient Rayleigh–Ritz } est 
\begin{displaymath}
  R_A(x) = \frac{x^TAx}{x^Tx}. 
\end{displaymath}
\end{definition}
Pour $x \in \R^n \setminus \{0\}$ $ x / \|x\| \in S^{n-1}$ et $R_A(x) = (x / \|x\| )^T \cdot A  (x / \|x\|)$. 

\begin{theorem}[Théorème Min-Max]
\label{thr:19}
Soit $A \in \R^n$ une matrice symétrique avec les valeurs propres 
$\lambda_1 \geq \dots \geq \lambda_n$.   Si $U$ dénote un sous-espace de $\R^n$ alors 
\begin{equation}
  \label{eq:12} 
  \lambda_k = \max_{ \dim(U) = k } \, \min_{x \in U \setminus \{0\}}  R_A(x)  
\end{equation}
et
\begin{equation}
  \label{eq:13}
  \lambda_k = \min_{ \dim(U) = n-k+1 } \, \max_{x \in U \setminus \{0\}}  R_A(x)  
\end{equation}
\end{theorem}

\begin{proof}
Nous démontrons \eqref{eq:12}. La partie \eqref{eq:13} est un exercice. Soit $\{u_1,\dots,u_n\}$ une base orthonormale de vecteurs propres associés à $\lambda_1 ≥ \dots ≥ \lambda_n$ respectivement. On fixe un entier $k$, et un espace $U$ de dimension $k$. Clairement $\spa\{u_k,\dots,u_n\} \cap U \supsetneq \{0\}$. Alors il existe un vecteur $0 \neq x = \sum_{i = k}^n \alpha_i u_i \in U$. Clairement $R_A(x) \leq \lambda_k$. Pour $U = \spa\{u_1,\dots,u_k\}$, 
$
  \min_{x \in U \setminus \{0\}}  R_A(x) = \lambda_k.
$ Ensemble ça démontre \eqref{eq:12}. 
\end{proof}



\subsection*{Exercices}

\begin{enumerate}
\item Une matrice réelle  symétrique, différente de la matrice zéro, telle que toute composante sur la diagonale est zéro ne peut pas être semi définie positive, ni semi définie négative. 
\item Une matrice réelle symétrique, différente de la matrice zéro, dont la diagonale est égale à zéro, possède un $2 ×2$ mineur symétrique négatif. \label{item:17}
\item Soit $L \in ℝ^{n × n}$ une matrice %triangulaire inférieure 
de la forme 
  \begin{displaymath}
L = \left(\begin{array}{c|c}
H & 0  \\
\hline
C  & I_{n-i+1} \\
\end{array}\right)
\end{displaymath}
où  $H ∈ ℝ^{i×i}$ et $i ≥0$. Soit $Q ∈ ℝ^{n ×n}$ la matrice de la permutation (transposition)  qui échange $μ, ν > i$. Montrer 
\begin{displaymath}
  Q \cdot L = \left(\begin{array}{c|c}
H & 0  \\
\hline
C'  & I_{n-i+1} \\
\end{array}\right) \cdot Q
\end{displaymath}
où $C'$ provient de $C$ en échangeant les lignes $μ-i$ et $ν-i$. 
\label{item:15}
\item \label{item:16}
En s'appuyant sur l'exercice~\ref{item:15}) montrer l'assertion suivante. Si l'algorithme~\ref{alg:1} a exécuté $k$-itérations et dans chacune de ces $k$ itérations, il existe un $j ≥i$ tel que  $b_{jj} \neq 0$ (avec la notation de la $i$-ème itération), alors il existe une matrice de permutation $Q$ telle que le résultat de ces premières $k$ itérations s'écrit 
\begin{displaymath}
  R^T Q^T A Q R, 
\end{displaymath}
où $R$ est une matrice triangulaire supérieure dont les éléments diagonaux sont $1$.  En autres mots, il existe une permutation si appliquée aux lignes et colonnes, l'algorithme~\ref{alg:1} n'échange pas de lignes et colonnes pendant ces premiers $k$ itérations. 
\item Soient $A \in \Bbb R^{n \times n}$ réelle symétrique et $A'$ la matrice obtenue de $A$ en échangeant les lignes $i$ et $j$ ($A' = Q^T A Q$ pour une matrice de permutation (transposition) $Q$). Montrer que pour tout $K \subseteq \{1, ..., n\}$, il existe $K' \subseteq \{1, ..., n\}$ tel que $\det(A_K) = \det(A'_{K'})$ (mineurs symétriques) et inversément. En d'autres termes, montrer que tous les mineurs symétriques de $A$ se retrouvent dans $A'$ et vice versa.
\item Montrer la partie b) du théorème~\ref{thr:17}. 

\emph{Indication pour montrer $⟸$:  Il existe une matrice de permutation $Q$ telle que l'algorithme~\ref{alg:1} n'échange pas de colonnes et lignes si confronté avec  $Q^T \cdot A \cdot Q$ comme input et toutes itérations sont telles que $b_{ii} \neq 0$ jusqu'à  un point, où tout le reste de la matrice est $0$. Il faut s'appuyer sur les exercices \ref{item:17} et \ref{item:16}.}

\item Une matrice symétrique $A \in \R^{n \times n}$ est définie négative, si et seulement si $\det(B_k) \neq 0$ et $\det(B_k) = (-1)^k|\det(B_k)|$ pour tout $k$. 
\item Une matrice symétrique $A \in \R^{n \times n}$ est semi-définie négative, si et seulement si $\det(B_K) = (-1)^{|K|}|\det(B_K)|$ pour tout $K\subseteq \{1,.\dots,n\}$.  
\item Montrer la partie \eqref{eq:13} du théorème~\ref{thr:19}. 
\item 
Soit $A \in \R^{n \times n}$ une matrice symétrique avec les valeurs propres $\lambda_1 \geq \dots \geq \lambda_n$. 
Soit $B_K$ une matrice comme décrite en dessus o\`u $|K| = k$ avec les valeurs propres  $\mu_1 \geq \dots  \geq \mu_{n-k}$. Pour $1 \leq i \leq k$, alors 
\begin{displaymath}
  \lambda_i \geq  \mu_i \geq  \lambda_{i+k}.
\end{displaymath}
 
\end{enumerate}








\begin{definition}
  \label{def:22}
  Une matrice hermitienne $A \in \C^{n \times n}$ est 
  \begin{itemize}
  \item définie positive, si $x^TA\overline{x}>0$ pour tout $x \in \C^n \setminus  \{0\},$ 
  \item définie négative, si $x^TA\overline{x}<0$ pour tout $x \in \C^n \setminus  \{0\},$ 
  \item semi-définie positive, si $x^TA\overline{x}\geq 0$ pour tout $x \in \C^n$,
  \item semi-définie négative, si $x^TA\overline{x}\leq 0$ pour tout $x \in \C^n$.  
  \end{itemize}
\end{definition}


Le théorème~\ref{thr:15} trouve son analogue comme suivant. La démonstration est un exercice. 

\begin{theorem}
\label{thr:20}
  Une matrice hermitienne $A \in \C^{n \times n}$ est 
  \begin{enumerate}
  \item définie positive,
    si et seulement si toutes ses valeurs propres sont (strictement) positives. 
  \item définie négative,  si et seulement si toutes ses valeurs propres sont (strictement) négatives. 
  \item semi-définie positive, si et seulement si toutes ses valeurs propres sont non-négatives (donc positives ou zéro).  
  \item semi-définie négative, si et seulement si toutes ses valeurs propres sont non-positives (négatives ou zéro).  
  \end{enumerate}
\end{theorem}


\section{La décomposition en valeurs singulières}
\label{sec:la-decomposition-en}


On commence avec un théorème qui décrit la décomposition en valeurs singulières et montre qu'elle existe. 


\begin{theorem}
  \label{thr:21}
  Une matrice $A \in \C^{m \times n}$ peut être décomposée comme 
  \begin{displaymath}
    A = P\cdot D \cdot Q
  \end{displaymath}
où $P \in \C^{m \times m}$ et $Q \in \C^{n \times n}$ sont unitaires et $D \in \R_{\geq 0}^{m \times n}$ est une matrice diagonale. Si $A$ est réelle, $P$ et $Q$ sont réelles. 
\end{theorem}


\begin{proof}
  La matrice $A^* \cdot  A$ est hermitienne et semi-définie positive dès que
  \begin{displaymath}
    x^*A^*Ax = (Ax)^* (Ax) \geq 0. 
  \end{displaymath}
Alors les valeurs propres de $A^*A$ sont non-négatives ($\lambda_i\geq 0$). Soient $\sigma_1^2 \geq \sigma_2^2 \dots \geq \sigma_n^2\geq 0$ les valeurs propres et soit $\{u_1,\dots,u_n\}$ une base orthonormale  correspondante de vecteurs propres. La matrice $Q \in \C^{n \times n}$ est la matrice dont les lignes sont $u_1^*, \dots, u_n^*$. 

Soit $ r \in \N_0$ tel que $\sigma_r >0$ et $\sigma_{r+1} = 0$; on a $\sigma_1 \geq \ldots \geq \sigma_r \geq 0=\sigma_{r+1} = \ldots = 0=\sigma_n  $. Nous construisons les vecteurs 
\begin{displaymath}
  v_i = A \, u_i  / \sigma_i, \, 1 \leq i \leq r. 
\end{displaymath}
Les $v_i$ sont orthonormaux, parce que 
\begin{displaymath}
  \|v_i\|^2 = (A \, u_i)^* A u_i / \sigma_i^2 = 1
\end{displaymath}
et pour $1 \leq i\neq j \leq r$, 
\begin{displaymath}
  v_i^* v_j = u_i^* u_j = 0. 
\end{displaymath}
Avec le procédé de Gram-Schmidt, nous complétons les $v_i$ tels que $\{v_1,\dots,v_m\}$ est une base orthonormale de $\C^m$. Les colonnes de la matrices $P$ sont alors $v_1,\dots,v_m$ dans cet ordre. La matrice $D \in \C^{m\times n}$ est la matrice diagonale dont les $r$ premières composantes sur la diagonale sont $\sigma_1,\dots,\sigma_r$ dans cet ordre. Avec ces matrices  $P,D$ et $Q$ nous avons 
\begin{displaymath}
  A = P\cdot D \cdot Q 
\end{displaymath}
ou de manière équivalente 
\begin{displaymath}
  P^* \cdot A \cdot Q^* = D,
\end{displaymath}

Nous montrons  ça en détail. Nous avons 
\begin{displaymath}
   (P^* \cdot A \cdot Q^*)_{ij} = v_i^* A u_j 
\end{displaymath}
et c'est égal à zéro si $j \geq r+1$, parce que dans ce cas $A u_j =0$ dès que $u_j^*A^*Au_j = 0$. 
Si $j \leq r$  et si $i>r$ alors $v_i$ est perpendiculaire à $A\, u_j$ et  $v_i^*Au_j = 0$. 

Et  si $i,j \leq r$, alors 
\begin{displaymath}
  u_i^* A^* A u_j / \sigma_i = u_i^* u_j \,  \sigma_j^2 / \sigma_i=
  \begin{cases}
    \sigma_i & \text{si } i=j\\
    0 & \text{autrement}.  
  \end{cases}
\end{displaymath}
\end{proof}


\begin{definition}
  \label{def:24}
  En suivant la notation du théorème~\ref{thr:21}, 
  les nombres $\sigma_1,\dots,\sigma_r$ sont les \emph{valeurs singulières} de $A$. La factorisation $A = P\cdot D \cdot Q$ est une \emph{décomposition en valeurs singulières}. 
\end{definition}




\begin{example}
  \label{exe:18}
  Trouver une décomposition en valeurs singulières de 
  \begin{displaymath}
    A =
    \begin{pmatrix}
      0 & -1.6  & 0.6 \\
      0 & 1.2 & 0.8 \\
      0 & 0 & 0 \\
      0 & 0 & 0
    \end{pmatrix}. 
  \end{displaymath}
On commence avec 
\begin{displaymath}
  A^* \cdot A =
  \begin{pmatrix}
    0 & 0 & 0 \\
    0 & 4 & 0 \\
    0 & 0 & 1
  \end{pmatrix}
\end{displaymath}
On obtient $\sigma_1 = 2, \sigma_2 = 1$ et $\sigma_3 = 0$. Les valeurs singulières sont les $\sigma_i>0$, i.e., $\sigma_1=2$ et $\sigma_2=1$. On calcule les vecteurs propres correspondant à $\sigma_1 = 2, \sigma_2 = 1$ et $\sigma_3 = 0$. La matrice $Q$ est 
\begin{displaymath}
  Q =
  \begin{pmatrix}
    0 & 1 & 0 \\
    0 & 0 & 1  \\
    1 & 0 & 0  
  \end{pmatrix}
\end{displaymath}
et
\begin{displaymath}
v_1 = \frac{1}{2} \cdot   A 
  \begin{pmatrix}
    0\\1\\0
  \end{pmatrix} =
  \begin{pmatrix}
    -0.8 \\ 0.6 \\ 0 \\ 0
  \end{pmatrix}, 
v_2 =  A 
  \begin{pmatrix}
    0\\0\\1
  \end{pmatrix} =
  \begin{pmatrix}
    0.6 \\ 0.8 \\ 0 \\ 0
  \end{pmatrix}.
\end{displaymath}
On complète avec $v_3 = e_3$ et $v_4 = e_4$, alors 
\begin{displaymath}
P =   \begin{pmatrix}-0.8 & 0.6 & 0 & 0\\0.6 & 0.8 & 0 & 0\\0 & 0 & 1 & 0\\0 & 0 & 0 & 1\end{pmatrix}
\end{displaymath}
et 
\begin{displaymath}
  \begin{pmatrix}-0.8 & 0.6 & 0 & 0\\0.6 & 0.8 & 0 & 0\\0 & 0 & 1 & 0\\0 & 0 & 0 & 1\end{pmatrix} \cdot
  \begin{pmatrix}
    2 & 0 & 0 \\
    0 & 1 & 0 \\
    0 & 0 & 0 \\
    0 & 0 & 0
  \end{pmatrix}
\cdot
\begin{pmatrix}
  0 & 1 & 0\\0 & 0 & 1\\1 & 0 & 0 
\end{pmatrix} = A. 
\end{displaymath}

\end{example}

\begin{definition}
  La \emph{pseudo inverse} d'une matrice 
  \begin{displaymath}
    D =
    \begin{pmatrix}
      \sigma_1 \\
      & \sigma_2 \\
      & & \ddots \\
      & & & \sigma_r \\
      & & & & 0 \\
      & & & & & \ddots  \\
      & & & & & & 0  \\      
    \end{pmatrix}
    \in \R^{m \times n}
  \end{displaymath}
où $\sigma_i \in \R_{>0}$ est 
\begin{displaymath}
  D^+ =  \begin{pmatrix}
      \sigma_1^{-1} \\
      & \sigma_2^{-1} \\
      & & \ddots \\
      & & & \sigma_r^{-1} \\
      & & & & 0 \\
      & & & & & \ddots  \\
      & & & & & & 0  \\      
    \end{pmatrix}
    \in \R^{n \times m}
\end{displaymath}
Toutes les composantes qui ne sont pas décrites sont zéro. 
La \emph{pseudo inverse} d'une matrice $A \in \C^{m \times n}$ avec une décomposition en valeurs singulières $A = P \cdot D \cdot Q$ est 
\begin{displaymath}
  A^+ = Q^* D^+ P^*. 
\end{displaymath}

\end{definition}


\begin{example}
  La pseudo inverse de la matrice $A$ d'exemple~\ref{exe:18} est 
  \begin{displaymath}
    A^+ = 
\begin{pmatrix}0 & 0 & 1\\1 & 0 & 0\\0 & 1 & 0\end{pmatrix} \cdot 
\begin{pmatrix}0.5 & 0 & 0 & 0\\0 & 1 & 0 & 0\\0 & 0 & 0 & 0\end{pmatrix}
\cdot
\begin{pmatrix}-0.8 & 0.6 & 0 & 0\\0.6 & 0.8 & 0 & 0\\0 & 0 & 1 & 0\\0 & 0 & 0 & 1\end{pmatrix} 
  \end{displaymath}
\end{example}

Pourquoi est-ce que nous parlons de \emph{la} pseudo inverse? Parce qu'elle est unique. 

\begin{theorem}
  \label{thr:22}
  Soit  $A \in ℂ^{m \times n}$, alors il existe au plus une seule matrice $X \in ℂ^{n \times m}$  telle que les quatre conditions de \emph{Penrose} sont satisfaites: 
  \begin{enumerate}[i)] 
  \item $AXA = A$ \label{pen1}
  \item $(AX)^* = AX$ \label{pen2}
  \item $XAX = X$ \label{pen3}
  \item $(XA)^* = XA$. \label{pen4}
  \end{enumerate}
\end{theorem}
\begin{proof}
  Soient $X$ et $Y$ deux matrices satisfaisant \ref{pen1}-\ref{pen4}. Alors
  \begin{eqnarray*}
    X & = & XAX \\
     & = & XAYAX \\
     & = & XAYAYAYAX\\
    & = & (XA)^*(YA)^*Y(AY)^*(AX)^*\\
    & = & A^*X^*A^*Y^*YY^*A^*X^*A^* \\
    & = & (AXA)^*Y^*YY^*(AXA)^* \\
    & = & A^* Y^* Y Y^* A^* \\
    & = & (YA)^* Y (AY)^* \\
    & = & YAYAY \\
    & = & YAY \\
    & = & Y. 
  \end{eqnarray*}
\end{proof}

\begin{theorem}
  \label{thr:23}
  La pseudo inverse d'une matrice $A \in \C^{m \times n}$ satisfait les conditions \ref{pen1}-\ref{pen4}. 
\end{theorem}

\begin{proof}
  Soit $A = PDQ$ une décomposition en valeurs singulières et $A^+ = Q^*D^+P^*$.  Il est facile de voir que $D^+$ satisfait les conditions \ref{pen1}-\ref{pen4} relatives à $D$. Les conditions sont aussi vite montrées pour $A$ et $A^+$. Par exemple \ref{pen1} est montrée comme suit : 
  \begin{eqnarray*}
    A A^+A & = & PDQQ^*D^+P^*PDQ \\
           & = & PDD^+DQ \\
           & = & PDQ\\
           & = & A. 
  \end{eqnarray*}
Il est un exercice de vérifier les  conditions \ref{pen2}-\ref{pen4}.
\end{proof}


\section{Encore les systèmes d'équations }
\label{sec:encore-les-systemes}
Nous considérons encore une fois un système 
\begin{equation}
  \label{eq:14}
  Ax = b, 
\end{equation}
où $A \in \C^{m \times n}$ et $b \in \C^m$. 

\begin{definition}
  \label{def:25}
  La \emph{solution minimale} de \eqref{eq:14} est 
  la solution du problème des moindres carrés
  \begin{displaymath}
    \min_{x \in \C^n} \|Ax - b\|^2 
  \end{displaymath}
    correspondant avec norme $\|x\|$ minimale.
\end{definition}



\begin{theorem}
  \label{thr:24}
  La solution minimale de \eqref{eq:14} est 
  \begin{displaymath}
    x = A^+ b,
  \end{displaymath}
où $A^+$ est la pseudo inverse de $A$. 
\end{theorem}

\begin{proof}
  On a 
  \begin{eqnarray*}
    \min_{x \in \C^n} \|Ax - b\| & = &      \min_{x \in \C^n} \|PDQx - b\| =     \min_{x \in \C^n} \|P^* (PDQx - b)\|        \\
     & = &     \min_{x \in \C^n} \|DQx - P^*b\| =     \min_{y \in \C^n} \|Dy - P^*b \| \\
      & = &       \min_{y \in \C^n} \|Dy - c \|
  \end{eqnarray*}
où $c  = P^*b$. Les solutions optimales sont les $y \in \C^n$ tels que $y_i = c_i / \sigma_i $ pour $1 \leq i \leq r$ et $y_{r+1} \dots y_n$ sont arbitraires. Parmi ces vecteurs, la solution où $y_{r+1} =\dots= y_n=0$ est celle de norme minimale. Elle est donnée par
\begin{displaymath}
  y = D^+ c. 
\end{displaymath}
La solution minimale de \eqref{eq:14} est alors 
\begin{displaymath}
  x = Q^*y = Q^* D^+ P^* b = A^+b. 
\end{displaymath}
\end{proof}

\begin{example}
  \label{exe:19}
  Trouver la solution minimale du système 
  \begin{displaymath}
     \begin{pmatrix}
      0 & -1.6  & 0.6 \\
      0 & 1.2 & 0.8 \\
      0 & 0 & 0 \\
      0 & 0 & 0
    \end{pmatrix} x =
    \begin{pmatrix}
      5\\7\\3\\-2
    \end{pmatrix}.
  \end{displaymath}

La pseudo-inverse  de la matrice ci-dessus est 
\begin{displaymath}
  A^+ = \begin{pmatrix}0 & 0 & 0 & 0\\-0.4 & 0.3 & 0 & 0\\0.6 & 0.8 & 0 & 0\end{pmatrix}
\end{displaymath}
et 
\begin{displaymath}
  \begin{pmatrix}0 & 0 & 0 & 0\\-0.4 & 0.3 & 0 & 0\\0.6 & 0.8 & 0 & 0\end{pmatrix} \cdot
  \begin{pmatrix}
    5\\7\\3\\-2
  \end{pmatrix}
 =
 \begin{pmatrix}
   0\\0.1\\8.6
 \end{pmatrix}. 
\end{displaymath}
\end{example}





\section{Le meilleur sous-espace approximatif} 
\label{sec:le-meilleur-sous}

Nous nous occupons du problème suivant. Soient $a_1,\dots,a_m \in \R^n$ des vecteurs et $1 \leq k \leq n$, trouver un sous-espace $H \subseteq \R^n$ de dimension $k$ tel que 
\begin{displaymath}
  \sum_i d(a_i,H)^2 
\end{displaymath}
soit minimale. Ici $d(a_i,H)$ est la \emph{distance} de $a_i$ a $H$. Si $H = \spa\{u_1,\dots,u_k\}$ où $\{u_1,\dots,u_k\}$  est une base orthonormale de $H$, alors $a_i = \sum_{j=1}^k \pscal{a_i,u_j} u_j + d_i$ où $d_i = a_i - \sum_{j=1}^k \pscal{a_i,u_j} u_j$ est orthogonal à $u_1,\dots,u_k$ et alors à $H$.  Avec le théorème de Pythagore (Proposition~\ref{prop:4}), on a
\begin{displaymath}
  d(a_i,H)^2 + \sum_{j=1}^k \pscal{a_i,u_j}^2 = \|a_i\|^2. 
\end{displaymath}
Le sous-espace $H$ de dimension $k$ qui minimise $\sum_i d(a_i,H)^2$ est alors celui qui maximise 
\begin{displaymath}
\sum_i  \sum_{j=1}^k \pscal{a_i,u_j}^2 = \sum_{j=1}^k \|A u_j\|^2 = \sum_{j=1}^k u_j^T A^TAu_j.
\end{displaymath}

Pour $k=1$, nous connaissons déjà une manière de résoudre ce problème. Il faut résoudre 
\begin{displaymath}
\max_{u \in S^{n-1}}   u^TA^TAu .
\end{displaymath}
La matrice $A^TA$ est symétrique, alors on peut la factoriser comme 
\begin{equation}
  \label{eq:17}
  A^TA = U
  \begin{pmatrix}
    \lambda_1 \\
    & \ddots \\
    & & \lambda_n
  \end{pmatrix} U^T
\end{equation}
où $U = \left(u_1,\dots,u_n\right) \in \R^{n \times n}$ est orthogonale. Nous pouvons supposer que les $\lambda_i$ sont ordonnés comme $\lambda_1 \geq \lambda_2 \geq \cdots \geq \lambda_n \geq 0$. Les valeurs propres sont non négatives dès que $A^TA$ est semi-définie positive. Selon Théorème~\ref{thr:18} la solution est $H = \spa\{u_1\}$. 

\medskip 
La généralisation suivante du Théorème~\ref{thr:18} est un exercice. 

\begin{theorem}
\label{thr:25}
  Soit $A \in \R^{n\times n}$ une matrice symétrique et $f(x) = x^TAx$ la forme quadratique correspondante à $A$.
Soit 
\begin{displaymath}
  A = U \cdot
  \begin{pmatrix}
    \lambda_1\\
    & \ddots \\
    & & \lambda_n
  \end{pmatrix}
U^T
\end{displaymath}
une factorisation de $A$ telle que $U = (u_1,\dots,u_n) \in \R^{n \times n }$ est orthogonale et $\lambda_1 \geq \cdots \geq \lambda_n$. 
Pour $1 \leq \ell <n$ on a
  \begin{equation}
\label{eq:15}
    \max_{\substack{x \in S^{n-1} \\ x \perp u_1, \dots, x\perp u_\ell}} f(x) = \lambda_{\ell +1}  = \min_{\substack{x \in S^{n-1} \\ x \perp u_{\ell+2}, \dots, x\perp u_n}} f(x)
  \end{equation}
et $u_{\ell+1}$ est une solution optimale. 
\end{theorem}



Maintenant, nous pouvons résoudre le problème central 
\begin{equation}
  \label{eq:16}
  \min_{\substack{H \trianglelefteq \R^n \\ \dim(H) = k}} \sum_{i=1}^m d(a_i,H)^2. 
\end{equation}

\begin{theorem}
  \label{thr:26}
  Soient $a_1,\dots,a_m \in \R^n$,
$
    A =\left(
      a_1,\cdots ,a_m\right)^T
 $
  et $u_1,\dots,u_k$  les premières colonnes de la matrice orthogonale $U \in \R^{n \times n}$ de la factorisation~\eqref{eq:17}. Le sous-espace $H = \spa\{u_1,\dots,u_k\}$ est une solution du problème~\eqref{eq:16}. 
\end{theorem}


\begin{proof}
Pour $k=1$ nous avons déjà montré l'assertion. Soit $k \geq 2$ et $W \trianglelefteq \R^n$ une solution optimale du problème~\eqref{eq:16} et soit $w_1,\dots,w_k$ une base orthonormale de $W$. 
Nous pouvons supposer $w_k \perp \spa\{u_1,\dots,u_{k-1}\}$, (voir Exercice~\ref{item:3}). 

Par induction, nous avons 
\begin{displaymath}
  \sum_{j=1}^{k-1} w_j^TA^TAw_j \leq  \sum_{j=1}^{k-1} u_j^TA^TAu_j.
\end{displaymath}
Dès que   
\begin{displaymath}
  \max_{\substack{x \in S^{n-1} \\ x \perp \spa\{u_1,\dots,u_{k-1}\}}} x^TA^TAx 
\end{displaymath}
est atteint à $u_k$ nous avons 
\begin{displaymath}
  w_k^TA^TAw_k \leq u_k^TA^TAu_k
\end{displaymath}
et alors
\begin{displaymath}
  \sum_{j=1}^{k} w_j^TA^TAw_j \leq  \sum_{j=1}^{k} u_j^TA^TAu_j.
\end{displaymath}

\end{proof}


\begin{definition}
  \label{def:26}
  Soit $A \in \R^{m \times n}$. La \emph{norme Frobenius} de $A$ est le nombre 
  \begin{displaymath}
    \|A\|_F = \sqrt{\sum_{ij}a_{ij}^2}.
  \end{displaymath}
\end{definition}

\begin{definition}
  \label{def:27}
  Soit $A \in K^{n \times n}$. La \emph{trace} de $A$ est la somme de ses coefficients diagonaux, $\Tr(A) = \sum_{i=1}^n a_{ii}$. 
\end{definition}

\begin{lemma}
  \label{lem:11}
  Pour $A,B \in K^{n \times n}$ $\Tr(AB) = \Tr(BA)$. 
\end{lemma}


\begin{lemma}
  \label{lem:10}
  Pour $A \in \R^{m \times n}$, $\|A\|_F^2 = \sum_{i=1}^r \sigma_i^2$ où $\sigma_1, \dots, \sigma_r$ sont les valeurs singulières de $A$. 
\end{lemma}

\begin{proof}
  On a $\|A\|_F^2 = \Tr(A^TA) = \Tr(U \cdot \diag(\sigma_1^2,\dots,\sigma_n^2) \cdot  U^T)$ où $U \in \R^{n \times n}$ est orthogonale. Alors 
  \begin{displaymath}
    \|A\|_F^2 = \Tr(\diag(\sigma_1^2,\dots,\sigma_n^2)) =  \sum_{i=1}^r \sigma_i^2
  \end{displaymath}
\end{proof}




Maintenant nous allons résoudre le problème suivant. Étant donnés $A \in \R^{m \times n}$ et $k \in \N$, trouver une matrice $B \in \R^{m \times n}$ de $\rank(B) \leq  k$ tel que 
\begin{displaymath}
  \|A - B\|_F
\end{displaymath}
soit minimale. 

Si $A = P \cdot \diag(\sigma_1,\dots,\sigma_r,0,\dots 0) \cdot Q$ est une décomposition en valeurs singulières où les colonnes de $P$ sont $v_1,\dots,v_m$ et les lignes de $Q$ sont $u_1^T,\dots,u_n^T$ on peut écrire
\begin{equation}
  \label{eq:18}
  A = \sum_{i=1}^r \sigma_i v_iu_i^T
\end{equation}
et on dénote la somme des premiers $k$ termes comme 
\begin{displaymath}
  A_k = \sum_{i=1}^k \sigma_i v_iu_i^T
\end{displaymath}
Le rang de $A_k$ est au plus $k$. 

\begin{lemma}
  \label{lem:12}
  Les lignes de $A_k$ sont les projections des lignes de $A$ dans le sous-espace $V_k = \spa\{u_1,\dots,u_k\}$. 
\end{lemma}

\begin{proof}
  Soit $a^T$ une ligne de $A$. La projection de $a$ dans le sous-espace $\spa\{u_1,\dots,u_k\}$ est 
  \begin{displaymath}
    \sum_{i=1}^k a^Tu_i \cdot u^T_i.
  \end{displaymath}
Alors les projections des lignes de $A$ dans le sous-espace $V_k$ sont données par $\sum_{i=1}^k A u_i u_i^T = \sum_{i=1}^k \sigma_i v_i u_i^T = A_k$.  
\end{proof}

\begin{theorem}
  \label{thr:27}
  Pour une matrice $B \in \R^{m \times n}$ de rang plus petit ou égal à $k$, on a 
  \begin{displaymath}
    \|A - A_k\|_F \leq \|A - B\|_F.
  \end{displaymath}
\end{theorem}

\begin{proof}
On dénote les lignes de $A$ par $a_1^T,\dots,a_m^T$ et soit $B$ une matrice de rang au plus $k$. Les lignes de $B$ sont dénotées comme $b_1^T,\dots,b_m^T$. Soit $H = span\{b_1,\dots,b_m\}$. La dimension de $H$ est $\rank(B) \leq k$. On a 
\begin{displaymath}
  \|A - B\|_F^2 = \sum_{i=1}^m \|a_i - b_i\|^2 \geq \sum_{i=1}^m d(a_i,H)^2.
\end{displaymath}
Soit $\wt{H} = \spa\{u_1,\dots,u_k\}$. Nous avons démontré que 
\begin{enumerate}[i)]
\item $\wt{H}$ est le meilleur sous-espace approximatif des lignes de $A$ alors $\sum_{i=1}^m d(a_i,H)^2 \geq  \sum_{i=1}^m d(a_i,\wt{H})^2$ et 
\item Les lignes de $A_k$ sont les projections des lignes de $A$ dans $\wt{H}$. 
\end{enumerate}
En dénotant les lignes de $A_k$ par $\wt{a}_1^T,\dots,\wt{a_m}^T$, alors 
\begin{displaymath}
  \|A - B\|_F^2 \geq  \sum_{i=1}^m d(a_i,\wt{H})^2 = \sum_{i=1}^m \|a_i - \wt{a}_i\|^2 = \|A - A_k\|_F^2. 
\end{displaymath}
\end{proof}

\subsection*{Exercices}

\begin{enumerate}
\item Est-ce que la décomposition en valeurs singulières est unique? Est-ce que les valeurs singulières sont uniques?  \label{item:2}
\item  Dans la démonstration du théorème~\ref{thr:21}, montrer que $\rank(A) = r$. 
\item Démontrer que la pseudo-inverse satisfait les conditions \ref{pen2}-\ref{pen4}. 
\item Si $Ax = b$ a plusieurs solutions, il existe une solution unique avec une norme minimale. 
\item Montrer Théorème~\ref{thr:25}. 
\item Soient  $G,H \subseteq \R^n$ des sous-espaces de $\R^n$ 
et $k=\dim(G) > \dim(H)$. Montrer que $G$ possède une base orthonormale $w_1,\dots,w_k$ telle que $w_k \perp H$. \label{item:3}  
\end{enumerate}






%%% Local Variables:
%%% mode: latex
%%% TeX-master: "notes"
%%% End:

\chapter{Systèmes différentiels linéaires}
\label{cha:syst-diff-line}

On considère le système suivant
\begin{equation}
  \label{eq:19}
  \begin{array}{ccccc}
    \x'_1(t) & = &a_{11}\x_1(t) & + \cdots + & a_{1n}\x_n(t) \\
    \x'_2(t) & = &a_{21}\x_1(t) & + \cdots + & a_{2n}\x_n(t) \\
            & \vdots \\             
    \x'_n(t) & = &a_{n1}\x_1(t) & + \cdots + & a_{nn}\x_n(t)
  \end{array}
\end{equation}
où les $a_{ij} \in \R$.  
En notation de vecteur matrice on peut écrire ça comme 
\begin{displaymath}
  \x' = A\,\x
\end{displaymath}
où 
\begin{displaymath}
  A =
  \begin{pmatrix}
    a_{11} & \cdots & a_{1n}\\
          & \vdots & \\
          a_{n1} & \cdots & a_{nn}\\          
  \end{pmatrix}.
\end{displaymath}

On cherche des fonctions dérivables $\x_i: \R \longrightarrow \R$  qui, ensembles constituent $\x$ et qui  satisfont \eqref{eq:19}. Un tel $\x$ est une \emph{solution} du système~\eqref{eq:19}. 



\begin{example}
  \label{exe:20}
  Considérons l'équation différentielle $\x'(t) = \x(t)$. Une solution est $\x(t) = e^t$. Une autre solution est $\x(t) = 2\cdot e^t$. Si on spécifie la \emph{condition initiale} $\x(0) = 1$, alors $\x(t) = e^t$ est la solution qui satisfait cette condition initiale. Autrement, si on spécifie $\x(0) = \alpha$, alors  $\x(t) = \alpha \cdot e^t$ est une solution qui satisfait la condition initiale. 

Considérons $\x'(t) = -\x(t)$, une solution est $\x(t) = e^{(-t)}$. %$\x(t) = \cos(t)$.
C'est aussi une solution qui respecte la condition initiale $\x(0) = 1$. 
\end{example}



Essayons d'abord de résoudre le système comme suivant $\x(t) = e^{\lambda t} v$ où $v \in \R^n$ est un vecteur constant. Dans ce cas $\x' = A\x$ se récrit comme   $\lambda e^{\lambda t}  v = e^{\lambda t} v$. Nous avons démontré le lemme suivant. 

\begin{lemma}
  \label{thr:29}
  Si $\lambda \in \R$ est une valeur propre de $A$ et si $v \in \R^n \setminus \{0\}$ est un vecteur propre correspondant, alors $\x(t) = e^{\lambda t} v$ est  une solution du système~\eqref{eq:19} pour les conditions initiales $\x(0) = v$. 
\end{lemma}


Le théorème suivant est démontré en cours \emph{analyse 2}. 
\begin{theorem}[Cours d'analyse II] 
  \label{thr:28}
  Étant données les \emph{conditions initiales} $\x(0)$
  il existe une solution $\x$ unique du système~\eqref{eq:19}. 
\end{theorem}
Nous sommes concernés avec le problème de \emph{trouver} la solution $\x$ explicitement. On commence avec une observation qui est un exercice simple. 

\begin{lemma}
  \label{lem:13}
  L'ensemble $\X = \{ \x \colon \x \text{ est une solution du système \eqref{eq:19}}\}$ est un espace vectoriel sur $\R$.  
\end{lemma}

Est-ce que c'est possible de donner  une base de $\X$ explicitement? Dans le cas où $A$ est diagonalisable comme 
\begin{displaymath}
  A = P \cdot \diag(\lambda_1,\dots,\lambda_n) \cdot P^{-1} 
\end{displaymath}
où $P \in \R^{n \times n}$ est inversible et les  $\lambda_i \in \R$ sont particulièrement agréables comme c'est décrit dans le lemme suivant. 

\begin{theorem}
  \label{thr:30}
  Si $\R^n$ possède une base $\{v_1,\dots,v_n\} \subseteq \R^n$ de vecteurs propres de $A$ telle que $A \, v_i = \lambda_i v_i$, alors  
  \begin{displaymath}
    \x^{(i)}(t) = e^{\lambda_i t} \cdot v_i, \, i=1,\dots,m
  \end{displaymath}
est une base de $\X$. 
\end{theorem}

\begin{proof}
  Montrons d'abord que les $\x^{(i)}$ sont linéairement indépendants. Supposons que $\sum_{i} \alpha_i \x^{(i)} = 0$. C'est à dire que les $n$ fonctions qui sont les composantes de $\sum_{i} \alpha_i \x^{(i)}$ sont toutes la fonction $f(x) = 0$. Dès que $e^{\lambda_i 0} = 1$, ça implique que 
  \begin{displaymath}
    0 = \sum_i \alpha_i v_i e^{\lambda_i 0} = \sum_i \alpha_i v_i.  
  \end{displaymath}
Mais les $v_i$ sont linéairement indépendants. Alors $\alpha_i = 0$ pour tout $i$ ce qui démontre que les $\x^{(i)}$ sont linéairement indépendants. 


Maintenant soit $\y \in \X$ et soient  $\alpha_i \in \R$  tels que 
\begin{displaymath}
  \y(0) = \sum_i \alpha_i v_i.  
\end{displaymath}
Alors $\x = \sum_i \alpha_i \x^{(i)} \in \X$  et dès que $\x(0) = \y(0)$, le Théorème~\ref{thr:28} implique $\x = \y$. C'est à dire que les $\x^{(i)}$ engendrent $\X$, alors $\{\x^{(1)},\dots, \x^{(n)}\}$ est une base de $\X$. 
\end{proof}


Est-ce qu'on peut aussi trouver une solution dans la cas où $A$ est diagonalisable dans les nombres complexes, donc si 
\begin{displaymath}
A=P \cdot \diag(\lambda_1,\dots,\lambda_n) \cdot P^{-1} 
\end{displaymath}
où $P \in \C^{n \times n}$ est inversible et les  $\lambda_i \in \C$? Pour discuter de ça, il faut d'abord définir, ce qu'est une solution complexe du système~\eqref{eq:19}. Toute fonction $f: \R \longrightarrow \C$ s'écrit comme 

\begin{displaymath}
  f(x) = f_{\Re}(x) + i \cdot f_{\Im}(x) 
\end{displaymath}
où $f_{\Re}(x), f_{\Im}(x)$ sont des fonctions de $ \R \longrightarrow \R$.  Si $f_\Re$ et $f_\Im$ sont dérivables, on dit que $f(x)$ est dérivable et on définit 
\begin{displaymath}
  f'(x) = f_\Re'(x) + i \cdot f_\Im'(x). 
\end{displaymath}
Si $\x_1,\dots,\x_n\colon \R \longrightarrow \C$ sont dérivables, comme avant 
\begin{displaymath}
  \x =
  \begin{pmatrix}
    \x_1\\ \vdots \\ \x_n
  \end{pmatrix}
\end{displaymath}
est une \emph{solution complexe} du système~\eqref{eq:19} si $\x' = A\x$.   Et comme avant, on peut noter le lemme suivant, en se rappellant que $e^{a + ib} = e^a (\cos b + i \cdot \sin b)$. 


\begin{lemma}
  \label{lem:15}
  Si $\lambda \in \C$ est une valeur propre de $A$ et si $v \in \C^n \setminus \{0\}$ est un vecteur propre correspondant, alors $\x(t) = e^{\lambda t} v$ est  une solution du système~\eqref{eq:19} pour les conditions initiales $\x(0) = v$. 
\end{lemma}
\begin{proof}
  On écrit 
  \begin{displaymath}
    \x' = \lambda e^{\lambda t} v = e^{\lambda t} Av = A\x.
  \end{displaymath}
\end{proof}


\begin{lemma}
  \label{lem:14}
  Étant donnée une solution complexe $\x = \x_\Re + i \x_\Im$ du système~\eqref{eq:19}, alors $\x_\Re$ et $\x_\Im$ sont des solutions réelles.  
\end{lemma}
\begin{proof}
  Dès que $\x_\Re + i \x_\Im $ est une solution, on a 
  \begin{displaymath}
   \x'_\Re+ i \x'_\Im = \x' = A \x = A\x_\Re + A\x_\Im. 
  \end{displaymath}
Dès que $A $ est réelle on voit $\x_\Re' = A\x_\Re$ et $\x'_\Im = A \x_\Im$. 
\end{proof}


Supposons alors que $A \in \R^{n \times n}$ est diagonalisable . Et soit $\{v_1,\dots,v_n\}$ une base de $\C^n$ de vecteurs propres associés à 
$\lambda_1,\dots,\lambda_n$ respectivement. Si $v_i = u_i + i \cdot w_i$  où $u_i,w_i \in \R^n$, les $u_1,\dots,u_n,w_1,\dots,w_n$ engendrent $\R^n$, voir exercice~\ref{item:5}. Comme nous avons noté 
\begin{displaymath}
  \x^{(j)} = e^{\lambda_j t} v_j
\end{displaymath}
sont des solutions complexes du système~\eqref{eq:19}.   

Aussi, on peut supposer que la base et les valeurs propres sont tels que les vecteurs/valeurs propres complexes viennent en paires conjugées complexes. Plus précisément
\begin{equation}
\label{eq:22}
  v_{2j-1} = \overline{v_{2j}}\, \text{ et } \, \lambda_{2j-1} = \overline{\lambda_{2j}} \, \text{ pour } \, 1 \leq j \leq k \leq n/2 
\end{equation}
et 
\begin{equation}
  \label{eq:23}  
  v_j \in \R^n, \lambda_j \in \R \text{ pour } j > 2k. 
\end{equation}
%
Considérons maintenant une solution impliquée par $v = u+iw$  $\lambda= a+ib$. 
\begin{eqnarray*}
  \x & = & e^{a \, t} \left(\cos (b t)  + i \sin (b t ) \right)  (u + i w)  \\
   & = & e^{a \, t} \left(\cos (b t) u - \sin (bt)w \right)  + ie^{a \, t} \left(\sin (b t ) u + \cos(bt)w \right). 
\end{eqnarray*}
Ça nous donne alors ces deux solutions réelles 
\begin{eqnarray*}
  \x^{(1)} & = & e^{a \, t} \left(\cos (b t) u - \sin (bt)w \right) \\
  \x^{(2)} & = &  e^{a \, t} \left(\sin (b t ) u + \cos(bt)w \right). 
\end{eqnarray*}
\begin{remark}
  \label{rem:2}
  Les solutions réelles impliquées par $v$ et $\lambda$ sont les mêmes que les solutions réelles impliquées par $\overline{v}$ et $\overline{\lambda}$. 
\end{remark}

Nous pouvons alors noter une marche à suivre pour résoudre le système~\eqref{eq:19} étant donné $\x(0)$ si $A$ est diagonalisable.
\begin{enumerate}
\item Trouver une base de vecteurs propres $v_1,\dots,v_n$ de $A$ ordonnée comme dans \eqref{eq:22} et \eqref{eq:23}. 
\item Pour chaque paire $v_{2j},\lambda_{2j}$, $1 \leq j \leq k$ trouver les  deux solutions réelles dénotées comme  $\x^{(2j-1)}$ et $\x^{(2j)}$. 
\item Pour chaque paire réelle $v_j, \lambda_j$ $n\geq j>2k$, trouver la solution $\x^{(j)}$. 
\item Trouver la combinaison linéaire 
  \begin{displaymath}
    \x(0) = \sum_{j} \alpha_j \x^{(j)}(0) 
  \end{displaymath}
\item La solution est 
  \begin{displaymath}
    \x = \sum_{j} \alpha_j \x^{(j)} 
  \end{displaymath}
\end{enumerate}



\begin{example}
  \label{exe:21}
  Résoudre le système $\x' = A\x$ où 
  \begin{displaymath}
    A  =
    \begin{pmatrix}
      1 & 2 \\
      -2 & 1
    \end{pmatrix} \, \text{ et } \, \x(0) =
    \begin{pmatrix}
      1\\1
    \end{pmatrix}. 
  \end{displaymath}
 On trouve que $\lambda_1 = 1 + 2 i$ et $\lambda_2 = 1 - 2i$ sont les valeurs propres de $A$ et 
 \begin{displaymath}
   v_1 =
   \begin{pmatrix}
     1\\i
   \end{pmatrix} \text{ et } v_2 =
   \begin{pmatrix}
     1 \\ -i
   \end{pmatrix}
 \end{displaymath}
sont les vecteurs propres correspondants. 
Les deux solutions impliquées par $v_1$ sont 
\begin{eqnarray*}
  \x^{(1)} & = & e^{ t} \left(\cos ( 2t)
                 \begin{pmatrix}
                   1\\0
                 \end{pmatrix}
- \sin (2t)\
  \begin{pmatrix}
    0\\1
  \end{pmatrix}
\right) \\
  \x^{(2)} & = &  e^{t} \left(\sin ( 2t )
                 \begin{pmatrix}
                   1\\0
                 \end{pmatrix}
+ \cos(2t)
  \begin{pmatrix}
    0\\1
  \end{pmatrix}
\right). 
\end{eqnarray*} 
La solution qu'on cherche est 
\begin{displaymath}
  \x = \begin{pmatrix}
           e^{t} \sin (2t) + e^{t} \cos(2t) \\
           - e^{t} \sin(2t) + e^{t} \cos(2t)
         \end{pmatrix}.
\end{displaymath}

\end{example}






% Une situation très agréable est si $A$ est diagonalisable. Soit $P^{-1}AP = \diag(\lambda_1,\dots,\lambda_n)$ où $P = (v_1,\dots,v_n)$. 
% Avec le changement de variables $\x = P\cdot \y $ on écrit
% \begin{displaymath}
%   \y' = P^{-1} \x' = P^{-1} A\x = P^{-1}AP\y = \diag(\lambda_1,\dots,\lambda_n) \y. 
% \end{displaymath}
% Le système 
% \begin{equation}
%   \label{eq:20}  
%   \y' = \diag(\lambda_1,\dots,\lambda_n) \y 
% \end{equation}
% est découplé et les conditions initiales sont $\y(0) = P^{-1} \x(0)$. La solution est $\y_i(t) = \y_i(0) e^{\lambda_i \cdot t}$ et $\x = P \cdot \y$ est la solution du système~\eqref{eq:19} pour les conditions initiales $\x(0)$. 

% \begin{remark}
%   \label{rem:1}
%   Notez que, même si $A$
%   est diagonalisable, seulement dans les nombre complexes, les
%   fonctions $\x$ sont réelles.
% \end{remark}





\subsection*{Exercices} 

\begin{enumerate}
\item Montrer Lemme~\ref{lem:13}. 
\item Une fonction $f:\C \longrightarrow \C$ est \emph{holomorphe} en $z_0 \in \C$ si
  \begin{displaymath}
    f'(z_0) = \lim_{z \rightarrow z_0} \frac{f(z) - f(z_0)}{z - z_0} 
  \end{displaymath}
existe. Soit $f$ holomorphe sur $\C$ et $g = f_{|\R}$ la fonction $f$ réduite à $\R$. 
 Montrer 
\begin{enumerate}[i)]
\item  $g(x) = g_\Re(x) + i \cdot g_\Im(x)$ est dérivable au  sens de notre définition, particulièrement $g_\Re(x)$ et $ g_\Im(x)$ sont dérivables. 
\item $f'_{| \R} (x) = g'_\Re(x) + i \cdot g'_\Im(x)$. 
\end{enumerate}
\item Soit $\{u_1+ i \cdot w_1,\dots,u_n + i \cdot w_n\}$ une base de $\C^n$ où $u_i,w_i \in \R^n$  pour tout $i$. Montrer que $\spa\{u_i, w_i \colon 1 \leq i \leq n\} = \R^n$. \label{item:5}
\end{enumerate}






\section{L'exponentielle d'une matrice}
\label{sec:lexp-dune-matr}



\begin{definition}
  \label{def:28}
  Pour $A \in \C^{n \times n}$ on définit 
  \begin{displaymath}
    e^A = I + A + \frac{1}{2!} A^2 + \frac{1}{3!}A^3 + \cdots 
  \end{displaymath}
\end{definition}

\noindent On rappelle la définition d'une série intégrable 
\begin{displaymath}
  \sum_{j=0}^\infty a_j z^j,
\end{displaymath}
où les coefficients $a_j \in\C$, et
qui converge sur un \emph{disque} de rayon $\rho$. C'est à dire que, si $|z|< \rho$ la série converge et la fonction $f\colon \{x \in \C \colon |x| < \rho \}  \rightarrow \C$ définie par $f(x) = \sum_{j=0}^\infty a_j x^j $ est \emph{holomorphe} avec dérivée $f'(x) =  \sum_{j=0}^\infty j a_j x^{j-1}$. 
Une série intégrable importante est la série
\begin{displaymath}
  e^{x} = \sum_{j=0}^\infty \frac{1}{j!} x^j,
\end{displaymath}
qui définit la fonction holomorphe $\exp: \C \longrightarrow \C$ 
\begin{displaymath}
  e^{a+i\,b} = e^a (\cos b + i \sin b).  
\end{displaymath}

\noindent On va maintenant généraliser la définition de la \emph{norme Frobenius} pour les matrices complexes. Pour $A \in \C^{m\times n}$, 
\begin{displaymath}
  \|A\|_F = \sqrt{\sum_{ij} |a_{ij}|^2 }. 
\end{displaymath}

\begin{lemma}
  \label{lem:16}
  Pour $A \in \C^{n \times m}$ et $B \in \C^{m × n}$  on a 
  \begin{displaymath}
    \|A\cdot B\|_F \leq \|A\|_F\cdot \|B\|_F. 
  \end{displaymath}
\end{lemma}

  \begin{proof}Soient $a_1^T,\dots,a_n^T \in \C^m$ les lignes de $A$ et $\overline{b_1},\dots,\overline{b_n} \in \C^m$ les colonnes de $B$. Avec Cauchy-Schwarz 
    \begin{displaymath}
          |(AB)_{ij}|^2 = (a_i^T \overline{b_j})(\overline{a_i}^T b_j)  \leq \|a_i\|^2 \|b_j\|^2
    \end{displaymath}
et donc 
\begin{displaymath}
  \|AB\|_F^2 = \sum_{ij} |(AB)_{ij}|^2 \leq \sum_i\|a_i\|^2 \cdot \sum_i \|b_i\|^2 = \|A\|_F^2 \cdot \|B\|_F^2. 
\end{displaymath}
  \end{proof}


  \begin{lemma}
    \label{lem:17}
    La série $e^A$ converge. 
  \end{lemma}

  \begin{proof}
Il est facile de se convaincre que la convergence pour la norme de Frobenius revient à prouver que la suite est de Cauchy. 

Evidemment $\forall c \in \mathbb{C}$ la série $\sum _{ j=0 }^{ +\infty  }{ \frac { { x }^{ j } }{ j! } ={ e }^{ j } } $ converge. La suite des sommes partielles est donc de Cauchy.

Considérons maintenant la suite $(b_n)_{n\in \mathbb{N}} = { (\sum _{ j=0 }^{ n }{ \frac { { A }^{ j } }{ j! } ) }  }_{ n\in \mathbb{N} }$. Soient $m,n \in \mathbb{N}$ alors $\parallel b_{ m }-b_{ n }\parallel _{ F }=\parallel \sum _{ j=n+1 }^{ m }{ \frac { A^{ j } }{ j! }  } \parallel _{ F }\quad \le \quad \sum_{j=n+1}^{m}{\frac{\parallel A\parallel_{F}^{j}}{j!}} \le \quad \epsilon $ pour $n,m \ge N_{\epsilon}$ (qui existe car la suite des sommes partielles de la série $\sum _{ j=0 }^{ +\infty }{ \frac { \parallel A\parallel _{ F }^{ j } }{ j! }  } = e^{\parallel A\parallel _{ F }^{ j }} $ est de Cauchy.)
  \end{proof}


Nous avons montré que $e^{At} = \sum_{k=0}^\infty \frac{t^k}{k!} A^k$ converge pour tout $t \in \R$. Plus précisément chaque composante $\sum_{k=0}^\infty \frac{t^k}{k!} A^k$ est une série intégrable avec un rayon de convergence $\infty$. Alors, nous pouvons dériver les éléments pour obtenir 
\begin{equation}
  \label{eq:24}
  \frac{d}{dt} e^{At} = A e^{At}. 
\end{equation}


\begin{theorem}
  \label{thr:31}
  La solution du problème initial $\x' = A\x$, $\x(0) =v$ est 
  \begin{displaymath}
    \x(t) = e^{At} v.
  \end{displaymath}
\end{theorem}

\begin{proof}
  Soit $\x(t) = e^{At} v$. Alors $\x'(t) = A e^{At}v = A\x(t)$. Plutôt $\x(0) = v$. 
\end{proof}

\begin{definition}
  \label{def:29}
  Une matrice $N$ est \emph{nilpotente} s'il existe un $k \in \N$ tel que $N^k = 0$. 
\end{definition}

Nous allons montrer ce théorème dans le prochain cours. 
\begin{theorem}
  \label{thr:32}
  Chaque matrice $A \in \C^{n \times n}$ peut être factorisée comme 
  \begin{displaymath}
    A = P ( \diag(\lambda_1,\dots,\lambda_n) + N) P^{-1}
  \end{displaymath}
où $N \in \C^{n \times n}$ est nilpotente, $P \in \C^{n \times n}$ est inversible,  $\lambda_1,\dots,\lambda_n \in \C$ sont les valeurs propres de $A$ et $\diag(\lambda_1,\dots,\lambda_n)$ et $N$ commutent. 
\end{theorem}


\begin{lemma}
  \label{lem:18}
  Pour $A,B \in \C^{n \times n}$, si $A\cdot B = B \cdot A$ on a $e^{A+B} = e^A e^B$. 
\end{lemma}



Comment peut-on maintenant résoudre le problème initial $\x' = Ax, \, \x(0) = v$ explicitement? Nous savons que cette solution est $\x = e^{tA} \cdot v$ et nous savons que c'est une solution réelle pour $A \in \R^{m \times n}$. Mais les premiers termes s'écrivent comme 
\begin{displaymath}
  \sum_{i=0}^m t^i A^i = P \left(\sum_{i=0}^m t^i \diag(\lambda_1,\dots,\lambda_n)^i + t^i N \right) P^{-1} 
\end{displaymath}
où nous avons utilisé le théorème \ref{thr:32}. Dès que $N$ et $\diag(\lambda_1,\dots,\lambda_n)$ commutent, la solution \emph{réelle} que l'on cherche est  
\begin{eqnarray*}
  \x & = &  P e^{t\diag(\lambda_1,\dots,\lambda_n)} e^{tN} P^{-1} v \\
     & = & P \left( \diag(e^{\lambda_1 \, t},\dots,e^{\lambda_n \, t})\cdot  \sum_{j=0}^{k-1} t^j N^{j} / j!\right)P^{-1},
\end{eqnarray*}
où $k \in \N$ est tel que $N^k = 0$. 

\section{Polynômes}
\label{sec:polyn-les-lalg}

Soit $K$ un corps. 
On dénote l'anneau des polynômes de $K$ par $K[x]$. 
Un élément de $K[x]$ s'écrit comme 
\begin{displaymath}
  p(x) = a_0 + a_1 x + \cdots + a_n x^n 
\end{displaymath}
où les \emph{coefficients} $a_i \in K$. 

La formule de multiplication de deux polynômes $f(x) = a_0+a_1x+ \cdots a_n x^n$ et $g(x) = b_0 + \cdots + b_m x^m$  est 
\begin{equation}
\label{eq:21}
  f(x) \cdot g(x) = \sum_{i = 0}^{m+n} \left(\sum_{k+l = i}  a_{k} b_l\right) x^i
\end{equation}


\begin{definition}
  \label{def:30}
  Un polynôme $f(x) \in K[x]$
  tel que $\deg(f) \geq 1$ est \emph{irréductible} si
  \begin{displaymath}
    f(x) = g(x) \cdot h(x) 
  \end{displaymath}
implique $\deg(g) \cdot \deg(h) = 0$, alors un des facteurs est une constante. 
\end{definition}




\begin{definition}
  \label{def:33}
  Un diviseur commun de $a(x) \in K[x]$
  et $b(x) \in K[x]$
  est un diviseur de $a(x)$
  et $b(x)$.
  Un diviseur commun le plus grand de $a(x)$
  et $b(x)$
  est un diviseur commun de $a(x)$
  et $b(x)$
  tel que tous les autres  diviseurs communs de $a(x)$ et $b(x)$ le divisent. On dénote les plus grands diviseurs communs de $a$ et $b$ par $\emph{pgdc}(a,b)$ (ou, en anglais, $\emph{gcd}(a,b)$, greatest common divisor).
\end{definition}


\begin{theorem}
  \label{thr:36}
  Soient $a(x),b(x)$ deux polynômes, tels que $\left\{ a,b \right\} \neq \left\{ 0 \right\}$. Un polynôme 
  \begin{equation}
    \label{eq:25}   
    d(x) = g(x) a(x) + h(x) b(x) \neq 0
  \end{equation}
  où $g,h \in K[x]$ 
  de degré minimal est un plus grand diviseur commun de $a$ et $b$. 
\end{theorem}

\begin{proof}
  On montre qu'un tel $d(x)$ est un diviseur commun de $a$ et $b$ en procédant par l'absurde. Supposons que $d$ ne divise pas $a$. Alors il existe $q$ et $r$ tels que 
  \begin{displaymath}
    a = q\cdot d +r 
  \end{displaymath}
et $\deg(r) < \deg(d)$. Alors 
\begin{displaymath}
  r = a - q\cdot d = (1 - g\,q) a - h\,q\,b
\end{displaymath}
est un polynôme de la forme~\eqref{eq:25} avec un degré strictement plus petit que celui de $d$. 

Il est clair que tous les diviseurs communs de $a$ et $b$ divisent $d$. 
\end{proof}




\begin{theorem}
  \label{thr:39}
  Soit $p(x)$ irréductible et supposons que $p(x) \mid f(x) \cdot  g(x)$, alors $p(x)\mid f(x)$ ou $p(x) \mid g(x)$. 
\end{theorem}

\begin{proof}
Si $p(x)$ ne divise ni $f(x)$ ni $g(x)$ alors $1 = f(x) h_1(x) + p(x)h_2(x)$ et 
$1 = g(x) h_3(x) + p(x) h_4(x)$ alors $\gcd(p(x), f(x)g(x))=1$. 
\end{proof}


\begin{theorem}
  \label{thr:40}
  Un polynôme  $f(x) \in K[x]$, $f(x) ≠ 0$  a une factorisation 
  \begin{displaymath}
    f(x) = a^* \prod_j p_j(x)
  \end{displaymath}
  où $a^* \in K$ et les $p_j(x)$ sont irréductibles avec coefficient dominant $1$. Cette factorisation est unique sauf pour des permutations des $p_j$. 
\end{theorem}





\begin{definition}
  \label{def:34}
  Soient $V$ un espace vectoriel sur un corps $K$, $A: V \rightarrow V$ un endomorphisme et $f(x) = a_0+ \cdots + a_n x^n\in K[x]$. L'\emph{évaluation de $f$ sur $A$} est l'endomorphisme $f(A): V \rightarrow V$ 
  \begin{displaymath}
    f(A) = a_n A^n + a_{n-1}A^{n-1}+ \cdots + a_1 A + a_0 \mathrm{id},
  \end{displaymath}
  où $A^n = \underbrace{A \circ A \circ \dots \circ A}_{n \text{ fois}}$.
\end{definition}



\begin{definition}
  \label{def:35}
  Soient $A:V \rightarrow V$ un endomorphisme et $W \subseteq V$ un sous-espace de $V$. On dit que $W$ est \emph{invariant sous $A$} si $A(x) \in W$ pour tout $x \in W$. 
\end{definition}



\begin{lemma}
  \label{lem:20}
  Soient $f(x) ∈ K[x]$ et $A:V ⟶  V$ un endomorphisme,  alors $\ker(f(A))$ est invariant sous $A$. 
\end{lemma}


\begin{proof}
Si $v ∈ \ker(f(A))$ on trouve que $f(A)\,Av = Af(A)\,v = 0$. Alors, $Av ∈ \ker(f(A))$. 
\end{proof}


\begin{theorem}
  \label{thr:37}
  Soit $A: V \rightarrow V$ un endomorphisme et soit $f(x) = f_1(x) \cdot f_2(x)$ tel que
  \begin{enumerate}[i)]
  \item $\deg(f_1) \cdot \deg(f_2) \neq 0$,
  \item $\gcd(f_1,f_2) = 1$ 
  \end{enumerate}
  alors 
  $
      \ker(f(A)) = \ker(f_1(A)) \oplus \ker(f_2(A)) 
   $.   
\end{theorem}
\begin{proof}
  Dès que $\gcd(f_1,f_2)=1$ il existe $g_1(x),g_2(x)$ tels que 
  \begin{displaymath}
    1 = g_1(x) f_1(x) + g_2(x) f_2(x)
  \end{displaymath}
  et alors 
  \begin{equation}
    \label{eq:26}   
    g_1(A) \cdot f_1(A) +  g_2(A) f_2(A) = I. 
  \end{equation}
  Pour $v \in \ker(f(A))$, alors 
\begin{displaymath}
   g_1(A) \cdot f_1(A) \cdot v  + g_2(A) f_2(A) \cdot v  = v. 
\end{displaymath}
Mais $g_1(A) \cdot f_1(A) \cdot v \in \ker(f_2(A))$  dès que 
\begin{displaymath}
  f_2(A) \cdot g_1(A) \cdot f_1(A) \cdot v =   g_1(A) \cdot f_1(A) ⋅ f_2(A) \cdot v = g_1(A) f(A) v = 0
\end{displaymath}
et d'une manière similaire on voit que $g_2(A) f_2(A) \cdot v \in \ker(f_1(A))$. Il reste à démontrer que la somme est directe. 

Soit alors $v ∈  \ker(f_1(A))  ∩  \ker(f_2(A))$.  L'équation~\eqref{eq:26} montre 
\begin{displaymath}
  v =  g_1(A) \cdot f_1(A) \, v+  g_2(A) f_2(A) \, v = 0,
\end{displaymath}
qui démontre que la somme est directe. 
\end{proof}



\begin{theorem}
\label{thr:min-poly}
Soient $V$ un espace vectoriel de dimension finie sur un corps $K$, et $A: \, V \rightarrow V$ un endomorphisme.
Il y a un polyn{\^o}me $m_A(x) \in K[x]$ de degr{\'e} minimal tel que $m_A(A) = 0$ et le coefficient dominant de $m_A(x)$ est $1$. En plus,
\begin{enumerate}
\item $m_A(x)$ est unique,
\item si $p(A) = 0$ pour un $p \in K[x]$, alors $m_A(x)$ divise $p$, et
\item pour $\lambda \in K$, on a $m_A(\lambda) = 0$ si et seulement si $p_A(\lambda) = 0$, o{\`u} $p_A(x)$ est le polyn{\^o}me charact{\'e}ristique de $A$.
\end{enumerate}
\end{theorem}
\begin{definition}
On appelle $m_A(x)$ de Th{\'e}or{\`e}me~\ref{thr:min-poly} le \emph{polyn{\^o}me minimal de $A$}.
\end{definition}
\begin{proof}
\underline{Existence:}
Car $V$ est de dimension finie, l'espace vectoriel des endomorphismes $V \rightarrow V$ est de dimension finie.
Alors, il existe un $k \in \N$ minimal tel que les endomorphismes
\[
 \mathrm{id}, A, A^2, \dots A^k
\]
sont lin{\'e}airement d{\'e}pendants,
$
0 = \sum_{j=0}^k \alpha_j A^j$ pour quelques $\alpha_j \in K$, et $\alpha_k \neq 0$.
On d{\'e}finit $m_A(x) = \sum_{j=0}^k (\alpha_j / \alpha_k) x^j$.
Car on choisit $k$ minimal, $\{\mathrm{id},\dots,A^{k-1}\}$ est un ensemble libre et alors $m_A(x)$ est de degr{\'e} minimal.

\underline{$i)$ et $ii)$:} Soit $p(x) \in K[x]$ polyn{\^o}me quelconque tel que $p(A) = 0$.
Car il y a deux polyn{\^o}mes $g,h$ tel que 
\[
\gcd (m_A, p) (x) = g(x)m_A(x) + h(x) p(x) \neq 0,
\]
on a $\gcd (m_A, p) (A) = 0$.
Car $m_A$ est de degr{\'e} minimal, on a $\deg (\gcd(m_A,p)) = \deg (m_A)$, et car le polyn{\^o}mes ont le coefficient dominant $1$, on a $m_A(x) = \gcd (m_A, p) (x)$, alors $ii)$ est vrai.
De plus, si $\deg (p) = \deg (m_A)$, on a $p = \gcd (p, m_A)$ aussi, ce qui implique $i)$.

\underline{$iii)$}
Car $m_A \mid p_A$, si $m_A(\lambda) = 0$ alors $p_A(\lambda) = 0$.

Si $p_A(\lambda) = 0$, il existe $v \in V$ tel que $Av = \lambda v$.
Alors,
\[
0 (v) = m_A(A) (v) = \left( \sum_{j=0}^k a_j A^j \right) (v) = \sum_{j=0}^k a_j A^j(v) = \left( \sum_{j=0}^k a_j \lambda^j \right) (v) = m_A (\lambda) v,
\]
et on a $iii)$.
\end{proof}




\subsection*{Exercices}
\label{sec:exercices}
\begin{enumerate}
\item Montrer que $K[x]$ est un anneau avec $1_{K[x]} = 1_K$.  
\item Montrer que  $a(x) \in K[x]$ et $b(x) \in K[x]$ $\deg(a)+\deg(b)>0$  possèdent exactement un diviseur commun le plus grand avec coefficient principal égal à $1_K$.  
\item Soit $V$ un espace vectoriel de dimension fini sur $ℂ$,  $T : V ⟶V$ un endomorphisme et $f(x) = (x - λ)^m ∈ ℂ[x]$. Montrer que $\ker(f(T)) \neq \{0\}$ si et seulement si $λ$ est un valeur propre de $T$. 
\end{enumerate}



\section{La forme normale de Jordan}
\label{sec:la-forme-normale}



\begin{definition}
  Un \emph{bloc Jordan} est une matrice de la forme 
  \begin{displaymath}
    \begin{pmatrix}
      λ & 1 \\
        & λ & 1 \\
        &   & \ddots & \ddots \\ 
        &   &             & λ & 1 \\
        &   &         &  & λ  \\
    \end{pmatrix}
  \end{displaymath}
où les éléments non décrits sont zéro. 

Une matrice $A \in \C^{n \times n}$ est en \emph{forme normale de Jordan} si $A$ est en forme bloc diagonale, où tous les blocs sur la diagonale sont des blocs Jordan, i.e. $A$ est de la forme
\begin{displaymath}
  A =
  \begin{pmatrix}
    B_1 \\
        & B_2 \\
        &    & \ddots \\
        &    &       & B_k
  \end{pmatrix}
\end{displaymath}
où les matrices $B_j \in \C^{n_j\times n_j}$ sont des blocs de Jordan. 
\end{definition}


Notre but est de montrer le théorème suivant. 

\begin{theorem}
  \label{thr:41}
  Soit $A \in \C^{n \times n}$, alors il existe des matrices $P,J \in \C^{n \times n}$ telles que $J$ est en forme normale de Jordan, $P$ est inversible et 
  \begin{displaymath}
    A = P^{-1} \,J \,P. 
  \end{displaymath}
\end{theorem}

\begin{definition}
  \label{def:36}
  Soit $V  = \C^n$. Le \emph{décalage}  est l'application linéaire 
  \begin{displaymath}
    U
    \begin{pmatrix}
      x_1 \\ \vdots \\ x_n
    \end{pmatrix}
     = 
     \begin{pmatrix}
       x_2 \\ x_3 \\ \vdots \\ 0
     \end{pmatrix}. 
  \end{displaymath}
  Le décalage plus une constante est aussi une application linéaire 
  \begin{displaymath}
    U + \lambda \cdot I. 
  \end{displaymath}
\end{definition}

Il est facile de voir que la matrice représentant le décalage plus $λ$ est 
un seul bloc de Jordan 
\begin{displaymath}
 \begin{pmatrix}
      λ & 1 \\
        & λ & 1 \\
        &   & \ddots & \ddots \\ 
        &   &             & λ & 1 \\
        &   &         &  & λ  \\
    \end{pmatrix}.   
\end{displaymath}



\begin{lemma}
  \label{lem:19}
  Soit $V$ un espace vectoriel de dimension finie sur $\C$ et soit $T\colon V ⟶ V$ une application linéaire. Alors $V$ est la somme directe de sous-espaces   $V = V_1 ⊕ \cdots ⊕ V_K$  tels que 
  \begin{enumerate}[i)]
  \item $T(V_i) ⊆ V_i$ pour tout $i$ et \label{item:10}
  \item $T_{∣V_i} \colon V_i ⟶ V_i$
    est de la forme $N_i + λ \, I$ où $N_i$ est nilpotente. \label{item:11}
  \end{enumerate}
\end{lemma}

\begin{proof}
%  Dès que l'espace des applications linéaires sur $V$ est un espace vectoriel sur $\C$ de dimension finie, il existe un $k \in \N$ tel que 
%  \begin{displaymath}
%    I, \, T, \, T^2, \dots, T^k
%  \end{displaymath}
%sont linéairement dépendants. De plus, il existe un polynôme $p(x) \in \C[x] \setminus \{ 0 \}$ tel que $p(T) = 0$. En effet, on peut choisir $p(x)$ le polynôme 
%caractéristique de $T$.
Soit $p(x) = m_T(x)$ le polyn{\^o}me minimal de $T$, alors $p(T) = 0$.
Le coefficient dominant de $p(x)$ est $1$.   Le théorème fondamental de l'algèbre implique que 
\begin{displaymath}
  p(x) = ( x - λ_1)^{m_1} \cdots ( x - λ_k)^{m_k} 
\end{displaymath}
avec des $λ_i$ différents. 
Le diviseur le plus grand de $( x - λ_i)^{m_i}$ et $p(x) / ( x - λ_i)^{m_i}$ est $1$ pour $i ≠ j$. En utilisant théorème~\ref{thr:37} en $k-1$ étapes, alors 
\begin{displaymath} 
V =   \ker p(T) = \ker (T - λ_1I)^{m_1} ⊕  \cdots ⊕ \ker( T - λ_kI)^{m_k}
\end{displaymath}
et avec $V_i = \ker( T - λ_iI)^{m_i}$ on a $V = V_1 ⊕ \cdots ⊕ V_K$  et \ref{item:10}) avec lemme~\ref{lem:20}. \newline

De plus, $$T_{∣V_i} = (T - λ_iI)_{∣V_i} + λ_iI_{∣V_i} =\colon N_i + λ_iI$$ et $N_i = (T - λ_iI)_{∣V_i}$ est bien nilpotente, car $V_i = \ker( T - λ_iI)^{m_i}$ et donc $N_i^{m_i} = (T - λ_iI)^{m_i}_{∣V_i} = 0$.
\end{proof}


\begin{remark}
  \label{rem:3}
  Lemme~\ref{lem:19} démontre qu'il existe une base 
  \begin{displaymath}
  \mathscr{B} =   b_1^1,\dots,b_{\ell_1}^1,b_1^2,\dots,b_{\ell_2}^2,\dots,b_1^k,\dots,b_{\ell_k}^k
  \end{displaymath}
  où $b_1^i,\dots,b_{\ell_i}^i$ est une base de $V_i$ telle que la matrice $A^T_\mathscr{B}$ de $T$ par rapport à la base $\mathscr{B}$ est une matrice bloc diagonale 
  \begin{displaymath}
      A^T_{\mathscr{B} }  =
      \begin{pmatrix}
        B_1 \\
        & B_2 \\
        & & \ddots \\
        & &  & B_k
      \end{pmatrix}
  \end{displaymath}
et les matrices $B_i \in \C^{\ell_i × \ell_i}$ sont de la forme $B_i = N_i + λ_i I$ où les $N_i$ sont  nilpotentes. 

Rappel: Si  $\phi_{\mathscr{B}}$ est l'ismorphisme $\phi_{\mathscr{B}} \colon V \longrightarrow \C^n$, où $\phi_{\mathscr{B}}(x) = [x]_{\mathscr{B}}$ sont les coordonnées de $x$ par rapport à la base ${\mathscr{B}}$,  on a le diagramme suivant 
\begin{displaymath}
  {
  \begin{CD}
    V     @>T>>  V\\
    @VV \phi_{\mathscr{B}} V        @VV \phi_{\mathscr{B}} V\\ 
    \C^n     @>A^T_{\mathscr{B}} \cdot x>>  \C^n
  \end{CD}} 
\end{displaymath} 
\end{remark}

Il est clair, qu'il faut s'occuper maintenant des applications linéaires 
\begin{displaymath}
  T_{∣V_i} \colon V_i ⟶ V_i 
\end{displaymath}
qui sont de la forme $N + λ I$ pour une application nilpotente $N$. Le théorème suivant s'occupe des applications linéaires nilpotentes. La matrice de $λI$ est toujours $λ I$ pour chaque base. Il est alors clair que le théorème suivant démontre le théorème~\ref{thr:41}. 

\begin{theorem}
  \label{thr:38}
  Soit $V$ un espace vectoriel sur $\C$ de dimension finie et $N\colon V ⟶V$  une application linéaire nilpotente.  Alors $V$ possède une base $ℬ$ de la forme 
  \begin{displaymath}
    x_1,Nx_1, \dots, N^{m_1-1}x_1, x_2,Nx_2, \ldots , N^{m_2-1}x_2, \quad \dots \quad , x_k,Nx_k, \dots, N^{m_k-1}x_k
  \end{displaymath}
telle que $N^{m_i}x_i = 0$ pour tout $i$. 
\end{theorem}

\begin{remark}
  \label{rem:4}
  Si on inverse l'ordre de la base $ℬ$ et si on liste les éléments de droite à gauche on obtient une base $ℬ'$ et la matrice $A_{ℬ'}^{N }$ de l'application $N$  a la forme 
  \begin{displaymath}
    A_{ℬ'}^N =
    \begin{pmatrix}
      J_1 \\
      & J_2 \\
      & & \ddots \\
      & & & J_k
    \end{pmatrix}
  \end{displaymath}
en forme normale de Jordan, où 
\begin{displaymath}
  J_i =
  \begin{pmatrix}
    0 & 1 \\
    &  0 & 1 \\
    &    & \ddots & \ddots \\
    &    &        & 0 & 1 \\
    & & & & 0
  \end{pmatrix} \in \C^{m_i × m_i}. 
\end{displaymath}
Par conséquent, $N+ λI$ est représentée par 
\begin{displaymath}
 A_{ℬ'}^{N + λI} =    A_{ℬ'}^N + λ I_n
\end{displaymath}
en forme normale de Jordan. 
\end{remark}


\begin{proof}[Démonstration du Théorème~\ref{thr:38}] 
  Pour $x \in V \setminus \{0\}$ on appelle 
  \begin{displaymath}
    m_x = \min \{ i \colon N^ix = 0\}
  \end{displaymath}
  la \emph{durée de vie} de $x$. 
  La séquence 
  \begin{displaymath}
    x, Nx, \dots, N^{m_x-1} x
  \end{displaymath}
  est l'\emph{orbite} de $x$ sous $N$. 
  
  En concaténant les orbites des éléments d'une base et en travaillant sur cet ensemble, nous obtiendrons un ensemble de vecteurs qui engendre $V$. 
  Supposons alors qu'au début de l'étape $q$, nous avons un ensemble $x_1,\dots,x_\ell$ avec $x_1,\dots,x_\ell \neq0$  dont les orbites 
  \begin{equation}
    \label{eq:27}
    x_1,Nx_1,\dots,N^{m_1-1}x_1, \,\dots \, ,  x_\ell,Nx_\ell,\dots,N^{m_\ell-1}x_\ell
  \end{equation}
  engendrent $V$ (pour la première étape, on prend $\ell = n$ avec des $x_i$ formant une base de $V$). Ici $m_i$ est la durée de vie de $x_i$. Si~\eqref{eq:27} est linéairement dépendant, nous allons soit supprimer un $x_i$ et son orbite (car superflus), soit remplacer un $x_i$ par un vecteur $y$ tel que 
  \begin{enumerate}[i)]
  \item Les orbites de $x_1,\dots, x_{i-1},y,x_{i+1},\dots,x_{\ell}$ engendrent aussi l'ensemble  $V$, 
  \item la somme des durées de vie de $x_1,\dots, x_{i-1},y,x_{i+1},\dots,x_{\ell}$  est strictement plus petite que la somme des durées de vie de $x_1,\dots,x_{\ell}$. 
  \end{enumerate}
Cela prouvera le théorème parce qu'un tel procédé doit se terminer. \\

Dès que l'ensemble~\eqref{eq:27} est linéairement dépendant, il existe une combinaison linéaire non triviale de~\eqref{eq:27} qui est égale a $0$ :
\begin{displaymath}
0 =   β_0^1 x_1 + β_1^1 Nx_1+ \dots+ β_{m_1-1}^1 N^{m_1-1}x_1 + \dots + 
β_0^\ell x_\ell + β_1^\ell Nx_\ell+ \dots+ β_{m_\ell-1}^\ell N^{m_\ell-1}x_\ell 
\end{displaymath}

\textbf{Cas 1 :} \\
Supposons que dans notre ensemble $x_1,Nx_1,\dots,N^{m_1-1}x_1, \,\dots \, ,  x_\ell,Nx_\ell,\dots,N^{m_\ell-1}x_\ell$, il existe $i$ tel que la durée de vie de $x_i$ est $1$ (i.e. $Nx_i = 0$ et l'orbite associée est seulement constituée de $x_i$) et supposons que ce $x_i$ apparaisse (avec un coefficient non nul) dans la combinaison linéaire ci-dessus. \\
En passant tous les termes sauf $x_i$ à gauche, on obtient  que $x_i$ est une combinaison linéaire non triviale des éléments de $\{ x_1,Nx_1,\dots,N^{m_1-1}x_1, \,\dots \, ,  x_\ell,Nx_\ell,\dots,N^{m_\ell-1}x_\ell \} \setminus \{x_i\}$. Donc on peut supprimer $x_i$ de cet ensemble et on obtient un nouvel ensemble de la même forme qu'en~\eqref{eq:27}, engendrant le même espace, mais avec une orbite en moins. \\

\textbf{Cas 2 :} \\
Supposons que nous avons la combinaison linéaire ci-dessus, mais que nous ne sommes pas dans le cas $1$. \\
Maintenant, nous allons appliquer l'application $N$ $k$-fois, où $k \geq 0$ est le plus grand entier tel que les termes 
\begin{displaymath}
  β_i^j N^{k+i}x_j 
\end{displaymath}
ne sont pas tous égaux à zéro. Ainsi, nous avons trouvé un sous-ensemble $J ⊆ \{1,\dots,\ell \}$ et des $γ_j ≠ 0$ tels que 
\begin{displaymath}
  \sum_{j \in J} γ_j N^{m_j-1}x_j = 0.
\end{displaymath}
Soit $m = \min_{j \in J} {m_j-1} \geq 1$ et soit $i \in J$ un index où le minimum est atteint. Alors 
\begin{displaymath}
 0 =  N^m  \sum_{j \in J} γ_j N^{m_j-1 - m}x_j  = N^m \left( γ_i x_i + \sum_{j \in J, j \neq i} γ_j N^{m_j-1 - m}x_j \right)
\end{displaymath}

Maintenant, posant $$y = \sum_{j \in J} γ_j N^{m_j-1 - m}x_j = γ_i x_i + \sum_{j \in J, j \neq i} γ_j N^{m_j-1 - m}x_j$$ 
Si $y \neq 0$, on remplace $x_i$ par $y$.
Il est alors facile de voir que les orbites de 
\begin{displaymath}
  x_1,\dots, x_{i-1},y,x_{i+1},\dots,x_\ell
\end{displaymath}
engendrent encore $V$. Et la durée de vie de $y$ est au plus $m<m_i$. \\
Sinon, les orbites de
\begin{displaymath}
  x_1,\dots, x_{i-1},x_{i+1},\dots,x_\ell
\end{displaymath}
suffisent alors à engendrer $V$.

On a alors démontré le théorème.  
\end{proof}



\subsection*{Exercices} 

\begin{enumerate}
\item Montrer que les \textbf{orbites} de 
$ x_1,\dots, x_{i-1},y,x_{i+1},\dots,x_\ell $ engendrent encore $V$. (Voir démonstration du théorème~\ref{thr:38}). 

\item Le but de cet exercice est de faire la preuve du Théorème~\ref{thr:41} "à l'envers". \newline
Soit $T \colon V \rightarrow V$ un endomorphisme. Soit $\phi : V \rightarrow \Bbb C^n$ l'isomophisme associé à une base $B$ de $V$ et à la base canonique $E$ de $\Bbb C^n$. Supposons que $A_{B} = ([T(b_1)]_E, ..., [T(b_n)]_E)$, la matrice de $T$ relativement à la base $B$, admette une forme normale de Jordan $J$ avec matrice de passage $P = (p_1, ..., p_n)$. \newline
Montrer qu'il existe des sous-espaces $V_1, ..., V_k$  de $V$ tels que pour tout $i$ :
\begin{enumerate}
\item $V = V_1 \oplus \cdots \oplus V_k$;
\item $V_i = \phi(\text{span}(p_{k_i}, ..., p_{k_i + l_i}))$;
\item $T(V_i) \subset V_i$;
\item $T_{∣V_i} = N_i + \lambda_i I$, où $N_i \colon V_i \rightarrow V_i$ est nilpotente;
\item $\{\lambda_1, ..., \lambda_k\} = \{J_{11}, ..., J_{nn}\}$.
\end{enumerate} 

\item Le but de cet exercice est de montrer les propriétés des décompositions comme dans le Lemme~\ref{lem:19}. \newline
Soit $T \colon V \rightarrow V$ un endomorphisme et soit $V_1, ..., V_k$ une décomposition de $V$ tel que $V = V_1 ⊕ \cdots ⊕ V_k$, $T(V_i) \subset V_i$ et $T_{∣V_i} = N_i + \lambda_i I$, où $N_i : V_i \rightarrow V_i$ est nilpotente. Montrer que :
\begin{enumerate}
\item[a)] $V_i \subset \ker (T - \lambda_i I)^{a_i}$ pour un entier $a_i$ tel que $N_i^{a_i} = 0$.
\item[b)] Les $\lambda_1, ..., \lambda_k$ sont des valeurs propres (pas forcément distinctes) de $T$. (\textit{Indice} : Utiliser par exemple le premier point).
\item[c)] Le polynôme $f(x) = \prod_{i=1}^{k} (x - \lambda_i)^{a_i}$ annule $T$. (\textit{Indice} : Montrer que $f(T)v = 0$ pour tout $v \in V$ en utilisant la décomposition de $V$ et le premier point).
\item[d)] En déduire que l'ensemble $\{ \lambda_1, ..., \lambda_q \}$ contient toutes les valeurs propres de $T$ (\textit{Indice} : Si $v \neq 0$ est un vecteur propre de $T$ de valeur propres $\lambda$, exprimer $f(T)v$ en fonction de $f$, $\lambda$, et $v$). 
\item[e)] En déduire que les valeurs sur la diagonale de n'importe quelle forme normale de Jordan de $T$ constituent l'ensemble des valeurs propres de $T$. (\textit{Indice} : Utiliser l'exercice 2.)
\end{enumerate}

\item Comparer les polynômes caractéristiques de $J$ et $A$. En déduire que les éléments diagonaux de $J$ contiennent exactement l'ensemble des valeurs propres de $A$ et le nombre d'apparitions de chaque valeur propre sur la diagonale de $J$ est égale à la multiplicité algébrique de ladite valeur propre.

\item Soit $A \in \Bbb C^{n \times n}$ et soient $J$ une forme normale de Jordan de $A$, $P$ la matrice de passage associée ($A = PJP^{-1}$). \newline 
Le but de cet exercice est de montrer que le nombre de blocs de Jordan sur $J$ associé à une valeur propre $\lambda$ est exactement $\dim \ker(A - \lambda I)$.
\begin{enumerate}

\item[a)] Soit $S = \begin{pmatrix} S_1 & 0 \\ 0 & S_2 \end{pmatrix}$ une matrice blocs diagonale. Montrer que $$\text{rang}(S) = \text{rang}(S_1) + \text{rang}(S_2)$$ 
Généraliser pour $p$ blocs sur la diagonale. (\textit{Indice} : Considérer les lignes linéairement indépendantes de $S_1, S_2$).
\item[b)] Soit $B = U + \lambda I \in \Bbb C^{q \times q}$ un bloc de Jordan, où $U$ est l'application de décalage. Montrez que la seule valeur propre de $B$ est $\lambda$ et que l'espace propre associé est engendré par $e_1$. Déduisez $\dim \ker (B - \lambda I) = 1$ et $\dim \text{Im} (B - \lambda I) = q-1$.
\item[c)] Soient $B_1, ..., B_k$ l'ensemble des blocs de Jordan sur $J$ associé à une valeur propre $\lambda$. Déduire de a) et b) que $\dim \text{Im} (J - \lambda I) = n - k$.
\item[d)] En déduire que $\dim \ker (A - \lambda I) = k$ et que $\ker (A - \lambda I) = \text{span}(Pe_{i_1}, ..., Pe_{i_k})$, où les $i_j$ sont les indices des premières lignes/colonnes des $B_1, ..., B_k$ dans $J$.
\end{enumerate}

\item Déduire des exercices $4$ et $5$ que si $A$ est diagonalisable, la forme normale de Jordan $J$ de $A$ est diagonale.

\item Soit $A \in \Bbb C^{n \times n}$ une matrice diagonalisable. Montrer que $\ker(A - \lambda I) = \ker(A - \lambda I)^k$ pour toute valeur propre $\lambda$ de $A$ et pour tout $k > 0$. (\textit{Indice} : Diagonaliser d'abord $(A - \lambda I)$ et $(A - \lambda I)^k$ de manière simultanée).

\item Trouver deux matrices $A \in \Bbb C^{n \times n}$ et $B \in \Bbb C^{n \times n}$ qui ont le même polynôme caractéristique, mais qui ne sont pas similaires (\emph{Rappel} : $A$ et $B$ sont dites similaires s'il existe une matrice $P$ inversible telle que $B = P^{-1} A P$).

\item Soit $J \in \Bbb C^{n \times n}$ une matrice en forme normale de Jordan. Montrer que $J$ et $J^T$ sont similaires. En déduire que pour tout $A \in \Bbb C^{n \times n}$, les matrices $A$ et $A^T$ sont similaires.
\end{enumerate}
%%% Local Variables:
%%% mode: latex
%%% TeX-master: "notes"
%%% End:


\bibliographystyle{alpha}
\bibliography{books}
\end{document}


%%% Local Variables:
%%% mode: latex
%%% TeX-master: t
%%% End:


