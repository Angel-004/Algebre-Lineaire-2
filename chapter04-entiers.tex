\chapter{Algèbre linéaire sur les entiers}
\label{cha:algebre-lineaire-sur}


Quand est-ce que un système d'équations linéaires possède une solution en nombre entiers? Étant donnés $A ∈ℤ^{m × n}$ et $b ∈ℤ^m$, on aimerait décider si 
\begin{equation}
  \label{eq:40}
  A x = b, \, x ∈ ℤ^n 
\end{equation}
est soluble et trouver tous les solutions.


Une condition nécessaire et suffisante pour \eqref{eq:40} étant soluble est qu'il existe une solution $x ∈ℚ^n$. et alors que $\rank(A) = \rank(A|b)$. Aussi on peut supprimer une ligne de $A$ qui est dans le span des autres lignes sans changer l'ensemble des solutions. Alors on peut supposer que $A$ est de plein rang lignes, c'est à dire $\rank(A) = m$. 

\begin{definition}
  \label{def:44}
  Un nombre entier $d ∈ℤ$ \emph{divise} un nombre entier $a ∈ℤ$ s'il existe un nombre entier $x ∈ℤ$ tel que $d⋅x =b$. On écrit $d\mid b$ et si $d$ ne divise pas $b$ on écrit $d \nmid b$. 
\end{definition}
% Si $d \mid a$, alors s'il existe $x ∈ℤ$ tel que $d ⋅x =a$ et si $a \neq 0$ on peut conclure $d ≤ |a|$, parce que $|a| = |d| ⋅|x|$. 
\begin{definition}
  \label{def:45}
  Un nombre $d ∈ℤ$ est un diviseur commun de $a∈ℤ$ et $b ∈ℤ$, si $d \mid a$ et $d \mid b$. Si $\max\{|a|,|b|\} ≥1$, l'ensemble des diviseurs commun de $a$ et $b$ est un ensemble fini. Dans ce cas, on dénote le \emph{plus grand diviseur commun} de $a$ et $b$ par $\gcd(a,b)$. 
\end{definition}


\begin{theorem}
  \label{thr:48}
  Soient $a,b ∈ℤ$ et $\max\{|a|,|b|\} ≥1$. On a
  \begin{displaymath}
    \gcd(a,b) = \min \{ x ⋅a + y ⋅ b: x,y ∈ℤ,  x ⋅a + y ⋅ b≥1\}.
  \end{displaymath}
\end{theorem}

\begin{proof}
  Soit $d$ un diviseur commun  positif de $a$ et $b$ (si $d$ est négatif, on a trivialement $d \le  \min\{ x ⋅a + y ⋅ b: x,y ∈ℤ,  x ⋅a + y ⋅ b≥1\}$). Alors il existent $x^*, y^* ∈ℤ$ tel que $a = d⋅x^*$ et $b = d ⋅y^*$. Si $x ⋅a + y ⋅ b≥1$ où $x,y ∈ℤ$, alors
  \begin{displaymath}
    x ⋅a + y ⋅ b = (x ⋅x^* + y ⋅ y^*)  d ≥ d. 
  \end{displaymath}
  Finalement on a  $d \le \min\{ x ⋅a + y ⋅ b: x,y ∈ℤ,  x ⋅a + y ⋅ b≥1\}$.
 Montrons que $\min\{ x ⋅a + y ⋅ b: x,y ∈ℤ,  x ⋅a + y ⋅ b≥1\}$ est un diviseur commun de $a$ et $b$. Supposons que $\min \nmid a$. Alors
  division avec reste implique l'existence de $q,r ∈ℤ$ tel que  
  \begin{displaymath}
    a = q ⋅ \min + r \quad \text{ et } \quad 1 \leq r < \min.
  \end{displaymath}
  Soient $x,y ∈ℤ$ où $\min = x ⋅a + y ⋅b$, alors
  \begin{displaymath}
    1 ≤ r = a - q ⋅ (x ⋅a + y ⋅b) = (1-q⋅x) a - qy ⋅b.
  \end{displaymath}
  C'est plus petit que $\min$ ce qui est absurde. C'est à dire $\min \mid a$. De façon analogue, on obtient $\min \mid b$.
  Aussi, on a montré que pour tout diviseur commun $d$ de $a$ et $b$, $\min\{ x ⋅a + y ⋅ b: x,y ∈ℤ,  x ⋅a + y ⋅ b≥1\}\geq d$ et donc $\min\{ x ⋅a + y ⋅ b: x,y ∈ℤ,  x ⋅a + y ⋅ b≥1\}= \gcd(a,b)$
  
\end{proof}



\begin{corollary}
  \label{co:5}
  Soient $a,b ∈ℤ$ et $\max\{|a|,|b|\} ≥1$. Le $\gcd(a,b)$ est le diviseur commun  positif qui est divisé par chaque diviseur commun de $a$ et $b$. 
\end{corollary}

Pour calculer le diviseur le plus grand de $a$ et $b$ on peut utiliser l'algorithme d'Euclide. Soient $a_o≥a_1 ∈ℤ$ pas tout les deux égaux à zéro. Si $a_1 = 0$, alors
\begin{displaymath}
\gcd(a_0,a_1) =   a_0. 
\end{displaymath}
Autrement on applique division avec reste
\begin{displaymath}
  a_0 = q_1 a_1 + a_2, 
\end{displaymath}
où $q_1,a_2 ∈ℤ$ et $0 ≤ a_2<a_1$. Un nombre entier $d ∈ℤ$ est un diviseur commun de $a_0$ et $a_1$ si et seulement si $d$ est un diviseur commun de $a_1$ et $a_2$. L'algorithme d'Euclide est la procédé de calculer la suite $a_0,a_1,a_2,\dots,a_{k-1},a_k ∈ℤ$  où $a_{k-1}>0$, $a_k=0$ et 
\begin{displaymath}
  a_{i-1} = q_i a_i + a_{i+1} 
\end{displaymath}
est le résultat de la division avec reste de $a_{i-1} $ par $a_i$. 

\begin{example}
  \label{exe:25}
  On calcule  le diviseur commun le plus grand de $a_0 = 52$ et $a_1=22$:
  \begin{displaymath}
    52 =   2 ⋅ 22 + 8, \quad 22 = 2 ⋅ 8  + 6, \quad 8 = 1 ⋅ 6 +2,\quad 6 = 3 ⋅2 + 0. 
  \end{displaymath}
  La suite est alors $a_{0}= 52, a_{1}= 22, a_{3} = 8, a_{4} = 6, a_{5} = 2, a_{6} = 0$. Alors $\gcd(52,22) = 2$. 
\end{example}

La calcule des suites $a_i$ et $q_i$ donne aussi une représentation $\gcd(a_0,a_1) = x ⋅a_0 + y ⋅a_1$, $x,y ∈ℤ$. Parce qu'on observe
\begin{displaymath}
  \begin{pmatrix}
    a_{i} \\  a_{i+1} 
  \end{pmatrix}
  =
  \begin{pmatrix}
    0 & 1 \\
    1 & -q_i
  \end{pmatrix}
  \begin{pmatrix}
     a_{i-1} \\  a_{i}
   \end{pmatrix}
\end{displaymath}
et alors
\begin{displaymath}
  \begin{pmatrix}
    a_{k-1} \\ a_k
  \end{pmatrix} =
  \begin{pmatrix}
    0 & 1 \\
    1 & -q_{k-1}
  \end{pmatrix} \cdots
  \begin{pmatrix}
    0 & 1 \\
    1 & -q_{1}
  \end{pmatrix}
  \begin{pmatrix}
    a_0 \\ a_1
  \end{pmatrix}
\end{displaymath}
\begin{example}
  \label{exe:26}
  On continue l'exemple~\ref{exe:25}.
  \begin{eqnarray*} 
    \begin{pmatrix}
      2 \\ 0
    \end{pmatrix} & = & 
    \begin{pmatrix}
      0 &1 \\
      1 & -3
    \end{pmatrix}
    \begin{pmatrix}
      0 &1 \\
      1 & -1
    \end{pmatrix}
    \begin{pmatrix}
      0 &1 \\
      1 & -2
    \end{pmatrix}
    \begin{pmatrix}
      0 &1 \\
      1 & -2
    \end{pmatrix}
    \begin{pmatrix}
     5 \\ 22
    \end{pmatrix} \\
   &  = & 
  \begin{pmatrix}
    3 & -7 \\
    -11 & 26 
  \end{pmatrix}
  \begin{pmatrix}
    52 \\22
  \end{pmatrix}
  \end{eqnarray*}
  Alors $2 = \gcd(52,22) = 3 ⋅52 -7 ⋅22$. 
\end{example}

Nous pouvons alors résoudre le problème~\eqref{eq:40} dans le cas où $m=1$ et $n=2$.

\begin{theorem}
  \label{thr:49}
  Soient $a,b ∈ℤ$ pas zéro tous le deux et $c ∈ℤ$. L'équation
  \begin{equation}
    \label{eq:41}
    x ⋅a + y ⋅b = c, \, x,y ∈ℤ
  \end{equation}
  possède une solution si et seulement si $\gcd(a,b) \mid c$. 
\end{theorem}
\begin{proof}
  Soient $x',y' ∈ℤ$ tel que $x' a + y'b = \gcd(a,b)$. Si $\gcd(a,b) \mid c$ alors il existe un $z ∈ℤ$ tel que $z ⋅\gcd(a,b) = c$ et $z x' a + z y'b = c$ est une solution en nombre entiers de~\eqref{eq:41}.

  S'il existe une solution de~\eqref{eq:41}, alors chaque diviseur commun de $a$ et $b$ est aussi un diviseur de $c$. 
\end{proof}

\section*{Exercices}

\begin{enumerate}
\item Démontrer Corollaire~\ref{co:5}.
\item Soient $n≥2$ et $a_1,\dots,a_n∈ ℤ$ pas tous égaux à zéro. On défini $\gcd(a_1,\dots,a_n)$ comme le plus grand diviseur commun de $a_1,\dots,a_n$. Montrer: 
  \begin{enumerate}[i)]
    \item  $\gcd(a_1,\dots,a_n) =  \min\{x_1a_1+ \cdots + x_n a_n : x_1a_1+ \cdots + x_n a_n≥1, \, x_i ∈ℤ, \, i=1,\dots,n\}$. 
  \item $\gcd(a_1,\dots,a_n) = \gcd(\gcd(a_1,a_2), a_3, \dots, a_n)$ pour $n≥3$. 
  \end{enumerate}
\item Soit $a_0≥a_1≥\dots≥a_k=0$ la suite calculé par l'algorithme d'Euclide. Montrer $a_{i-1} ≥ 2⋅ a_{i+1}$ et conclure  que $k ≤2 ⋅ \log_2(a_0)+1$.  
\end{enumerate}



\section{La forme normale d'Hermite}
\label{sec:la-forme-normale-1}

Maintenant on s'occupe du problème~\eqref{eq:40} où $A ∈ ℤ^{m ×n}$ et $b ∈ℤ^m$ et $\rank(A) = m$.

\begin{lemma}
  \label{lem:23}
  Soit $A ∈ℤ^{n ×n}$ une matrice inversible (sur $ℚ$), alors $A^{-1} ∈ ℤ^{n ×n}$ si et seulement si $\det(A) = \pm 1$. 
\end{lemma}

\begin{proof}
  Supposons que $A^{-1} ∈ℤ^{n ×n}$. Alors $1 = \det(I_n)  = \det(A^{-1}) \det(A)$. Les deux facteurs $\det(A^{-1})$ et $ \det(A)$ sont des nombres entiers. Les seul diviseurs de $1$ en nombre entiers sont $1$ et $-1$.

  Autrement, si $\det(A) = \pm 1$, on a
  \begin{displaymath}
    A^{-1} = \mathrm{ad}(A) / \det(A) ∈ℤ^{n ×n}. 
  \end{displaymath}
  où $\mathrm{ad}(A) ∈ ℤ^{n ×n}$ est la matrice adjoint de $A$. On se rappelle que $(\mathrm{ad}(A))_{ij} = (-1)^{i+j}\det(A_{ji})$ où $A_{ji}∈ℤ^{(n-1)×(n-1)}$ est la matrice qu'on obtient de $A$ en supprimant la $j$-ème ligne et $i$-ème colonne. 
  \end{proof}

  \begin{definition}
    \label{def:46}
    Une matrice $U ∈ℤ^{n ×n}$ tel que $\det(U) = \pm 1$ est appelée  \emph{unimodulaire}.
  \end{definition}

  \begin{remark}
    Soit $U ∈ℤ^{n ×n}$  une matrice unimodulaire. 
    Un $x^* ∈ ℤ^n$ est une solution du problème~\eqref{eq:40} si et seulement $U^{-1} x^*∈ ℤ^n$ est une solution du problème
    \begin{equation}
      \label{eq:42}
      A U x = b, \, x ∈ℤ^n. 
    \end{equation}
  \end{remark}
  %
L'idée est maintenant de trouver une matrice unimodulaire $U ∈ℤ^{n ×n}$ tel que
\begin{equation}
  \label{eq:44}
  A ⋅ U = [H | 0]
\end{equation}
où
\begin{displaymath}
  H =
  \begin{pmatrix}
    h_{11} \\
    h_{21} & h_{22}\\
    &  & \ddots \\
    h_{m1} & \hdots & \hdots & h_{mm}
  \end{pmatrix}
\end{displaymath}
est une matrice triangulaire. Le problème~\eqref{eq:40} alors est soluble si et seulement si
$H^{-1} b ∈ℤ^n$.

\begin{definition}
  \label{def:47}
  Soit $A ∈ ℤ^{m ×n}$ une matrice. Une \emph{opération élémentaire unimodulaire} est une des trois
  \begin{enumerate}[i)]
    \item Multiplier une colonne avec $-1$.  \label{item:21}
    \item Échanger deux colonnes de $A$.  \label{item:22}
    \item Additionner un multiple entier d'une colonne de $A$ sur une autre colonne de $A$. \label{item:23}
  \end{enumerate}
\end{definition}



\begin{example}
  \label{exe:27}
  Une suite d'opérations élémentaire unimodulaire sur $A$ correspond à la multiplication $A⋅U$ où $U ∈ℤ^{n ×n}$ est une matrice unimodulaire.
Soit 
  \begin{displaymath}
    A =
    \begin{pmatrix}
      3 & 6 & 2 \\
      11 & 5 & 7
    \end{pmatrix}. 
  \end{displaymath}
  Échanger les colonnes $1$ et $2$ correspond à la multiplication de $A$ avec la matrice unimodulaire 
  \begin{displaymath}
    \begin{pmatrix}
    0& 1 & 0 \\
    1 & 0 & 0 \\
    0 & 0 & 1
      
    \end{pmatrix}
  \end{displaymath} à droite
  \begin{displaymath}
     \begin{pmatrix}
      6  & 3& 2 \\
       5 & 11 & 7
     \end{pmatrix} = A ⋅
     \begin{pmatrix}
       0& 1 & 0 \\
       1 & 0 & 0 \\
       0 & 0 & 1
     \end{pmatrix}
   \end{displaymath}
   Additionner $-3$ fois la colonne $3$ sur la colonne $1$ est la multiplication de $A$ avec la matrice unimodulaire 
   \begin{displaymath}
     \begin{pmatrix}       
     1 & 0 & 0\\ 
     0 & 1 & 0 \\
     -3 & 0 & 1
   \end{pmatrix}
 \end{displaymath} 
   \begin{displaymath}
     \begin{pmatrix}
      0  & 3& 2 \\
      -16 & 11 & 7
    \end{pmatrix} = A ⋅  \begin{pmatrix}
      
     1 & 0 & 0\\
     0 & 1 & 0 \\ 
     -3 & 0 & 1
   \end{pmatrix}
 \end{displaymath}
\end{example}


\begin{example}
  \label{exe:28}
  Est-ce que
  \begin{equation}
    \label{eq:43}
     \begin{pmatrix}
      3 & 6 & 2 \\
      11 & 5 & 10
    \end{pmatrix} x =
    \begin{pmatrix}
      2 \\ 2
    \end{pmatrix}, x ∈ℤ^3
      \end{equation}
  possède une solution.
  Soustraire  colonne $3$ de colonne $1$:
  \begin{displaymath}
     \begin{pmatrix}
      1 & 6 & 2 \\
      1 & 5 & 10
    \end{pmatrix}
  \end{displaymath}
  Soustraire six fois colonne une de colonne deux et deux fois colonne une de colonne trois:
 \begin{displaymath}
     \begin{pmatrix}
      1 & 0 & 0 \\
      1 & -1 & 8
    \end{pmatrix}
  \end{displaymath}
  Additionner huit fois colonne deux sur colonne trois:
  \begin{displaymath}
  \begin{pmatrix}
      1 & 0 & 0 \\
      1 & -1 & 0
    \end{pmatrix}. 
  \end{displaymath}      
    Alors on peut conclure que \eqref{eq:43} possède une solution entière. En fait
    \begin{displaymath}
    \label{eq:43}
     \begin{pmatrix}
      3 & 6 & 2 \\
      11 & 5 & 10
    \end{pmatrix} x =
    b, x ∈ℤ^3
  \end{displaymath} est soluble pour chaque $b ∈ℤ^2$. 
\end{example}


\begin{lemma}
  \label{lem:24}
  Soit  $A ∈ℤ^{ m ×n}$ une matrice en nombre entiers, alors il  existe une matrice unimudulaire $U ∈ℤ^{ n ×n}$ tel que le première ligne de $AU$ est de la forme $(d,0,\cdots,0)$, où $d ∈ℤ$.
\end{lemma}

\begin{proof}
On note $\begin{matrix}  A =& \begin{pmatrix}  l_1  \\ ...  \\l_m   \end{pmatrix}  \end{matrix}$  et $l_1=\begin{pmatrix}  a_1 &... & a_n  \end{pmatrix}$.
$ \\ $
$ \\ $
On raisonne par récurrence sur $\sum_{i=0}^{n}{|a_i|}$
$ \\ $$ \\ $
$\underline{k=0}$ alors $\forall i \in \mathbb{N}_n$, $a_i=0$ et l'assertion est trivialement vraie.
$ \\ $
$\underline{k>0}$ alors $\exists j_1 \neq j_2 \in \mathbb{N}_n$ tels que $a_{j_1} \neq 0$ et $a_{j_2} \neq 0$.
$ \\ $
On peut alors supposer $|a_{j_1}| \ge |a_{j_2}|$. Ainsi en effectuant la division euclidienne de $a_{j_1}$ par $a_{j_2}$ on obtient 
$ \\ $ 
$a_{j_1}=q \times a_{j_2}+r$ avec $q, r \in \mathbb{Z}$ et $0 \le r < |a_{j_2}|$
$ \\ $ 
Dès lors on applique l'opération unimodulaire \emph{Soustraire $q$ fois la colonne $j_2$ à colonne $j_1$} ce qui remplace alors $a_{j_1}$ par $r$ et laisse les autres composantes de la première lignes inchangées.
Ainsi on a on a $\sum_{i=0\\i\neq j_1}^{n}{|a_i|} + r < \sum_{i=0}^{n}{|a_i|}=k$ et on peut appliquer notre hypothèse de récurrence.

  \end{proof}

  \begin{corollary}
    \label{co:9}
    Soit  $A ∈ℤ^{ m ×n}$ une matrice en nombre entiers alors il  existe une matrice unimudulaire $U ∈ℤ^{ n ×n}$ tel que $A ⋅U$ est de la forme~\eqref{eq:44}.
  \end{corollary}
  \begin{proof}
    Par induction sur $m$. le cas $m=1$ est impliqué par Lemme~\ref{lem:24}. Soit $m>1$. 
    Lemme~\ref{lem:24} implique qu'il existe une matrice unimodulaire $U_1 ∈ℤ^{n ×n}$ tel que
    \begin{displaymath}
      A ⋅ U_1 =
      \begin{pmatrix}
        d & 0 \cdots 0 \\
        a  & A'
      \end{pmatrix}
    \end{displaymath}
    où $d ∈ℤ$, $a ∈ ℤ^{m-1}$ et $A' ∈ ℤ^{(m-1) × (n-1)}$. Par l'hypothèse d'induction, il existe une matrice unimudulaire $U_2 ∈ℤ{(n-1) ×(n-1)}$ tel $A' U_2$ est de la forme désirée. Clairement
      \begin{displaymath}
         \begin{pmatrix}
          1 & 0^T \\
          0 & U_2
        \end{pmatrix} ∈ℤ^{n ×n}
      \end{displaymath}
      est une matrice unimodulaire et 
      \begin{displaymath}
        A U_1
        \begin{pmatrix}
          1 & 0^T \\
          0 & U_2
        \end{pmatrix}
      \end{displaymath}
      est de la  forme~\eqref{eq:44}. 
  \end{proof}



  \begin{definition}
    \label{def:49}
    Une matrice $A ∈ℤ^{m ×n}$ du rang $m$ est en \emph{forme normale de Hermite}, si elle est de la forme \eqref{eq:44}, où $r_{ii}>0$ pour tous $i$ et $0≤ r_{ij} < r_{ii}$ pour tout $j<i$. 
  \end{definition}
	
\begin{theorem}
  \label{thr:28}
    Soit  $A ∈ℤ^{ m ×n}$ une matrice en nombre entiers alors il  existe une matrice unimudulaire $U ∈ℤ^{ n ×n}$ tel que $A ⋅U$ est en \emph{forme normale de Hermite}.
  \end{theorem}
  
  \begin{definition}
    \label{def:49}
    Soit $A \in \mathbb{Z}^{m\times n}$ tq. $rang(A)=m$ alors on pose $\Lambda(A):=\left\{ Ax,x\in \mathbb{Z}^{n\times n} \right\}$ Le réseau entier de $A$. Aussi $U \in \mathbb{Z}^{n\times n}$ tq. $\Lambda(A)=\Lambda(B)$ est appelée base de $\Lambda(A)$.
     \end{definition}
	
 \begin{corollary}
    \label{co:9}
    Chaque réseau entier possède une base.
     \end{corollary}
     \begin{proof}
     On a $U ∈ℤ^{ n ×n}$ unimodulaire tq. $A ⋅ U = [H | 0]$ et $H$ est en forme normale de Hermite. Il est alors simple de remarquer que $\Lambda(A)=\Lambda(H)$. Dès lors H est une base du réseau entier de $A$.
     
     \end{proof}
      \begin{theorem}
  \label{thr:28}
  Soient $A,B \in \mathbb{Z}^{m\times n}$ tq. $rang(A)=rang(B)=m$. Alors $\Lambda(A)=\Lambda(B) \Leftrightarrow H_A=H_B$ ou $H_A$(resp $H_B$) dénote une forme normale de Hermite de $A$ (resp $B$).
  \end{theorem}
      

  
    
    
%%% Local Variables:
%%% mode: latex
%%% TeX-master: "notes"
%%% End:
